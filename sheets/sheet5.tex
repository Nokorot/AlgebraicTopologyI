\sheet 5

\DeclareMathOperator\map{map}
\def\op{\text{op}}


\begin{document}
\mmaketitle

\begin{exercise}[1]\ 
\end{exercise}

\begin{exercise}[2]\ 
We have $RM := R \smsh M$.

We define 
\begin{align*}
\Psi : \pi_k(R)[M] &\to \pi_k(RM) \\
\sum_i [f_i] \cdot m_i &\mapsto \sum_i [f_i \smsh m_i]
\end{align*}
where the sum on the left is the formal sum in the definition of the
monoid ring and the sum in the right is the sum in $\pi_k(RM)$ as an abelian
group.

It is clear from the definition that this defines a homomorphism of abelian
groups and, by prop. 3.12 in the notes, we have that 
\[ \pi_k(RM_k) = \pi_k(\vee_{m\in M} R_k) \cong \oplus_{m\in M} \pi_k(R_k), \]
and thus it is clear that $\Psi$ is also bijective. 
So all that remains it to show that it preserves the graded multiplication.

Let $\hat f = \sum_{i\in I} [f_i] \cdot m_i \in \pi_k(R)[M]$ and $\hat g =
\sum_{j\in J} [g_j] \cdot m_j \in \pi_l(R)[M]$. ($I$ and $J$ are finite indexing
sets. So in particular, we may wlog assume that $n$ is independent of $i$.)

Then 
\begin{equation}
\label{eq:5.2.a}
\Psi(\hat f) \cdot \Psi(\hat g) = (-1)^{nk} [\mu_{m,n} \circ (\til f
\smsh \til g)], 
\end{equation}
where $\til f: S^{m+k} \to R_m \smsh M_+$ and $\til g: S^{n+l} \to R_n \smsh
M_+$, such that 
\begin{equation*}
[\til f] = \sum_i [f_i \smsh m_i] = \Psi(\hat f) \q\tand 
[\til g] = \Psi(\hat g).
\end{equation*}
We may wlog assume that $\til f = (\vee_i f_i) \circ \eta^{|I|}$, where $\eta^{|I|}$ is a
$|I|$-fold pinch map and similarly for $g$.

On the other hand we have 
\[ \Psi\left(\hat f \cdot \hat g\right) = \sum_{i,j} (-1)^{nk} [\mu_{m,n} \circ (f_i \smsh
g_j) \smsh (m_i \cdot m_j)]. \] 
Since the addition on the right hand side is defined by pre-composition with a
$|I|+|J|$-fold pinch map, it is clear that this is equal to \pref{eq:5.2.a}.
Which concludes the proof.
\end{exercise}

\begin{exercise}[3]\ 
It is immediately clear that $\pi_*(R^\op) = (\pi_*(R))^\op$ considered as
abilian groups, so all we need to check is that the multiplication is the same. 

Let $f: S^{m+k} \to R_n$ and $g: S^{n+l}\to R_m$. Then the product of $[f]$ and
$[g]$ in $\pi_*(R^\op)$ is given by
\[ [f]\cdot^\op [g] := (-1)^{nk}[\mu^\op_{m,n} \circ (f\smsh g)]. \]
We consider the following diagram, where top row defines $[f]\cdot^\op [g]$. 
\[ \begin{tikzcd}
S^{m+k+n+l} \ar[r,"f\smsh g"] \ar[d, "\chi_{m+k,n+l}"] 
& R_m\smsh R_n \ar[r, "\mu^\op_{m,n}"] \ar[d, "\tau"] 
& R_{m+n} \ar[d, "\tau_*"] \\
S^{n+l+m+k} \ar[r, "g\smsh f"] &
R_n\smsh R_m \ar[r, "\mu_{n,m}"] & R_{n+m}
\end{tikzcd} \]
where the vertical map are the obvious twisting maps. 
It is clear that the left square commutes, and the right square commutes by the
definition of $\mu^\op$. So we have 

\begin{align*}
[f]\cdot^\op [g] &:= (-1)^{nk}[\mu^\op_{m,n} \circ (f\smsh g)] \\
&= (-1)^{nk}(-1)^{(m+k)(n+l)}(-1)^{mn} [\mu_{n,m}\circ (g\smsh f)] \\
&= (-1)^{nk}(-1)^{(m+k)(n+l)}(-1)^{mn} (-1)^ml [g]\cdot[f] \\
&= (-1)^{kl}[g]\cdot[f].
\end{align*}


\end{exercise}


\begin{exercise}[4]\ 
Let $f : S^{n+k} \to M_d(R)$. 
We have 
\[ M_d(R) := \map_*(d_+, R \smsh d_+) \cong \prod_{i=1}^{d}
\bigvee_{j=1}^{m} R. \] 
So we have
\[ f = \prod_{i=1}^m f_i \q{\text{for}} f_i : S^{n+k} \to R_n \smsh m_+ \cong \bigvee_{j=1}^{m} R.  \]

Furthermore, by prop. 3.12 in the notes, we may write 
\[ [f_i] = \sum_{j=1}^m [f_{i,j}] \q{\text{for some}} f_{i,j} : S^{n+k} \to R
\smsh j \cong R. \]
(Note that we might need to change/increase $n$, by this does not really make any
difference.) 

We define 
\begin{align*}
\Psi : \pi_k(M_d(R)) &\to M_m(\pi_k(R)) \\
[f] &\mapsto ([f_{i,j}])_{i,j}
\end{align*}

Again it is clear by the construction that $\Psi$ is an isomorphism of abelian
groups, so what we need to check, is that the graded multiplication is
preserved.

Let $f : S^{m+k} \to \map_*(d_+, \R_m\smsh d_+)$ and
$g : S^{n+l} \to \map_*(d_+, \R_n\smsh d_+)$. Then 
%
\begin{align*}
\Psi([f])\cdot \Psi([g]) &= \left(
\sum_{q=1}^d [f_{i,q}] \cdot [g_{q,j}] 
\right)_{i,j} 
= (-1)^{nk} \left(
\sum_{q=1}^d [\mu_{m,n} \circ (f_{i,q} \smsh g_{q,j})] 
\right)_{i,j}
\end{align*}
%
On the other hand we have 
\begin{align*}
[f] \cdot [g] &= (-1)^{nk} [\xi_{m,n} \circ (f \smsh g)]
\end{align*}
where $\xi_{m,n}$ is the multiplication defined on the matrix ring spectra, ie.
\[ \xi_{m,n};\quad a\smsh b \mapsto \left(d_+ \xto{a} R_n \smsh d_+ \xto{\R_n \smsh m} R_n
\smsh R_m \smsh d_+ \xto{\mu_{m,n}\smsh d_+} R_{n+m} \smsh d_+ \right), \]
for $a \in \map_*(d_+, R^n \smsh d_+)$ and $b \in \map_*(d_+,R_m\smsh d_+)$.

It follows that 
\[ \left(\xi_{m,n} \circ (f\smsh g) \right)_i = \xi_{m,n} \circ (f_i \circ g), \]
since the $i$'th component is by pre-composing with the inclusion $\{i\} \into d_+$.

Furthermore, we have that 
\[ \left(\xi_{m,n} \circ (f\smsh g)\right)_{i,j} = \mu_{m,n} \circ
(\bigvee_{q=1}^d (f_{i,q}\smsh g_{q,j})) \circ \eta^d.  \]
This is less clear, so let's take it in a couple steps. 
Firstly we get $i,j$'th component by pre-composing with $\{i\} \into d_+$ (as
above) and post-composing with the projection $d_+ \to \{j\}$. 
The part with the $d$-fold pinch map followed by acting on each of the
components, comes from the isomorphism from prop. 3.12.

With that it is clear that $\Psi([f]\cdot [g]) = \Psi([f]) \cdot \Psi([g])$. 
% And we are done
\end{exercise}



\end{document}
