\sheet 1

\begin{document}
\mmaketitle

\begin{exercise}[1 \& 2]\ 
% Let $Y$ be a pointed space, $m \ge 1$ and $f:S^m \to S^m$ a continuous based
% map of degree $k$. Show that for every homotopy class $x \in \pi_n(S^m \wedge Y
% )$ the classes $(f\wedge Y)^*(x)$ and $k\cdot x$ become equal in
% $\pi_{1+n}(S^{1+m}\wedge Y)$ after applying the suspension homomorphism.
% Give an example to show that $(f\wedge Y)^*(x)$ need not equal $k\cdot x$ in
% $\pi_n(S^m\wedge Y)$ before applying the suspension homomorphism.

We will denote the degree $k$ map, by $f_m : S^m \to S^m$. 

Let $x = [\phi : S^n \to S^m \wedge Y]$. 
We consider the (no necessarily commuting) diagram 
\[ \begin{tikzcd}
S^n \arrow[r, "\phi"] \arrow[d, "f_n"]& S^m \wedge Y \arrow[d, "f_m \wedge Y"]
\\
S^n \arrow[r, "\phi"] & S^m \wedge Y
\end{tikzcd}  \]
Where $[ (f_m \wedge Y) \circ \phi ] = (f\wedge Y)_*(x)$ and $[ \phi \circ f_n ]
= k \cdot x$. 

Applying the suspension functor, this diagram is sent to 
\[ \begin{tikzcd}
S^1 \wedge S^n \arrow[r, "S^1 \wedge \phi"] \arrow[d, "S^1 \wedge f_n"]& S^1 \wedge S^m \wedge Y \arrow[d, "S^1 \wedge f_m \wedge Y"]
\\
S^1 \wedge S^n \arrow[r, "S^1 \wedge \phi"] & S^1 \wedge S^m \wedge Y
\end{tikzcd}  \]
We want to show that this diagram commutes up to homotopy.

We observe that the map $S^m \wedge f_1 : S^{m+1} \to S^{m+1}$ is a $k$-fold
map. Since $(S^m \wedge -)_*$ is isomorphism of homology groups, we have $(S^m
\wedge -)_*([f_1]) = [S^m \wedge f_1] = k \in H_{m+1}$, so by Hurewicz theorem $[S^m \wedge f_1] = k \in \pi_{m+1}(S^{m+1})$.
So we may wlog assume $f_m = S^{m-1} \wedge f_1$. 
Furthermore the map $\tau : S^1 \wedge S^1 \to S^1 \wedge S^1$, which swaps the two
spheres, is base homotopic to the identify. So identifying $S^n$ with
$\Wedge_{i=1}^n S^1$, the diagram can be written, up to homotopy, as 
\[ \begin{tikzcd}
S^1 \wedge S^n \arrow[r, "S^1 \wedge \phi"] \arrow[d, "f_1 \wedge S^n"]& S^1 \wedge S^m \wedge Y \arrow[d, " f_1 \wedge S^m \wedge Y"]
\\
S^1 \wedge S^n \arrow[r, "S^1 \wedge \phi"] & S^1 \wedge S^m \wedge Y
\end{tikzcd}  \]
which clearly commutes.

For the counter example we consider the example given in question 2. 
In question 2 we have the commutative diagram

\begin{equation}
\label{dgr:1}
\begin{tikzcd}
S^3 \arrow[r, "\eta"] \arrow[d, "g"] & \C P^1 \arrow[d, "f"] \\
S^3 \arrow[r, "\eta"] & \C P^1 
\end{tikzcd}
\end{equation}

The map $f$ has degree $-1$ and the map $g$ had degree $1$. So if we suppose 
the two classes are the same without suspension, then 
\[ -[\eta] = (f \wedge S^0)_* ([\eta]) = [ \eta \circ g ] = [\eta] \quad\in
\pi_3(S^2, *), \]
where the second equality follows form the commutativity of the diagram
\pref{dgr:1}. However $-[\eta] \ne [\eta]$ in $\pi_3(S^2, *) = \Z$, so this is a contradiction.

On the other hand, by suspending the diagram \pref{dgr:1} and using the fact
from exercise 1, that $\Sigma_* (-[\eta]) = \Sigma_* ( (f \wedge S^0)_* ([\eta])
)$, we get that 
\[ -\Sigma_*[\eta] = \Sigma_* [ \eta \circ g ] = \Sigma_* [\eta]
\quad\in\pi_4(S^3,*). \]
So we may conclude that $2 \cdot (\Sigma_*[\eta]) = 0 \in \pi_4(S^3,*)$.
\end{exercise}

\begin{exercise}[3]\

(a) Consider $f:S^{m+k}\to S^m$ and $g:S^{n+l}\to X_n$. By definition of stable homotopy groups $[f\cdot g]=[\sigma_1\circ(S^1\smsh(f\cdot g))]$. First we check that this equals both $[(S^1\smsh f)\cdot g]$ and $(-1)^k[f\cdot(\sigma_1\circ(S^1\smsh g))]$. The first one is because
\[\sigma_1\circ(S^1\smsh(f\cdot g))=\sigma_1\circ(S^1\smsh\sigma^n)\circ(S^1\smsh f\smsh g)=\sigma^{n+1}\circ(S^1\smsh f\smsh g)=(S^1\smsh f)\cdot g.\]
For the second one, observe that in the following diagram
\[
\begin{tikzcd}
S^1\smsh S^{m+k}\smsh S^{n+l} \ar[r,"S^1\smsh f\smsh g"] & S^1\smsh S^m\smsh X_n \ar[d,"\tau_2\smsh X_n"] \ar[r,"S^1\smsh\sigma^m"] & S^1\smsh X_{n+m} \ar[r,"\sigma_1"] & X_{n+m+1}\\
S^{m+k}\smsh S^1\smsh S^{n+l} \ar[u,"\tau_1"] \ar[r,"f\smsh S^1\smsh g"] & S^m\smsh S^1\smsh X_n \ar[r,"S^m\smsh\sigma_1"] & S^m\smsh X_{n+1} \ar[r,"\sigma^m"] & X_{n+m+1}
\end{tikzcd}
\]
where $\tau_1$ and $\tau_2$ are the obvious homeomorphisms induced by swapping factors in the smash product, the composition of the bottom row is $f\cdot(\sigma_1\circ(S^1\smsh g))$, while the top row is $\sigma_1\circ(S^1\smsh(f\cdot g))$. The maps $\tau_1$ and $\tau_2$ are of degree respectively $(-1)^{m+k}$ and $(-1)^m$, and the commutativity of the left square is clear. Observe that the map $S^1\smsh f\smsh g$ is a suspension, so it induces a morphism on homotopy groups by precomposition. This shows that $[S^1\smsh f\smsh g]=(-1)^k[f\smsh S^1\smsh g]$ in the group $\pi_{m+k+n+l+1}(S^{m+1}\smsh X_n)$. By definition of $\sigma^n$ we have $\sigma_1\circ S^1\smsh\sigma^m=\sigma^m\circ S^m\smsh\sigma_1$ and combining with the previous equality we get $[\sigma_1\circ(S^1\smsh(f\cdot g))]=(-1)^k[f\cdot(\sigma_1\circ(S^1\smsh g))]$.

If now we have different maps $f':S^{m'+k}\to S^{m'}$ and $g':S^{n'+l}\to X_{n'}$ representing $[f]$ and $[g]$ respectively in $\pi_k(\mathbb S)$ and $\pi_l(X)$, we can take representatives of all everything $(f,g,f',g')$ in an advanced enough point of the spectra by smashing with $S^1$ as many times as needed. This process does not change $[f]\cdot[g]$ or $[f']\cdot[g']$ by what we have seen above. Hence we may assume wlog that $m=m'$, $n=n'$ and $f\simeq f'$ and $g\simeq g'$. But this implies $f\smsh g\simeq f'\smsh g'$, thus $[f\cdot g]=[f'\cdot g']$ and so we have proved well-definedness.

(c) Let $S^m$ be the identity on the $m$-sphere. Then
\[[S^m]\cdot[g]=[S^m\cdot g]=[\sigma^m\circ(S^m\smsh g)]=[g]\]
where the first equality comes from the definition of the pairing, the second equality is just by definition of $S^m\cdot g$ and the third equality is by definition of the stable homotopy groups.

(b), (d), (e) missing for lack of time...

\end{exercise}

\end{document}
