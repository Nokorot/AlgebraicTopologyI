\sheet 8

\def\op{\text{op}}

\begin{document}
\mmaketitle

\begin{exercise}[1]\ 

\end{exercise}


\begin{exercise}[2]\ 
Suppose $C$ is a preadditive category. Then 
for any $f,g : A \to B$, define
\[ f^\op +_\op g^\op := (f + g)^\op. \]
Given $f,g : A \to B$ and $h,k : B \to C$, we have
\begin{align*}
h^\op \circ_\op (f^\op +_\op g^\op)  
&= ((f+g)\circ h)^\op \\
&= ((f\circ h) + (g\circ h))^\op \\
&= (h^\op \circ_\op f^\op) +_\op (h^\op \circ_\op g^\op)
\end{align*}
And similarly 
\[ (h^\op + k^\op) \circ_\op f^\op = (h^\op \circ_\op f^\op) +_\op (k^\op \circ_\op
f^\op) \]

Hence $C^\op$ is preadditive. 
Furthermore, if $C$ is additive, i. e. has finite coproducts. 
Since the coproduct is also a product, it also defines a coproduct in the
opposite category, and thus $C^\op$ is additive.
\end{exercise}


\begin{exercise}[3]\ 

\begin{enumerate}
\item[(T0)] 
If $(-f',-g',-h')$ is isomorphic to $(-f,-g,-h)$ (anti-distinguished) by the morphism $(a,b,c)$, then 
$(A,b,c)$ also defines an isomorphism from $(f', g',h')$ to $(f,g,h)$. So
$(f',g',h')$ is distinguished and thus $(-f', -g',-h')$ is anti-distinguished.

\item[(T1)]
For any morphism $-f$, there exists a distinguished $(f,g,h)$. So $(-f,-g,-h)$
is anti-distinguished.

\item[(T2)]
$(0,-\id_X,0)$ is isomorphic to $(0,\id_X,0)$ by $(0,\id_X, -\id_X)$. 
So $(0,-\id_X,0)$ is distinguished and thus $(0,\id_X,0)$ is
anti-distinguished.

\item[(T3)]
If $(f,g,h)$ is distinguished, then $(g,h,-\Sigma f)$ is distinguished. So 
$(-g,-h,\Sigma f)$ is anti-distinguished.

\item[(T4)]
This follows immediately from the T4 property of $\T$, and the fact that 
a square commutes iff and only if the square commutes after changing the sign of
the horizontal (or vertical) maps. ie.
\[ 
(-b) \circ f = f' \circ (-a)
\qq{\Leftrightarrow}
b \circ (-f) = (-f') \circ a
\qq{\Leftrightarrow}
b \circ f = f' \circ a \] 

\item[(T5)]
Let $f:A\to B$ and $f':B \to D$ be morphisms, then by (T5) (with $-f$ and $-f'$) for $\T$, we we can find the following diagram
%
\[ \begin{tikzcd}
A \ar[r, "-f"] \ar[d, equal] &
B \ar[r, "-g"] \ar[d, "-f'"] &
C \ar[r, "-h"] \ar[d, "x"] &
\Sigma A      \ar[d, equal]
\\
A \ar[r, "f'f"]              &
D \ar[r, "-g'"]  \ar[d, "-g''"] &
E \ar[r, "-h''"] \ar[d, "y"]  &
\Sigma A        \ar[d, "-\Sigma f"]
\\
&
F \ar[r, equal] \ar[d, "-h'"] &
F \ar[r, "-h'"] \ar[d, "(\Sigma g)\circ h'"]  &
\Sigma B    
\\
&
\Sigma B \ar[r, "-\Sigma g"] &
\Sigma C &
\end{tikzcd} \]
such that $(-f,-g,-h), (f'f,-g',-h''), (-f',-g'',-h')$ and $(x,y,(\Sigma
g)\circ h')$ are distinguished. 

By changing some of the signs an even number of signs in each square, we
preserve the commutativity. Doing so, we get the following commutative diagram

\[ \begin{tikzcd}
A \ar[r, "f"] \ar[d, equal] &
B \ar[r, "g"] \ar[d, "f'"] &
C \ar[r, "h"] \ar[d, "-x"] &
\Sigma A      \ar[d, equal]
\\
A \ar[r, "f'f"]              &
D \ar[r, "g'"]  \ar[d, "g''"] &
E \ar[r, "-h''"] \ar[d, "y"]  &
\Sigma A        \ar[d, "\Sigma f"]
\\
&
F \ar[r, equal] \ar[d, "h'"] &
F \ar[r, "h'"] \ar[d, "(\Sigma g)\circ h'"]  &
\Sigma B
\\
&
\Sigma B \ar[r, "\Sigma g"] &
\Sigma C &
\end{tikzcd} \]

Since $(-f,-g,-h)$ and $(-f',-g'',-h')$ were distinguished, $(f,g,h)$ and
$(f',g'',h')$ are anti-distinguished. 

Furthermore, also $(-f'f, -g', h'')$ is isomorphic to $(f'f,-g', -h'')$
(which is distinguished) by $(-\id, \id, \id)$, so also $(f'f, g', -h'')$ is anti-distinguished.
Similarly $(x,-y,-(\Sigma g)\circ h')$ is isomorphic to $(x,y,(\Sigma g)\circ
h')$ by $(\id,\id,-\id)$, so $(-x,y,((\Sigma g)\circ h'))$ is
anti-distinguished.

Which shows $(T5)$ for the class of anti-distinguished triangles.
\end{enumerate}

\end{exercise}



\end{document}
