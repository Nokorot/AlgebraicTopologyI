\sheet 2

\def \L{\mathbb{L}}
\def \simto{\xrightarrow{\sim}}
\def \O{\mathcal{O}}

\begin{document}
\mmaketitle

\begin{exercise}[1]\ 
\end{exercise}

\begin{exercise}[2]\ 
\begin{enumerate}
\item[(a)]
Fix a linear isometry $\psi: \R^n \simto W$. Let 
$q : X_n \to X^\flat(W)$ denote the quotient map 
\[ x \mapsto [\psi \wedge x]. \]
and define $\eta: X^\flat(W) \to X_n$ by 
\[ [\phi \wedge x] \mapsto (\psi^{-1} \circ \phi) . x. \]
Where $(\psi^{-1} \circ \phi)$ is consider as an element in $O(n)$ action on
$x$.

Then, for $x \in X^n$ and $\phi \in \L(\R^n, W)_+$ 
\begin{align*}
(q \circ \eta) ([x \wedge \phi]) 
&= [\psi \wedge (\psi^{-1} \circ \phi) . x ] 
= [\psi \circ \psi^{-1} \circ \phi \wedge x]
= [\phi \wedge x], \q{\tand} \\
(\eta \circ q) (x) 
&= (\psi^{-1} \circ \psi) . x 
= x
\end{align*}
Clearly both $q$ and $\eta$ are continues, proving the clime.

Furthermore, we note that both maps are base point preserving.

\item[(b)]
We denote the map in the exercise by $\mu$ and
let $A \in O(n)$, and $x, \psi, \phi$ as in the exercise, then 
\begin{align*}
\mu( \psi \wedge (\phi \circ A) \wedge x ) 
&= (\psi \circ \phi \circ A) \wedge x  \\
&\sim (\psi \circ \phi) \wedge A . x \\
&= \mu( \psi \wedge \phi \wedge A . x ) 
\end{align*}
So the map to descend ('factor' through) to the quotient. 

\item[(c)]
% We have the following diagram 
% \[ \begin{tikzcd}
% S^V \wedge \L(\R, W)_+ \wegde X_n \arrow[r]
% & \L(\R^{m+n}, V \oplus W)_+ \wedge X_{m+n} \\
% v \wedge \phi \wedge x \arrow[r, mapsto]
% & (\psi \oplus \phi) \wedge \sigma^m(\psi^{-1}(v)\wedge x)
% \end{tikzcd} \]

We have 
\begin{align*}
v \wedge (\phi \circ A) \wedge x 
&\mapsto (\psi \oplus (\phi \circ A)) \wedge \sigma^n(\psi^{-1}(v) \wedge x) \\
&\sim (\psi \oplus \phi) \wedge (I_m\oplus A) . \sigma^n (\psi^{-1}(v) \wedge x)
\\
&= (\psi \oplus \phi) \wedge \sigma^n (\psi^{-1}(v) \wedge A.x)
\end{align*}
Where $I_m \in O(m)$ is the unit matrix and $I_m\oplus A$ the diagonal block
matrix. So the mapping respects the equivalence relation. 

Let $\psi' : \R^m \to V$, then $\psi' = \psi \circ B$ for some $B \in O(m)$, and
we have
\begin{align*}
&(\psi' \oplus \phi) \wedge \sigma^n ((\psi')^{-1}(v) \wedge x) \\
&\sim (\psi \oplus \phi) \wedge (B \oplus I)\sigma^n ((\psi')^{-1}(v) \wedge x) \\
&= (\psi \oplus \phi) \wedge \sigma^n (\psi^{-1}(v) \wedge x) \\
\end{align*}
So the definition is independent of the choice of $\psi$.

\item[(d)]
To define an orthogonal spectra, we define a based continues functor $X^\flat :
\O \to T_*$. 


Let $U, V$ and $W$ denote inner product spaces, 
Firstly,we define $X^\flat (W)$ as in (a). Secondly we need to define 
$\mu_{V,W}: \O(V, W) \wedge X^\flat(V) \to X^\flat(W)$, satisfying
associativity, ie. 
\[ \begin{tikzcd}[column sep = 7em]
\O(V, W) \wedge \O(U,V) \wedge X^\flat(U) 
\arrow[r, "O(V{,}W) \wedge \mu_{U{,}V}"] 
\arrow[d, "\circ \wedge X^\flat(U)"] 
& \O(V, W) \wedge X^\flat(V) 
\arrow[d, "\mu_{V{,}W}"] \\ 
\O(U,W) \wedge X^\flat(U) 
\arrow[r, "\mu_{U{,}W}"] 
& X^\flat(W)
\end{tikzcd} \]
commutes. Furthermore, we want the functionality to be given as in (b) and the
structure maps as in (c), which may be expressed by the following to relations
\begin{equation}
\begin{split}
X^\flat(V,W) &= \mu_{V, W} \circ (j_{V,W} \wedge \id_{X^\flat(W)}) : 
\L(V, W)_+ \wedge X^\flat (V) \to X^\flat(W),
\\
\sigma_{V,W} &= \mu_{W,V\oplus W} \circ (i_V \wedge \id_{X^\flat(W)}) :
S^V \wedge X^\flat (W) \to X^\flat(V \oplus W).
\end{split}
\end{equation}
Where 
\begin{align*}
j_{V,W} : \L(V,W)_+ \into \O(V,W); & \phi \mapsto (0,\phi), \\
i_V : S^V \into \O(W, V\oplus W); & v \mapsto ((v,0), (0,\id_W)).
\end{align*}


We consider $(w, \xi) \in \Xi(V,W)$, $[\phi \wedge x] \in X^\flat(V)$. Let 
\[ W' = W/\im(\xi), \: m = \dim(W') \q{\tand} \psi : \R^{m'} \to W'. \]

Then we define $\mu$ as follows 
\[ \mu((w,\xi) \wedge [\phi \wedge x]) :=
[ \psi \oplus (\xi \circ \phi) \wedge \sigma^m(\psi^{-1} \wedge x)] 
\]

This is clearly continues and extends continuously to $\O(V,W)$, since both
$\psi^{-1}$ and $\sigma^m$ extends continuously to the one point
compactification.



We check that this satisfy the associativity: 
Let $(w, \xi) \in \Xi(V,W)$, $(v, \zeta) \in \Xi(U,V)$, $[\phi \wedge x] \in
X^\flat(U)$. Let $W'$, $m$ and $\psi$ as above and 
\[ V' = V/\im(\zeta),\:\: n = \dim(V'),\:\: \psi' : \R^n \to V', \] 
\[ W'' = W/\im(\xi \circ \zeta),\:\: m' = \dim(W'') = m + n,\:\: \psi'' = \psi
\oplus \psi' : \R^{m'} \to W''. \] 

Then we have
\begin{align*}
P_1 = \mu(\circ \wedge \id)((w,\xi) & \wedge (v, \zeta) \wedge [\phi \wedge x]) \\
&= [(\psi'' \oplus (\xi\circ \zeta \circ \phi)) 
\wedge \sigma^{m'}(\psi''^{-1}(w + \zeta(v))\wedge x) ]
\end{align*}

\begin{align*}
P_2 = \mu (\id \wedge \mu)((w,\xi)& \wedge (v, \zeta) \wedge [\phi \wedge x]) \\
&= [ \psi \oplus (\xi \circ (\psi' \oplus (\zeta \circ \phi)))
\wedge 
\underbrace{\sigma^m(\psi^{-1}(w) \wedge \sigma^n(\psi'^{-1}(v) \wedge x)
)}_{= \sigma^{m+n}(\psi^{-1}(w) \wedge \psi'^{-1}(v) \wedge x)} ]
\end{align*}


This is not DONE

\item[(e)]
Let $f : X \to Y$ be a morphism of coordinatized orthogonal spectra, then we
define $f^\flat : X^\flat \to Y^\flat$, for $[\phi \wedge x] \in X^\flat(V)$, by
\[ f^\flat_V([\phi \wedge x]) := [\phi \wedge f_n(x)], \]
where $n = \dim(V)$. This is well-defined, since $f_n$ is $O(n)$-equivariant.

For this to be functional we need the following diagram
\[ \begin{tikzcd}
\O(V,W) \wedge X^\flat(V) \ar[r, "\mu"] \ar[d, "\id \wedge f^\flat_V"]
& X^\flat(W) \ar[d, "\id \wedge f^\flat_W"] \\
\O(V,W) \wedge Y^\flat(V) \ar[r, "\mu"]
& Y^\flat(W)
\end{tikzcd} \]
to commute. But this is clear from the definition.

\item[(f)]
For $n \ge 0$, we have 
\[ (UX^\flat)_n = X^\flat(\R^n) = (\L(\R^n,\R^n)_+  \wedge X_n). \]
We know from part (a), that this space is (based)homomorphic to $X_n$, by the
map $\eta$, which is clearly natural, since any morhphism is $O(n)$-equivariant.
% The maps $q$ and $\eta$ are also clearly natual
We also need to check that the structure maps match up. We have

\[ \til\sigma_n = \sigma_{\R,\R^n} : S^1 \wedge (UX^\flat)_n \to (U
X^\flat)_{n+1} \]
given by
% \begin{align*}
% v \wedge [\phi \wedge x] &\mapsto [(\psi \oplus \phi) \wedge
% \sigma^1(\psi^{-1}(\lambda) \wedge x)] \\
% & \quad = \sigma_n(\lambda \wedge x)
% \end{align*}

Conversly we define 
\[ \rho : Y \to (UY)^\flat \]
for an inner product space $W$, by

\begin{align*}
\rho_W : (UY)^\flat(W) &= (\L(\R^n,W)_+ \wedge Y(\R^n))/\sim \q{\to} Y(W) \\
\text{$[$} \phi \wedge x] &\mapsto (Y(\R^n,w) \circ (j_{\R^n,W} \wedge \id))(\phi \wedge
x) \\
&= (Y(\R^n,w) \circ (j_{\R^n,W} \wedge \id))(\phi \wedge x) 
\end{align*}

% 
% \begin{align*}
% \rho_W : (UY)^\flat(W) &= (\L(\R^n,W)_+ \wedge Y(\R^n))/\sim \q{\to} Y(W) \\
% \text{$[$} \phi \wedge x] &\mapsto (Y(\R^n,w) \circ (j_{\R^n,W} \wedge \id))(\phi \wedge
% x) \\
% &= (Y(\R^n,w) \circ (j_{\R^n,W} \wedge \id))(\phi \wedge x) 
% \end{align*}





\end{enumerate}
\end{exercise}

\end{document}
