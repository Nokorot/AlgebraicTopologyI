\sheet 6

\def\Eps{\mathcal{E}}

\begin{document}
\mmaketitle

\begin{exercise}[1]\ 

\begin{enumerate}
\item[(a)] 
Given $\alpha: A \to X$, $\beta : C \to Y$ and $(p : X\to Y) \in \Eps$, such
that the outer square in the following diagram commutes
\[ \begin{tikzcd}
A \ar[r,"\alpha"] \ar[d, "j \circ i"] 
& X \ar[d, "p"] \\
B \ar[r, "\beta"] \ar[ru, dotted, "\lambda"]
& Y
\end{tikzcd} \]
We want to construct the map $\lambda: C \to X$, such that the hole diagram
commutes.

Since $i \in \Eps^\orth$ we have the lift $\lambda' : B \to X$, such that 
\[ \begin{tikzcd}
A \ar[r,"\alpha"] \ar[d, "i"] 
& X \ar[d, "p"] \\
B \ar[r, "\beta \circ j"] \ar[ru, dotted, "\lambda'"]
& Y
\end{tikzcd} \]
commutes. Since also $j\in \Eps^\orth$ we have the lift $\lambda$, such that 
\[ \begin{tikzcd}
A \ar[r,"\lambda'"] \ar[d, "j"] 
& X \ar[d, "p"] \\
C \ar[r, "\beta"] \ar[ru, dotted, "\lambda"]
& Y
\end{tikzcd} \]
commutes. It is clear that this lambda makes the original diagram commute.

\item[(b)]
We consider the following diagram.
\[ \begin{tikzcd}
A \ar[r, "\alpha"] \ar[rr, bend left, "\beta \circ \alpha"] 
\ar[d, "i"] 
& C \ar[r, "\beta"] \ar[d, "j"] 
& X \ar[d, "p"]
\\ 
B \ar[r, "\gamma"] \ar[rr, bend right, "\delta \circ \gamma"] 
& D \ar[r, "\delta"] 
\ar[rru, dotted, "\lambda"]
& Y
\end{tikzcd} \]
where $p\in \Eps$ and $\alpha, \beta, \gamma$ and $\delta$ are such that
the solid diagram commutes. 
We want to define $\labda : D \to X$ such that the
full diagram commutes.

Since $i \in \Eps^\orth$, we have $\lambda' : B \to X$, with respect to the
outer square of the diagram. With that we consider the following diagram
\[ \begin{tikzcd}
A \ar[r,"\alpha"] \ar[d, "i"] 
& C \ar[d, "j"] \ar[rdd, bend left, "\beta"] 
& \\
B \ar[r, "\gamma"] \ar[rrd, bend right, "\lambda'"] 
& D \ar[rd, dotted, "\lambda"]
& \\ 
& & X
\end{tikzcd} \]
The solid part commutes by the previous diagram and we get a unique map 
$\lambda : D \to X$, by the universal property of the push out making the full
diagram commute.

It is clear that this is the map we want.
\item[(c)]
Let $p\in \Eps$, $\alpha: C \to X$ and $\beta: D \to Y$, such that the solid
part of the following diagram commutes.
\[ \begin{tikzcd}
C \ar[r,"\alpa"] \ar[d, "j"] 
& X \ar[d, "p"] \\
D \ar[r, "\beta"] \ar[ru, dotted, "\lambda"]
& Y
\end{tikzcd} \]
We want to define $\lambda$ such that the full diagram commutes.

We consider the following diagram
\[ \begin{tikzcd}
C \ar[r, "s"] \ar[rr, bend left, "\id"] \ar[d, "j"]
& A \ar[r, "r"] \ar[d, "i"] 
& C \ar[r, "\alpha"] \ar[d, "j"]
& X \ar[d, "p"] 
\\
D \ar[r, "t"] \ar[rr, bend right, "\id"] 
& B \ar[r, "u"] \ar[rru, dotted, "\lambda'"] 
& D \ar[r, "\beta"]
& Y
\end{tikzcd} \]
The solid part clearly commutes and thus, since $i\in \Eps^\orth$, we get the
map $\lambda'$, such that the full diagram commutes.

It is clear that $\lambda = \lambda' \circ t$ gives us the map we want, by
considering the outermost square in the previous diagram.

\item[(d)]
Let $\alpha : A_0 \to X$, $\beta : A:=\colim_n A_n \to Y$ and $(p : X \to Y)
\in \Eps$, such that the solid part of the following diagram commutes
\[ \begin{tikzcd}
A_0 \ar[r,"\alpha"] \ar[d, "\hat i_0"] 
& X \ar[d, "p"] \\
A \ar[r, "\beta"] \ar[ru, dotted, "\lambda"]
& Y
\end{tikzcd} \]
where $\hat i_0 : A_0 \to A$ is the canonical morphism.
We want to define $\lambda$ so that the full diagram commutes.

By the left lifting property of the sequential maps $i_j : A_j \to A_{j+1}$,
(for $j= 0,1,...$). Let $\lambda_j = \alpha$, then we will inductively define
maps $\lambda_j : A_j \to X$ such that the following diagram commutes.
\[ \begin{tikzcd}
A_j \ar[r,"\lambda_j"] \ar[d, "i_j"] 
& X \ar[d, "p"] \\
A_{j+1} \ar[r, "\beta \circ \hat i_{j+1}"] \ar[ru, dotted, "\lambda_{j+1}"]
& Y
\end{tikzcd} \]
The commutativity of the solid square follows by the inductive hypothesize,
(and the definition of $\alpha$ and $\beta$ in the case of $j=0$) and the existence
of the map $\labbda_{j+1}$ follows since $i_j \in \Eps^\orth$. 

It is imminently clear from the inductive construction that $\lambda_{j+1}
\circ i_j = \lambda_j$, so by the universal property of the commit, 
there is a unique map $\lambda : A \to X$ such that 
$\lambda_j = \lambda \circ \hat i_j$.

It follows by the defining property of $\lambda_1$ (and $\lambda$) that 
\[ \begin{tikzcd}[row sep = 4em]
A_0 \ar[rr,"\alpha"] \ar[d, "i_0"] &
& X \ar[d, "p"] \\
A_1 \ar[r, "\hat i_1"] \ar[rru, "\lambda_1"]
& A \ar[r, "\beta"] \ar[ru, "\lambda"]
& Y
\end{tikzcd} \]
commutes. So $\lambda$ is the map that we want, since $\hat i_0 = \hat i_1
\circ i_0$.


\end{enumerate}



\end{exercise}

\end{document}
