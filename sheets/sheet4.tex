\sheet 4

\def \L{\mathbb{L}}
\def \S{\mathbb{S}}
\def \O{\mathcal{O}}

\def \Sp{\textit{Sp}}
\def \SMod{\S\text{-}\hspace{-.15em}\Mod}

\def \smsh {\wedge}

\renewcommand{\Xi}{{\mbox{\larger[2]$\pmb\xi$}}}

\begin{document}
\mmaketitle

\begin{exercise}[1]\ 
Let $\S^{\ge i}$ be the spectrum 
\[ (\S^{\ge i})_n = \pwf{S^n &\tif n \ge i \\ * &\tif n < i} \]
with the obvious structure maps. 
Then the natural map 
\[ \pi_0\left( \prod_{i\in\N} \S^{\ge i} \right) \to 
    \prod_{i\in\N} \pi_0(\S^{\ge i}) \]
is not surjective.

Any element on the left hand side is given by a representative 
$f : S^n \to \prod_{i\in\N} \S^{\ge i}$ for some $n in \N$.
The element $[f]$ is mapped to $\prod_{i\in\N} [\pr_i \circ f]$. So every
element in the image of the map may be written as a product $[\prod_{i\in\N}
f_i : S^n \to S^i]$, for some $n\in \N$ independent of $i$.

Hence $\iota = \prod_{i\in\N} [\id_i : S^i \to S^i] \in \prod_{i\in\N}
\pi_0(\S^{\ge i})$ not in the image, since for any $n\in \N$, $[\id_i]$ for
all $i > n$, may not be represented by a map $f : S^n \to S^i$, as $\pi_n(S^i)
= 0$ for all $i > n$.
\end{exercise}


\begin{exercise}[4]\ 
Let $F: \Sp \to \SMod$ be the functor defined by sending 
$X \in \Sp$, to the $\S$-module, with underlying spectrum $X$ and action map 
$\alpha_{VW} = \sigma_{VW}$ and sending every morphism of spectra to is self.
Then $F$ defines an inverse of the forgetfull functor. 

Firstly, we need to check that the structure maps satisfy associativity and
unitality. Associativity follows immediately from the associativity from the
associativity in the definition of a spectra as a functor $X: \O \to T_*$. 
The generalized unit $\iota_V : \S^V \to \S^{V \oplus 0} \cong S^V$ is the map
identity map. So $\sigma_{VW} \circ (\iota_V \smsh \id) = \sigma_{VW}$.

A morphism $f: M \to N$ in $\SMod$ is a morphism of the underlying spectra, satisfying that the following
diagram commutes
\[ \begin{tikzcd}[column sep = 7em]
S^V \smsh M(W) \ar[r, "\sigma_{VW}"] \ar[d, "S^V \smsh"] 
& M(V\oplus W) \ar[d, "f(V\oplus W)"] \\
S^V \smsh M(W) \ar[r, "\sigma_{VW}"] 
& N(V\oplus W)
\end{tikzcd} \] 
However this diagram is exactly the same as the commutative diagram in the
definition of a morphism of spectra. So a morphism of spectra is already a
morphism of $\S$-modules. It follows immediately that  $F$ is functorial. 

Finally, it is immediately clear that $F$ defines a right inverse of the
forgetfull functor. To check that it is also a left inverse consider $M \in
\SMod$. Then by unitality $\alpha_{VW} \circ (\iota_V \smsh M(W)) =
\sigma_{VW}$, but $\iota_V = \id : S^V \to \S^V$, so $\alpha_{VW} =
\sigma_{VW}$ and thus $M = F(U(M))$.
\end{exercise}

\end{document}
