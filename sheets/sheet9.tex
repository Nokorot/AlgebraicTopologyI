\sheet 9

\def\op{\text{op}}

\DeclareMathOperator{\coker}{coker}

\begin{document}
\mmaketitle

\begin{exercise}[1]\ 

\begin{enumerate}
\item[(T0)] Follows by the five lemma.
\item[(T1)]
Suppose $f: U \to V$. Let 
\[ W = \coker(f) \oplus \ker(f), \quad g: V \onto \coker(f) \into W \qq\tand h:
W \onto \ker(f) \into U, \]
then $(f,g,h)$ is a distinguished
triangle.
\item[(T2)]
Clearly $(0,\id,0)$ is distinguished.
\item[(T3)]
Suppose $(f,g,h)$ is distinguished.
It is imitate that $(g,h,f)$ is also distinguished. And since $\ker(f) = \ker(-f)$
and $\im(f) = \im(-f)$, it also follows that $(g,h,-f)$ is distinguished.
\item[(T4)]
This is where the arguments break in the case of modules over a general ring.
Suppose $(f,g,h)$ and $(f',g',h')$ is distinguished, and $a,b$ are maps such
that the solid part of the following diagram commutes
\[ \begin{tikzcd}
U \ar[r,"f"] \ar[d, "a"] 
& V \ar[d, "b"] \ar[r, "g"]
& W \ar[r, "h"] \ar[d, "\exists c", dashed] 
& U \ar[d, "a"] 
\\
U' \ar[r, "f'"] 
& V' \ar[r, "g'"] 
& W' \ar[r, "h'"]
& U'
\end{tikzcd} \]

Since we are in the category of vector spaces, all exact sequences splits, 
so we may wlog assume that 
\[ W=\im(g)\oplus\coker(g) \qq\tand W=\im(g')\oplus \coker(g'). \] 
Furthermore, since 
\[ g : V \to \im(g) \qq\tand h'|_{\coker(g')} : \coker(g') \to \im(h') \] 
are surjective, we may choose right inverses 
\[ s: \im(g) \to V \qq\tand s' : \im(h') \to \coker(g'). \] 

Note that $\im(ah) \in \ker(f') = \im(h')$, since $f'ah = bfh = 0$. So we may
define
\[ c: W = \im(g)\oplus \coker(g) \to W'; \quad (x + y) \mapsto g'bs(x) + s'ah(y).
\]

It remains to show that the full diagram commutes. The fact that $h'c = ah$ is
immediate from the definition of $c$. For the other side, consider $x \in V$,
by definition $cg (x) = g'bsg(x)$. Since $g(x) = gsg(x)$, we have $(sg(x) - x) \in \ker(g) =
\im(f)$. So, since $bf=f'a$, 
\[ b(sg(x) - x) \in \im(f') = \ker(g'), \qq{\text{ie.}} 
 g'bsg(x) = g'b(x), \]
and thus $g'b = cg$.

\item[(T5)]
Consider the following commutative diagram 
\[ \begin{tikzcd}%[column sep = 7em]
U \ar[r, "f"] \ar[d, equal] 
& V \ar[r, "g"] \ar[d, "f'"] 
& W \ar[r, "h"] \ar[d, "x"]  %%%
& U \ar[d, equal] \ar[r, "f"] 
& V \ar[d, "f'"] 
\\
U \ar[r, "f'f"]
& X \ar[r, "g'"] \ar[d, "g''"] 
& Y \ar[r, "h''"] \ar[d, "y", dashed]  %%%
& U \ar[d, "f"] \ar[r, "f'f"] 
& X 
\\
& Z \ar[r, equal] \ar[d, "h'"] 
& Z \ar[r, "h'"] \ar[d, "g\circ h'"] 
& V
\\
& V \ar[r, "g"] \ar[d, "f'"] 
& W \ar[d, "x"] 
\\
& X \ar[r, "g'"]
& Y 
\end{tikzcd} \]
Where we assume $(f,g,h), (f'f,g',h'')$ and $(f',g'',h')$ are distinguished.
Also $x$ is defined according to $(T4)$. 

We want to define $y$ such that the diagram commutes and $(x,y,gh')$ is
distinguished. 


Let $Z' = \coker(x)\oplus\ker(x)$, 


Since $(f'f, g', h'')$ is distinguished, we may assume $Y = \im(g') \oplus
\coker(g')$, and since $(f',g'',h')$ is distinguished, we may assume $V =
\im(h') \oplus \coker(h')$.
As before, 
\[ g' : X \to \im(g') \qq\tand h' : Z \to \im(h') \]
are surjective, so we may choose right inverses 
\[ s : \im(g') \to X \qq\tand s' : \im(h') \to Z \]


\end{enumerate}
\end{exercise}

\end{document}
