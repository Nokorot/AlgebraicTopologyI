\documentclass[a4paper,11pt,english]{article}
\usepackage{.styles/basic}
\usepackage{.styles/envs}
\frenchspacing

\usepackage{circuitikz}

\usepackage{caption}
\usepackage{subcaption}

%%%%%%% Title %%%%%%%%%%%%%%%%%%%%%%%%%%%
\title{\textbf{Algebraic Topology} - Exercise Sheet 7}
\author{Tor Gjone (2503108) \& Michele Lorenzi (3461634)}

%%%%%%% Definitions %%%%%%%%%%%%%%%%%%%%%

% Michele's stuff
% (I hope I don't break something...)
% (Most of the command are thought for livetexing but some became habits)
% ( :D don't worry. I'm a huge fan of shortcuts myself )

\makeatletter 
% \ni was used in the definitions of other commands. This is a solution
\newcommand{\@ni}{{n-1}}

\newcommand{\pin}{\pi_n(X,x_0)}
\newcommand{\pini}{\pi_\@ni(X,x_0)}
\newcommand{\pinr}{\pi_n(X,A,x_0)}
\newcommand{\pinir}{\pi_\@ni(X,A,x_0)}
\newcommand{\hn}{H_n(X;\Z)}
\newcommand{\hni}{H_\@ni(X;\Z)}
\newcommand{\hnr}{H_n(X,A;\Z)}
\newcommand{\hnir}{H_\@ni(X,A;\Z)}
\newcommand{\dn}{D^n}
\newcommand{\dni}{D^\@ni}
\newcommand{\sn}{S^n}
\newcommand{\sni}{S^{n-1}}
\newcommand{\sph}{(D^n,\de D^n)}
\newcommand{\sphi}{(D^\@ni,\de D^\@ni)}
\newcommand{\sm}{\smallsetminus}
\newcommand{\til}{\tilde}
\newcommand{\fa}{\;\;\forall}
\newcommand{\ul}[1]{\,\underline{#1}\,}
\newcommand{\ol}{\overline}
\newcommand{\nf}{\normalfont}
\newcommand{\xs}{{x_1,\dots,x_n}}
\newcommand{\xso}{{x_0,\dots,x_n}}
\newcommand{\nn}[2]{{{#1}_1,\dots,{#1}_{#2}}}
\newcommand{\nno}[2]{{{#1}_0,\dots,{#1}_{#2}}}
\newcommand{\nns}[3]{{{#1}_1\,{#2}\,\dots\,{#2}\,{#1}_{#3}}}
\newcommand{\nnso}[3]{{{#1}_0\,{#2}\,\dots\,{#2}\,{#1}_{#3}}}
\newcommand{\cb}[1]{\{#1\}}
\newcommand{\de}{\partial}
%\newcommand{\xto}[1]{\xrightarrow{#1}}
\newcommand{\hto}{\hookrightarrow}
\newcommand{\nii}{\@ni}
\newcommand{\cont}{\reflectbox{$\in$}}
\newcommand{\mi}{{m-1}}
\newcommand{\ki}{{k-1}}
\DeclareMathOperator{\colim}{colim}
\DeclareMathOperator{\tel}{tel}
\DeclareMathOperator{\proj}{proj}


% Tor's stuff
\newcommand{\oc}[1]{\overset{\circ}{#1}}
\newcommand{\into}{\hookrightarrow}
\newcommand{\nb}{\nabla}
\renewcommand{\hat}{\widehat}

\usepackage{scalerel,stackengine}
\stackMath
\renewcommand\widehat[1]{%
\savestack{\tmpbox}{\stretchto{%
  \scaleto{%
    \scalerel*[\widthof{\ensuremath{#1}}]{\kern.1pt\mathchar"0362\kern.1pt}%
    {\rule{0ex}{\textheight}}%WIDTH-LIMITED CIRCUMFLEX
  }{\textheight}% 
}{2.4ex}}%
\stackon[-6.9pt]{#1}{\tmpbox}%
}


\newcommand{\diagSquare}[8]{
\begin{tikzpicture}[node distance=2cm, auto]
\node (a)              { $ #5 $ };
\node (b) [right of=a] { $ #6 $ };
\node (c) [below of=a] { $ #7 $ };
\node (d) [right of=c] { $ #8 $ };
\draw[-to] (a) to node { $ #1 $ } (b);
\draw[-to] (a) to node { $ #2 $ } (c);
\draw[-to] (b) to node { $ #3 $ } (d);
\draw[-to] (c) to node { $ #4 $ } (d);
\end{tikzpicture}
}

\newcommand{\congto}{\xrightarrow{\cong}}
\newcommand{\incto}[1][]{ \xhookrightarrow{#1} }
\newcommand{\xto}[1]{\xrightarrow{#1}}

% Text key words
\newcommand{\tif}{\text{if }}
\newcommand{\tand}{\text{and }}
\newcommand{\tsince}{\text{since }}


% Projective spaces
\def\RP{\mathbb{RP}}
\def\CP{\mathbb{CP}}
\def\HP{\mathbb{HP}}

% Common Categories
\DeclareMathOperator{\Top}   {\bf Top}
\DeclareMathOperator{\Ab}    {\bf Ab}
\DeclareMathOperator{\Cat}   {\bf Cat}
\DeclareMathOperator{\CAT}   {\bf CAT}
\DeclareMathOperator{\Mod}   {\bf Mod}
\DeclareMathOperator{\Ring}  {\bf Ring}
\DeclareMathOperator{\Group} {\bf Group}
\DeclareMathOperator{\sSet}  {{\bf sSet}}

% Homological Algebra
\DeclareMathOperator{\Hom}   {Hom}
\DeclareMathOperator{\Tor}   {Tor}
\DeclareMathOperator{\Ext}   {Ext}

% Common Lie Groups 
\DeclareMathOperator{\GL}    {GL}
\DeclareMathOperator{\SU}    {SU}
\DeclareMathOperator{\U}     {U}
\DeclareMathOperator{\Sp}    {Sp}

% Maths Operators
\DeclareMathOperator{\pr}    {pr}
\DeclareMathOperator{\id}    {id}
\DeclareMathOperator{\Ker}   {Ker}
\DeclareMathOperator{\im}    {Im}

% Standard sets
\def \N {\mathbb{N}}
\def \Z {\mathbb{Z}}
\def \Q {\mathbb{Q}}
\def \R {\mathbb{R}}
\def \C {\mathbb{C}}
\def \H {\mathbb{H}}

\def \E {\mathbb{E}}
\def \Z {\mathbb{Z}}
\def \I {\mathbb{I}}
\def \J {\mathbb{J}}

% Vector calculus
\newcommand{\dif}[3][]{
	\ensuremath{\frac{d^{#1} {#2}}{d {#3}^{#1}}}}
\newcommand{\pdif}[3][]{
	\ensuremath{\frac{\partial^{#1} {#2}}{\partial {#3}^{#1}}}}

% Vectors and matrices
\newcommand{\mat}[1]{\begin{matrix} #1 \end{matrix}}
\newcommand{\pmat}[1]{\begin{pmatrix} #1 \end{pmatrix}}
\newcommand{\bmat}[1]{\begin{bmatrix} #1 \end{bmatrix}}

% Add space around the argument
\newcommand{\qq}[1]{\quad#1\quad}
\newcommand{\q}[1]{\:\:#1\:\:}

% Implications
\newcommand{\la} {\ensuremath{\Longleftarrow}}
\newcommand{\ra} {\ensuremath{\Longrightarrow}}
\newcommand{\lra}{\ensuremath{\Longleftrightarrow}}

\newcommand{\pwf}[1]{\begin{cases} #1 \end{cases}}

% Shorthand
\newcommand{\vphi}{\varphi}
\newcommand{\veps}{\varepsilon}

\newcommand{\<}[1]{\langle #1 \rangle}

% Notation
\newcommand{\wddef}[1]{\underline{#1}}
\newcommand{\pref}[1]{(\ref{#1})}

% Maths Operators
\theoremstyle{plain}
\theoremstyle{definition}
\newtheorem{thrm}{Theorem}[section]
\newtheorem{prop}[thrm]{Proposition}
\newtheorem{corol}[thrm]{Corollary}
\newtheorem{lemma}[thrm]{Lemma}

\newtheorem{defn}[thrm]{Definition}
\newtheorem{exmp}[thrm]{Example}
\newtheorem{clame}[thrm]{Clame}

\theoremstyle{remark}
% \newtheorem{remark}[thrm]{\normalfont\large\textit Remark}
\newtheorem{remark}[thrm]{Remark}
\newtheorem{note}[thrm]{Note}


\newSimpleHeaderEnvironment{exercise}{Exercise }

\setlength{\parindent}{0cm}
\setlength{\parskip}{3pt}

\newcommand{\orth}{\bot}

\renewcommand{\S}[1]{\mathcal{S}(#1)}
%%%%%%%% Content %%%%%%%%%%%%%%%%%%%%%%%%
\begin{document}

\mmaketitle

\begin{exercise}[1]

Suppose $f_n \to f$ in the compact-open topology. That is for all $K \se
X$ compact and for all $U \se Z$ open, such that $f(K)\se U$, there exists $N \in \N$, such that for all
$n \ge N$, $f_n(K) \se U$.

Let $x \in X$, then $\{x\} \se X$ is compact, so for all neighbourhoods $U$ of $f(x)$, there
exists $N\in\N$ such that, for all $n\ge N$, $f_n(x) \in U$. So $f_n \to
f$ pointwise (since $Z$ is Hausdorff, so that limits are unique).

Now let $X = Z = [0,1]$ and 
\[ f_n(x) = \pwf{
2nx &\tif x \le (2n)\i \\
2 - 2nx &\tif (2n)\i \le x \le n\i \\
0 &\tif n\i \le x
} \]

\begin{figure}[h]
\centering
\begin{tikzpicture}
\draw[thick,->] (-.2,0) -- (2.2,0); % node[anchor=west] {x axis};
\draw[thick,->] (0,-.2) -- (0,1.2); % node[anchor=east] {y axis};

\draw[cyan] (0,0) -- (.2,1) -- (.4,0) -- (2,0);
\draw (.4,1pt) -- (.4,-1pt) node[anchor=north] {$1/n$};
\draw (2,1pt) -- (2,-1pt) node[anchor=north] {$1$};
\draw (1pt,1) -- (-1pt,1) node[anchor=east] {$1$};
\end{tikzpicture}

\caption{Graph of $f_n$.}
\label{fig:1}
\end{figure}

Then $f_n \to f = 0$ pointwise. However $f_n([0,1]) =
[0,1] \not\se [0,1/2)$, whereas $f([0,1]) = \{0\} \se [0,1/2)$. So $f_n$ does
not converge to $f$ in the compact-open topology.
\end{exercise}



\begin{exercise}[2]

It is easy to show that there are isomorphisms $\pi_{n}(F,e) \cong \pi_n(\Omega B,c_b)$. Indeed, since $p : E \to B$ is a Serre fibration, we have the long exact
sequence
\[ \cdots \to \pi_n(E,e) \to \pi_n(B,b) \to \pi_{n-1}(F,e) \to \pi_{n-1}(E,e) \to
\cdots \]
where $p\i(b)=e \in F$. Since $E$ is contractible we have $\pi_n(E,e) = 0$,
for $n \ge 0$. So the middle map is an isomorphism and thus 
\[ \pi_{n}(F,e) \cong \pi_{n+1}(B,b) \cong \pi_n(\Omega B,c_b) \]
for $n \ge 0$.

To show that $F$ and $\Omega B$ are homotopy equivalent we need to find a map $F\to\Omega B$ inducing these isomorphisms. A reasonable choice seems to be the map $f$ constructed using a deformation retraction $H$ of $E$ to the point $e$: for any point $x$ in the fiber,  $p\circ H(x,-)$ defines a loop at $b$. We need to show that this induces isomorphisms $\pi_n(F,x)\to\pi_n(\Omega B,f(x))$ for any $x\in F$ and any $n\geq0$, i.e. we are left to show injectivity and surjectivity of these maps. This can be done using the exponential law and the homotopy lifting property, but there is an easier proof: we can define a map $E\to B^{[0,1]}$ by considering $p\circ H(x,-)$ for any $x\in E$, then the long exact sequence for $F\to E\to B$ maps to the long exact sequence for $\Omega B\to B^{[0,1]}\to B$, and since $E$ and $B^{[0,1]}$ are contractible, by the five-lemma we obtain the claim.
\end{exercise}


\begin{exercise}[3](The solution to exercise 6.3 might be useful for reference.)

Given $\sigma \in \Sigma_3$, let $\Psi_\sigma$ be the composite map defined in
the exercise. 

Consider a pair $(\phi,\psi) \in \pi_1(X) \times \pi_1(X)$, with $\phi=[f]$ and $\psi=[g]$. The first isomorphism maps, this pair to the class $[(f,g)]$, which is mapped to $[h] \in \pi_0(E)$, where $h: \nb^2 \to E$ is such that $h |_{I_{0,1}} = f$ and $h |_{I_{1,2}} = g$. Furthermore $h|_{I_{0,2}}$ is homotopy equivalent to $fg$ (see figure). The map $\pi_0(\sigma^*)$ permutes these edges according to $\sigma$. The rest of the maps then restricts to the edges $I_{0,1}$ and $I_{1,2}$. 

\begin{figure}[h]
\centering
\begin{tikzpicture}
\node[above] (c) at (1,1.6) {$2$};
\node[left]  (a) at (0,0) {$0$};
\node[right] (b) at (2,0) {$1$};
\draw[line width=1.5px] (0,0) 
    -- (2,0)    node[sloped, scale=.9, pos=0.5] {$<$} 
                node[scale=1, pos=0.5, below] {$f$} 
    -- (1,1.6)  node[sloped, scale=.9, pos=0.5] {$>$} 
                node[scale=1, pos=0.5, above right] {$g$}
    -- (0,0)    node[sloped, scale=.9, pos=0.5] {$<$} 
                node[scale=1, pos=0.5, above left] {$fg$};
\end{tikzpicture}
\caption{}
\label{fig:my_label}
\end{figure}

Referring to the figure, it is quite clear that the action of swapping the vertices $0$ and $1$, reverses the orientation of $f$ and exchanges $g$ and $fg$. Then we see that:

\begin{eqnarray*}
\Psi_{(0\; 1)};&& (\phi, \psi) \mapsto (\phi^{-1}, \phi \psi) \\
\Psi_{(0\; 2)};&& (\phi, \psi) \mapsto (\psi^{-1}, \phi^{-1}) \\
\Psi_{(1\; 2)};&& (\phi, \psi) \mapsto (\phi\psi, \psi^{-1})
\end{eqnarray*}

Furthermore it is easy to observe that if $\sigma = \sigma_1 \sigma_2$, then $\Psi_\sigma = \Psi_{\sigma_1} \circ \Psi_{\sigma_2}$. So we are done since the tree other permutations are products of two of the cases listed above.
\end{exercise}

\end{document}
