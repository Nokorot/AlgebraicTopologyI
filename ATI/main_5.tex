\documentclass[a4paper,11pt,english]{article}
\usepackage{.styles/basic}
\usepackage{.styles/envs}

\usepackage{circuitikz}

%%%%%%% Title %%%%%%%%%%%%%%%%%%%%%%%%%%%
\title{\textbf{Algebraic Topology} - Exercise Sheet 5}
\author{Tor Gjone (2503108) \& Michele Lorenzi (3461634)}

%%%%%%% Definitions %%%%%%%%%%%%%%%%%%%%%

% Michele's stuff
% (I hope I don't break something...)
% (Most of the command are thought for livetexing but some became habits)
% ( :D don't worry. I'm a huge fan of shortcuts myself )

\makeatletter 
% \ni was used in the definitions of other commands. This is a solution
\newcommand{\@ni}{{n-1}}

\newcommand{\pin}{\pi_n(X,x_0)}
\newcommand{\pini}{\pi_\@ni(X,x_0)}
\newcommand{\pinr}{\pi_n(X,A,x_0)}
\newcommand{\pinir}{\pi_\@ni(X,A,x_0)}
\newcommand{\hn}{H_n(X;\Z)}
\newcommand{\hni}{H_\@ni(X;\Z)}
\newcommand{\hnr}{H_n(X,A;\Z)}
\newcommand{\hnir}{H_\@ni(X,A;\Z)}
\newcommand{\dn}{D^n}
\newcommand{\dni}{D^\@ni}
\newcommand{\sn}{S^n}
\newcommand{\sni}{S^{n-1}}
\newcommand{\sph}{(D^n,\de D^n)}
\newcommand{\sphi}{(D^\@ni,\de D^\@ni)}
\newcommand{\sm}{\smallsetminus}
\newcommand{\til}{\tilde}
\newcommand{\fa}{\;\;\forall}
\newcommand{\ul}[1]{\,\underline{#1}\,}
\newcommand{\ol}{\overline}
\newcommand{\nf}{\normalfont}
\newcommand{\xs}{{x_1,\dots,x_n}}
\newcommand{\xso}{{x_0,\dots,x_n}}
\newcommand{\nn}[2]{{{#1}_1,\dots,{#1}_{#2}}}
\newcommand{\nno}[2]{{{#1}_0,\dots,{#1}_{#2}}}
\newcommand{\nns}[3]{{{#1}_1\,{#2}\,\dots\,{#2}\,{#1}_{#3}}}
\newcommand{\nnso}[3]{{{#1}_0\,{#2}\,\dots\,{#2}\,{#1}_{#3}}}
\newcommand{\cb}[1]{\{#1\}}
\newcommand{\de}{\partial}
%\newcommand{\xto}[1]{\xrightarrow{#1}}
\newcommand{\hto}{\hookrightarrow}
\newcommand{\nii}{\@ni}
\newcommand{\cont}{\reflectbox{$\in$}}
\newcommand{\mi}{{m-1}}
\newcommand{\ki}{{k-1}}
\DeclareMathOperator{\colim}{colim}
\DeclareMathOperator{\tel}{tel}
\DeclareMathOperator{\proj}{proj}


% Tor's stuff
\newcommand{\oc}[1]{\overset{\circ}{#1}}
\newcommand{\into}{\hookrightarrow}
\newcommand{\nb}{\nabla}
\renewcommand{\hat}{\widehat}

\usepackage{scalerel,stackengine}
\stackMath
\renewcommand\widehat[1]{%
\savestack{\tmpbox}{\stretchto{%
  \scaleto{%
    \scalerel*[\widthof{\ensuremath{#1}}]{\kern.1pt\mathchar"0362\kern.1pt}%
    {\rule{0ex}{\textheight}}%WIDTH-LIMITED CIRCUMFLEX
  }{\textheight}% 
}{2.4ex}}%
\stackon[-6.9pt]{#1}{\tmpbox}%
}


\newcommand{\diagSquare}[8]{
\begin{tikzpicture}[node distance=2cm, auto]
\node (a)              { $ #5 $ };
\node (b) [right of=a] { $ #6 $ };
\node (c) [below of=a] { $ #7 $ };
\node (d) [right of=c] { $ #8 $ };
\draw[-to] (a) to node { $ #1 $ } (b);
\draw[-to] (a) to node { $ #2 $ } (c);
\draw[-to] (b) to node { $ #3 $ } (d);
\draw[-to] (c) to node { $ #4 $ } (d);
\end{tikzpicture}
}

\newcommand{\congto}{\xrightarrow{\cong}}
\newcommand{\incto}[1][]{ \xhookrightarrow{#1} }
\newcommand{\xto}[1]{\xrightarrow{#1}}

% Text key words
\newcommand{\tif}{\text{if }}
\newcommand{\tand}{\text{and }}
\newcommand{\tsince}{\text{since }}


% Projective spaces
\def\RP{\mathbb{RP}}
\def\CP{\mathbb{CP}}
\def\HP{\mathbb{HP}}

% Common Categories
\DeclareMathOperator{\Top}   {\bf Top}
\DeclareMathOperator{\Ab}    {\bf Ab}
\DeclareMathOperator{\Cat}   {\bf Cat}
\DeclareMathOperator{\CAT}   {\bf CAT}
\DeclareMathOperator{\Mod}   {\bf Mod}
\DeclareMathOperator{\Ring}  {\bf Ring}
\DeclareMathOperator{\Group} {\bf Group}
\DeclareMathOperator{\sSet}  {{\bf sSet}}

% Homological Algebra
\DeclareMathOperator{\Hom}   {Hom}
\DeclareMathOperator{\Tor}   {Tor}
\DeclareMathOperator{\Ext}   {Ext}

% Common Lie Groups 
\DeclareMathOperator{\GL}    {GL}
\DeclareMathOperator{\SU}    {SU}
\DeclareMathOperator{\U}     {U}
\DeclareMathOperator{\Sp}    {Sp}

% Maths Operators
\DeclareMathOperator{\pr}    {pr}
\DeclareMathOperator{\id}    {id}
\DeclareMathOperator{\Ker}   {Ker}
\DeclareMathOperator{\im}    {Im}

% Standard sets
\def \N {\mathbb{N}}
\def \Z {\mathbb{Z}}
\def \Q {\mathbb{Q}}
\def \R {\mathbb{R}}
\def \C {\mathbb{C}}
\def \H {\mathbb{H}}

\def \E {\mathbb{E}}
\def \Z {\mathbb{Z}}
\def \I {\mathbb{I}}
\def \J {\mathbb{J}}

% Vector calculus
\newcommand{\dif}[3][]{
	\ensuremath{\frac{d^{#1} {#2}}{d {#3}^{#1}}}}
\newcommand{\pdif}[3][]{
	\ensuremath{\frac{\partial^{#1} {#2}}{\partial {#3}^{#1}}}}

% Vectors and matrices
\newcommand{\mat}[1]{\begin{matrix} #1 \end{matrix}}
\newcommand{\pmat}[1]{\begin{pmatrix} #1 \end{pmatrix}}
\newcommand{\bmat}[1]{\begin{bmatrix} #1 \end{bmatrix}}

% Add space around the argument
\newcommand{\qq}[1]{\quad#1\quad}
\newcommand{\q}[1]{\:\:#1\:\:}

% Implications
\newcommand{\la} {\ensuremath{\Longleftarrow}}
\newcommand{\ra} {\ensuremath{\Longrightarrow}}
\newcommand{\lra}{\ensuremath{\Longleftrightarrow}}

\newcommand{\pwf}[1]{\begin{cases} #1 \end{cases}}

% Shorthand
\newcommand{\vphi}{\varphi}
\newcommand{\veps}{\varepsilon}

\newcommand{\<}[1]{\langle #1 \rangle}

% Notation
\newcommand{\wddef}[1]{\underline{#1}}
\newcommand{\pref}[1]{(\ref{#1})}

% Maths Operators
\theoremstyle{plain}
\theoremstyle{definition}
\newtheorem{thrm}{Theorem}[section]
\newtheorem{prop}[thrm]{Proposition}
\newtheorem{corol}[thrm]{Corollary}
\newtheorem{lemma}[thrm]{Lemma}

\newtheorem{defn}[thrm]{Definition}
\newtheorem{exmp}[thrm]{Example}
\newtheorem{clame}[thrm]{Clame}

\theoremstyle{remark}
% \newtheorem{remark}[thrm]{\normalfont\large\textit Remark}
\newtheorem{remark}[thrm]{Remark}
\newtheorem{note}[thrm]{Note}


\newSimpleHeaderEnvironment{exercise}{Exercise }

\setlength{\parindent}{0cm}
\setlength{\parskip}{3pt}

\newcommand{\orth}{\bot}

\renewcommand{\S}[1]{\mathcal{S}(#1)}
%%%%%%%% Content %%%%%%%%%%%%%%%%%%%%%%%%
\begin{document}

\mmaketitle

\begin{exercise}[1] (Incomplete for lack of time)

\paragraph{a)}
Consider a fixed $n$-frame $x = (x_1, ..., x_n) \in V_n(\R^k)$ and let 
$S = \span(x_1,...,x_n) \in \Gr_n(\R^k)$. 

The pre-image $q\i(S) \in V_n(\R^k)$, is given by all $n$-frames in $S$, that
spans $S$. So clearly $q\i(S) \cong V_n(\R^n) \cong O(\R^n)$, so the fibers of the
fibre bundle are $\cong O(\R^n)$.

Let $U_x = \{ T \in \Gr_n(\R^k) \q: T \cap S^{\orth} = 0 \}$. 

Claim: $U_x \cong (S^\orth)^n \cong (\R^{k-n})^n$.
\begin{proof}[Proof of claim]
Let $\psi: (S^\orth)^n \to U_x$, defined by
\[ \hat x = (\hat x_1, ..., \hat x_n) \mapsto \span(x+\hat x) = \span(x_1 + \hat x_1, ..., x_n + \hat x_n
), \]
where $\hat x_i \in S^{\orth}$. Since $x$ and $\hat x$ are contained in
orthogonal complements and $x$ consists of orthonormal vectors, it is clear that
$x + \hat x$ also consists of linearly independent vectors. So $\span(x+\hat x)$
defines an $n$-dimensional subspace of $R^k$, i.e. an element in
$\Gr_n(\R^k)$. 

It is also quite clear that the map is continuous. By applying the Gram-Smith's
algorithm (in order for this to produce a continuous mapping, we must apply the
algorithm in a fixed order, say according to the indices) on $x + \hat x$, we
get an $n$-tuple of orthonormal vectors, i.e. an element of $V_n(\R^k)$, that
varies continuously with $\hat x$.
This defines a continuous map $\xi: (S^{\orth})^n \to V_n(\R^k)$, such that $\psi = q \circ \xi$. So since $q$ is continuous by the definition of the topology on $\Gr_n(\R^k)$, $\psi$ is continuous.

To show that $\psi$ is bijective, we want to construct an inverse $\phi: U_x \to (S^{\orth})^n$, which is continuous, since 
\todo{open mapping theorem?}

Let $T \in U_x$. Since $T \cap S^\orth = 0$, the (orthogonal) projection of $T$ onto
$S$ defines is a homeomorphism, $p: T \to S$. We define \[ \phi(T) = p\i(x) - x. \]
Clearly $p\i(x) - x \in S^\orth$ and also clearly this defines an inverse of
$\psi$.
\end{proof}

It remains to check that $q\i(U_x) \cong U_x \times \O(\R^n)$. Let $y = (y_1, ...,
y_2) \in q\i(U_x)$. Then, by definition and the claim, $\span(y) =
\span(x + \hat x)$ for some $\hat x \in (S^{\orth})^n$. Furthermore, the mapping
$y \to \hat x$ is continuous, since its the composite of $q$ and the map from the
claim. 

Applying the Gram-Smith algorithm on $x + \hat x$ (in the order of the indices)
like in the proof of the claim, we get an $n$-frame in $\span(y)$ depending,
continuously on $\hat x$. There is a unique element $A_y\in \O(S)$, mapping
this $n$-frame to $y$. 

We define $\Psi: q\i(U_x) \to U_x \times O(\R^n)$, by $y \mapsto (\hat x,
A_y)$, identifying $U_x$ with $(S^\orth)^n$ according to the claim and
$\O(\R^n)$ with $\O(\span(y))$ by associating the standard basis with the
$n$-frame we get by applying the Gram-Smith algorithm on $x+\hat x$.

% Projecting $y$ onto $S$ (orthogonally) we get a linearly independent
% $n$-tuple of vectors in $S$. Applying the Gram-smith algorithm (in the order of
% the indices,) we have a $n$-frame in $S$ and there is a unique element $A_y\in
% O(S)$, mapping this $n$-frame to $x$. 

% Since the orthogonal projection and
% the Gram-smith algorithm produces continues maps, its is clear that the mapping
% $y \mapsto A_y$ is continues. We defines $\Psi: q\i(U_x) \to U_x \times
% O(\R^n)$, by $y \mapsto (\hat x, A_y)$, identifying $U_x$ with $(S^\orth)^n$
% according to the claim and $\O(\R^n)$ with $\O(\span(S))$ by associating the
% standard basis with $x$.

Conversely, let $(\hat x, A_y) \in (S^\orth)^n \times O(\R^n)$. We define $\Phi:
(S^\orth)^n \times O(\R^n) \to q\i(U_x)$, by $(\hat x, A) \mapsto
A(\widetilde{x+\hat x})$, where $\til \cdot$, indicates the Gram-Smith
algorithm. This map is clearly continuous and it defines an inverse of $\Psi$.

\todo{$\Psi$ continuous}


\paragraph{b)}
In this case the same argument as in part (a) applies. The only differences being
that the fiber is $U(\C^n)$, since two $n$-frames in $\C^n$ differ by a unitary
map, rather than an orthogonal map like in part (a).

\paragraph{c)}
Let $x(x_1,...,x_m) V_m(\R^k)$. The set
$p\i(x) \se V_n(R^k)$, is given by $n$-frames $(x_1,...,x_m, y_1,..., y_{n-m})$. Clearly the
tuples $(y_1,...,y_{n-m})$ are $(n-m)$-frames in $\span(x)^\orth \cong
\R^{k-m}$. So the fiber of the fibre bundle is $V_{n-m}(\R^{k-m})$.

We identify $\span(x)^\orth$ with $\R^{k-m}$, by choosing a fixed orthonormal
basis $x' = (x'_1, ..., x'_{k-m}) \in \span(x)^\orth$.

Let $\hat U \subset \O(\R^k)$, be a "small" neighbourhood of the identity (small
in the sense that for any $A \in \hat U$ and $x \in \R^k$, $\left<x, Ax\right>
\ge 1/2$.) Then $U = \hat Ux$, defines a "small" neighbourhood of $x$.

We want to show that $p\i U \cong U \times V_{n-m}(\R^{k-m})$. 

Let $y = (y_1, ..., y_n) \in p\i U$ and $\hat y = p(y) \in U$. 

Since $\hat y$ is in the small neighbourhood of $x$, $\span(\hat y)^\orth \cap
\span(x) = 0$, so the orthogonal projection of $\span(\hat y)^\orth$ onto
$\span(x)^\orth$ defines a homeomorphism. 

By projecting the basis $x'$ onto $\span(\hat y)^\orth$ and applying the
Gram-Smith algorithm (in the order of the indices), we have a basis 
of $\span(\hat y)^\orth$, that varies continuously with $y$. 

With this identification of $\span(\hat y)^\orth \cong \R^{k-m}$, we get an
element in $y' \in V_{n-m}(\R^{k-m})$, defined by the vectors $y_{m+1}, ..., y_n$.

We define $\Psi : p\i(U) \mapsto U \times V_{n-m}(\R^{k-m})$, by $y \mapsto
(\hat y, y')$.

Conversely, let $(\hat y, y') \in U \times V_{n-m}(\R^{k-m})$. By the same
identification of $\span(\hat y)^\orth$ with $\R^{k-m}$ as above (i.e. projection
and Gram-Smith) we have orthonormal vectors $y_{m+1}, ..., y_n \in \span(\hat
y)^\orth$ defined by $y'$. Clearly the mapping $(\hat y, y') \mapsto (\hat y_1,
..., \hat y_m, y_{m+1}, ..., y_m)$, defines an inverse of $\Psi$ and its also
clearly continuous by the same argument ans for $\Psi$.

\paragraph{d)}
Like in the case of part (b), the same argument as in (c) applies also in this
case, substituting the orthogonal group with the unitary group. Giving the fibre
$V_{n-m}(\C^{k-m})$.


\paragraph{$V_n(\R^k)$, $(k-n-1)$-connected.)}
We will prove this by induction on $n$ use the long exact sequence of homotopy
groups induced by the fibre bundle $p$ form part (c). Let $x = (x_1,...,x_n) \in
V_n(\R^k)$ and $b = p(x)$. Then we have the following exact sequence
\[ \dots \to \pi_i(V_{n-m}(\R^{k-m}), x) \to \pi_i(V_n(\R^k), x) \to  \pi_i(V_m(\R^k),
b) \to \dots \]
In the case $n = 1$, we don't have such a sequence since $1 \le m < n$. However, 
$V_1(\R^k)$ consists of the set of unit vectors in $\R^k$, so $V_1(\R^k) \cong
S^{k-1}$, which we already know is $(k-2)$-connected.

In the case $n \ge 2$, let $m = n-1$ and $i \le k-n-1$, and we have the sequence 
\[ \dots \to \pi_i(V_1(\R^{k-m}), x) \to \pi_i(V_n(\R^k), x) \to  \pi_i(V_{n-1}(\R^k),
b) \to \dots \]
By induction $\pi_i(V_{n-1}(\R^k), b) = 0$ and since $V_1(\R^{k-m}) \cong
S^{k-m-1} = S^{k-n}$, so also $\pi_i(V_1(\R^{k-m}), b) = 0$ and thus
$\pi_i(V_n(\R^k), x) = 0$, since the sequence is exact.


\paragraph{$V_n(\C^k)$, $(2k-2n)$-connected.)}
The argument is quite similar to the case above. The main difference being that
we know have $V_1(\C^k) \cong S^{2k-1}$. The case $n=1$ follows immediately from
this fact, since $S^{2k-1}$ is $(2k-2)$-connected.

For $n\ge 2$, we let $m=n-1$ and $i\le 2k-2n$, then we have the exact sequence
\[ \dots \to \pi_i(V_1(\C^{k-m}), x) \to \pi_i(V_n(\C^k), x) \to  \pi_i(V_{n-1}(\C^k),
b) \to \dots \]
By induction $\pi_i(V_{n-1}(\C^k)$ and $V_1(\C^{k-m} \cong S^{2k-2n+1}$, so
$\pi_i(V_1(\C^{k-m}), x) = 0$. Hence $\pi_i(V_n(\C^k), x) = 0$, and thus
$V_n(\C^k)$ is $(2k-2n)$-connected.


\paragraph{$\pi_{k-n}(V_n(\R^k))$.)}
For the case $n = k$, we have $\pi_0(V_n(\R^{k})) = \{0\}$, since $V_n(\R^{k})$
is connected.

\todo{This might need proving}

Since $V_1(\R^{k}) \cong S^{k-1}$, we have $\pi_{k-1}(V_1(\R^{k})) = \Z$, for
all $k \ge 2$. 

For $n \ge 2$, let $m = n-1$, then we have the exact sequence 
\[ \dots \to \pi_{k-n}(V_1(\R^{k-m}), x) \xto{i_*} \pi_{k-n}(V_n(\R^k), x) \to
\pi_n(V_{n-1}(\R^k),b) \to \dots \]
By induction the last term is zero and since $V_1(\R^{k-m}) \cong S^{k-n}$ the
first term is $\Z$. Since the last term is zero the map $i_*$ most be
surjective. For $k \ge n+2$, we know, by the Hurewicz theorem, that
$\pi_{k-n}(V_n(\R^k), x) \cong H_n(V_n(\R^k); \Z)$, so in particular free. So
\todo{it might be 0}
$\pi_{k-n}(V_n(\R^k), x) \cong \Z$. For $k = n+1$, $V_n(\R^k) \cong S^{k-1}$, so 
$\pi_1(V_n(\R^k), x) = 0$, since $k-1 = n \ge 2$.


% \paragraph{$pi_{k-n}(V_n(\C^k))$}


To compute $\pi_{k-n}(V_n(\R^k))$, we consider a different segment of the
sequence with $i = m = n-1 \ge 1$. That is 
\[ \dots \to \pi_n(V_1(\R^{k-m}), x) \to \pi_n(V_n(\R^k), x) \to  \pi_n(V_{n-1}(\R^k),
b) \to \dots \]



\end{exercise}

\begin{exercise}[2](Unfinished, for lack of time)

\paragraph{a)} The canonical projection $\pi:G\to G/H$ is an open map. Indeed, $\pi^{-1}(\pi(U))=UH$ which is open since it is a union of sets of the form $Uh$ for $h$ in $H$ and right multiplication is an homeomorphism. Then the map $\pi\times\pi:G\times G\to G/H\times G/H$ is also open. Now, consider the set $V=\cb{(x,y)\in G\times G\mid x^{-1}y\in G\sm H}$: this is open because it is the preimage of the open set $G\sm H$ under the map $f:G\times G\to G$, $(x,y)\mapsto x^{-1}y$ and it maps to the complement of the diagonal in $G/H$ under $\pi\times\pi$. Therefore, the diagonal in $\pi\times\pi$ is closed, hence $G/H$ is Hausdorff.

\paragraph{b)}

\end{exercise}

\begin{exercise}[3](Unfinished, for lack of time.)

Clearly, once the claim that $\pi_n(B,b)\cong\pi_n(E,x)\times\pi_{n-1}(F,x)$ if $F\to E$ is homotopic to a constant map is shown, considering the Hopf fibrations $S^7\to S^4$ with fibre $S^3$ and $S^{15}\to S^8$ with fibre $S^7$, we will have: \[\pi_7(S^4,z)\cong\pi_7(S^7,z)\times\pi_6(S^3,z)\cong\Z\times\pi_6(S^3,z)\]
\[\pi_{15}(S^8,z)\cong\pi_{15}(S^{15},z)\times\pi_{14}(S^7,z)\cong\Z\times\pi_{14}(S^7,z)\]

\end{exercise}

\end{document}
