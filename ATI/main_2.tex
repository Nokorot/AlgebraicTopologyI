\documentclass[a4paper,11pt,english]{article}
\usepackage{.styles/basic}
\usepackage{.styles/envs}

%%%%%%% Title %%%%%%%%%%%%%%%%%%%%%%%%%%%
\title{\textbf{Algebraic Topology} - Exercise Sheet 2}
\author{Tor Gjone (2503108) \& Michele Lorenzi (3461634)}

%%%%%%% Definitions %%%%%%%%%%%%%%%%%%%%%

% It's here because it's messing up my syntax highlighting 
%% NOTE: The align* environment doesn't work for some reason.
% this is a temporary fix
\renewenvironment{align*}{
\[ \arraycolsep=2pt \def\arraystretch{1.5}
\begin{array}{rl}
}{ \end{array} \] }


% Michele's stuff
% (I hope I don't break something...)
% (Most of the command are thought for livetexing but some became habits)
% ( :D don't worry. I'm a huge fan of shortcuts myself )

\makeatletter 
% \ni was used in the definitions of other commands. This is a solution
\newcommand{\@ni}{{n-1}}

\newcommand{\nmi}{{n-1}}
\newcommand{\npi}{{n+1}}
\newcommand{\nii}{\@ni} % \ni is an inverted inclusion

\newcommand{\mi}{{m-1}}
\newcommand{\ki}{{k-1}}

\newcommand{\smsh}{\wedge}
\newcommand{\pin}{\pi_n(X,x_0)}
\newcommand{\pini}{\pi_\@ni(X,x_0)}
\newcommand{\pinr}{\pi_n(X,A,x_0)}
\newcommand{\pinir}{\pi_\@ni(X,A,x_0)}
\newcommand{\hn}{H_n(X;\Z)}
\newcommand{\hni}{H_\@ni(X;\Z)}
\newcommand{\hnr}{H_n(X,A;\Z)}
\newcommand{\hnir}{H_\@ni(X,A;\Z)}
\newcommand{\dn}{D^n}
\newcommand{\dni}{D^\@ni}
\newcommand{\sn}{S^n}
\newcommand{\sni}{S^{n-1}}
\newcommand{\sph}{(D^n,\de D^n)}
\newcommand{\sphi}{(D^\@ni,\de D^\@ni)}
\newcommand{\sm}{\smallsetminus}
\newcommand{\til}{\tilde}
\newcommand{\fa}{\;\;\forall}
\newcommand{\ul}[1]{\,\underline{#1}\,}
\newcommand{\ol}{\overline}
\newcommand{\nf}{\normalfont}
\newcommand{\xs}{{x_1,\dots,x_n}}
\newcommand{\xso}{{x_0,\dots,x_n}}
\newcommand{\nn}[2]{{{#1}_1,\dots,{#1}_{#2}}}
\newcommand{\nno}[2]{{{#1}_0,\dots,{#1}_{#2}}}
\newcommand{\nns}[3]{{{#1}_1\,{#2}\,\dots\,{#2}\,{#1}_{#3}}}
\newcommand{\nnso}[3]{{{#1}_0\,{#2}\,\dots\,{#2}\,{#1}_{#3}}}
\newcommand{\cb}[1]{\{#1\}}
\newcommand{\de}{\partial}
%\newcommand{\xto}[1]{\xrightarrow{#1}}
\newcommand{\hto}{\hookrightarrow}
\newcommand{\cont}{\reflectbox{$\in$}}
\DeclareMathOperator{\Colim}{colim}
\DeclareMathOperator{\tel}{tel}
\DeclareMathOperator{\proj}{proj}
\DeclareMathOperator{\ev}{ev}


% Tor's stuff
\newcommand{\orth}{\bot}
\newcommand{\se}{\subseteq}
\renewcommand{\ss}{\subset}
\renewcommand{\i}{^{-1}}
\newcommand{\oc}[1]{\overset{\circ}{#1}}
\newcommand{\into}{\hookrightarrow}
\newcommand{\onto}{\twoheadrightarrow}
\newcommand{\nb}{\nabla}
\renewcommand{\hat}{\widehat}
\newcommand{\wtil}{\widetilde}

\usepackage{scalerel,stackengine}
\stackMath
\renewcommand\widehat[1]{%
\savestack{\tmpbox}{\stretchto{%
  \scaleto{%
    \scalerel*[\widthof{\ensuremath{#1}}]{\kern.1pt\mathchar"0362\kern.1pt}%
    {\rule{0ex}{\textheight}}%WIDTH-LIMITED CIRCUMFLEX
  }{\textheight}% 
}{2.4ex}}%
\stackon[-6.9pt]{#1}{\tmpbox}%
}

\makeatletter
\newsavebox{\@Wedge}
\sbox\@Wedge{%
% $\mathlarger{ \mathlarger{\mathlarger{ \mathlarger{ \wedge }}}}$ }
\scalebox{1.5}{\raisebox{-1pt}{$\wedge$}} }
\newcommand{\Wedge}{\usebox{\@Wedge}}
\makeatother


% Direct limit and colimit
\makeatletter
\newcommand{\lim@}[2]{%
  \vtop{\m@th\ialign{##\cr
    \hfil$#1\operator@font lim$\hfil\cr
    \noalign{\nointerlineskip\kern1.5\ex@}#2\cr
    \noalign{\nointerlineskip\kern-\ex@}\cr}}%
}
\newcommand{\colim}{%
  \mathop{\mathpalette\varlim@{\rightarrowfill@\scriptscriptstyle}}\nmlimits@
}
\newcommand{\invlim}{%
  \mathop{\mathpalette\varlim@{\leftarrowfill@\scriptscriptstyle}}\nmlimits@
}
\makeatother


\newcommand{\diagSquare}[8]{
\begin{tikzpicture}[node distance=2cm, auto]
\node (a)              { $ #5 $ };
\node (b) [right of=a] { $ #6 $ };
\node (c) [below of=a] { $ #7 $ };
\node (d) [right of=c] { $ #8 $ };
\draw[-to] (a) to node { $ #1 $ } (b);
\draw[-to] (a) to node { $ #2 $ } (c);
\draw[-to] (b) to node { $ #3 $ } (d);
\draw[-to] (c) to node { $ #4 $ } (d);
\end{tikzpicture}
}

\newcommand{\simto}{\xrightarrow{\sim}}
\newcommand{\incto}[1][]{ \xhookrightarrow{#1} }
\newcommand{\xto}[1]{\xrightarrow{#1}}

% Text key words
\newcommand{\tif}{\text{if }}
\newcommand{\tand}{\text{and }}
\newcommand{\tsince}{\text{since }}


% Projective spaces
\def\RP{\mathbb{RP}}
\def\CP{\mathbb{CP}}
\def\HP{\mathbb{HP}}

% Common Categories
\DeclareMathOperator{\Top}   {\bf Top}
\DeclareMathOperator{\Ab}    {\bf Ab}
\DeclareMathOperator{\Cat}   {\bf Cat}
\DeclareMathOperator{\CAT}   {\bf CAT}
\DeclareMathOperator{\Mod}   {\bf Mod}
\DeclareMathOperator{\Ring}  {\bf Ring}
\DeclareMathOperator{\Group} {\bf Group}
\DeclareMathOperator{\sSet}  {{\bf sSet}}

% Homological Algebra
\DeclareMathOperator{\Hom}   {Hom}
\DeclareMathOperator{\Tor}   {Tor}
\DeclareMathOperator{\Ext}   {Ext}

% Common Lie Groups 
\let \O \undefined

\DeclareMathOperator{\GL}    {GL}
\DeclareMathOperator{\SU}    {SU}
\DeclareMathOperator{\U}     {U}
\DeclareMathOperator{\O}     {O}
\DeclareMathOperator{\SO}    {SO}
\DeclareMathOperator{\Sp}    {Sp}
\DeclareMathOperator{\Gr}    {Gr}

% Maths Operators
\DeclareMathOperator{\pr}    {pr}
\DeclareMathOperator{\id}    {id}
\DeclareMathOperator{\Ker}   {Ker}
\DeclareMathOperator{\im}    {Im}

% \let \span \undefined
\DeclareMathOperator{\tspan}  {span}

% Standard sets
\def \N {\mathbb{N}}
\def \Z {\mathbb{Z}}
\def \Q {\mathbb{Q}}
\def \R {\mathbb{R}}
\def \C {\mathbb{C}}
\def \H {\mathbb{H}}

\def \E {\mathbb{E}}
\def \Z {\mathbb{Z}}
\def \I {\mathbb{I}}
\def \J {\mathbb{J}}

% Vector calculus
\newcommand{\dif}[3][]{
	\ensuremath{\frac{d^{#1} {#2}}{d {#3}^{#1}}}}
\newcommand{\pdif}[3][]{
	\ensuremath{\frac{\partial^{#1} {#2}}{\partial {#3}^{#1}}}}

% Vectors and matrices
\newcommand{\mat}[1]{\begin{matrix} #1 \end{matrix}}
\newcommand{\pmat}[1]{\begin{pmatrix} #1 \end{pmatrix}}
\newcommand{\bmat}[1]{\begin{bmatrix} #1 \end{bmatrix}}

% Add space around the argument
\newcommand{\qq}[1]{\quad#1\quad}
\newcommand{\q}[1]{\:\:#1\:\:}

% Implications
\newcommand{\la} {\ensuremath{\Longleftarrow}}
\newcommand{\ra} {\ensuremath{\Longrightarrow}}
\newcommand{\lra}{\ensuremath{\Longleftrightarrow}}

\newcommand{\pwf}[1]{\begin{cases} #1 \end{cases}}

% Shorthand
\newcommand{\vphi}{\varphi}
\newcommand{\veps}{\varepsilon}
\newcommand{\eps}{\epsilon}

\newcommand{\<}[1]{\langle #1 \rangle}

% Notation
\newcommand{\wddef}[1]{\underline{#1}}
\newcommand{\pref}[1]{(\ref{#1})}

% Maths Operators
\theoremstyle{plain}
\theoremstyle{definition}
\newtheorem{thrm}{Theorem}[section]
\newtheorem{prop}[thrm]{Proposition}
\newtheorem{corol}[thrm]{Corollary}
\newtheorem{lemma}[thrm]{Lemma}

\newtheorem{defn}[thrm]{Definition}
\newtheorem{exmp}[thrm]{Example}
\newtheorem{clame}[thrm]{Clame}

\theoremstyle{remark}
% \newtheorem{remark}[thrm]{\normalfont\large\textit Remark}
\newtheorem{remark}[thrm]{Remark}
\newtheorem{note}[thrm]{Note}


\newSimpleHeaderEnvironment{exercise}{Exercise }

\setlength{\parindent}{0cm}
\setlength{\parskip}{3pt}
%%%%%%%% Content %%%%%%%%%%%%%%%%%%%%%%%%
\begin{document}

\mmaketitle

\begin{exercise}[1]

\begin{enumerate}
\item[(a)]
The universal cover $\til X$ of $X=S^1\vee S^2$ is the 
real line with a copy of $S^2$ attached at to each integer, as illustrated in the exercise sheet and in the figure below.
\begin{center}
    \includegraphics[scale=0.6]{img-000.jpg}
\end{center}

To show that this is in fact the universal cover, consider the $x$-axis in $\R^3$ with spheres (of radius $1$ say) attached every $2k\pi$ for $k\in\Z$. It is clear that this construction is simply connected, since both the sphere and the line are. The projection map is then the one that sends the points of the $x$-axis to $e^{2\pi ix}$ in $S^1$ and the point on the spheres to the corresponding points in $S^2$. 

The deck transformations will then be exactly the deck transformations of the real line as a covering of the circle, that is, the translations by $2k\pi$ along the $x$-axis, so that in particular $\pi_1(X,x_0)=\Z$ (which is also possible to show by using Van Kampen theorem).

\item[(b)] We first observe that the universal cover is homotopy equivalent to a countable wedge of spheres $\bigvee_{i\in\Z}S^2$, by contracting the line to a point. Hence we have that its second homology group is equal to the direct sum $\bigoplus_{i\in\Z}\Z$. Since the universal cover is simply-connected, we can apply Hurewicz theorem, so the second homotopy group of $\til X$ is equal to $\bigoplus_{i\in\Z}\Z$. Hence $\pi_2(X,x_0)=\bigoplus_{i\in\Z}\Z$, since the $n$-th homotopy group of any space equals that of its covering spaces for all $n \ge 2$.

\item[(c)] 
Consider the construction of $\til X$, as a subset of $\R^3$, from (a), with $x_0 = 0$ as a base point.

Let $\hat p: S^2 \to I \vee S^2$ be a map that projects the left hemi-sphere onto the central axis (see the illustration below.) This map can be though of as a pinch map followed by a map that projects the first summand of $S^2 \vee S^2$ onto the central axis.

\begin{center}
\begin{tikzpicture}
\node[above left] at (0,0) {$1$};
\fill (0,0) circle (2pt);
\node[above] at (-2,0) {$0$};
\fill (-2,0) circle (2pt);
\draw (-2,0) -- (0, 0);
\draw (-2,0) -- (0, 0);
\draw (1,0) circle (1);
\draw (1,0) ellipse (1 and 0.25);
\end{tikzpicture}
\end{center}


We want to consider $\hat p$ as a based map sending the base point in $S^2$ to the beginning of the interval, that is the point that is not attached to the sphere and we will denote by $0$ (like in the figure).

For $k\in \Z$, let $\iota_k : I \vee S^2$ be the maps defined by $x \mapsto (2\pi k x)$ for $x \in I =[0,1]$ and on the sphere by inclusion into the sphere attached to the point $2\pi k$ on the $x$-axis.

Then by composing $\iota_k$ with the projection $\pi : \til X \to X$, we have a family of continuous maps that generate $\pi_2(X, x_0)$.

It remains to determine the action of $\pi_1(X,x_0)$ on $\pi_2(X,x_0)$. 
The action is defined precisely in terms of the map $\hat p$. 
Let $[\omega] \in \pi_1(X,x_0)$ and $[\xi] \in \pi_2(X,x_0)$, then the action
of $\omega$ on $\xi$, is the map 
\[ \omega \star \xi := (\omega \vee \xi) \circ \hat p. \] 
That is, the map $\hat p$ followed by acting with $\omega$ on the interval and $\xi$ on the sphere such that $1 \in I$ is mapped to $x_0$.

In particular, the action of $\omega$ on $\hat \iota_k = \pi \circ \iota_k$ is the map 
\[ \omega \star \hat \iota_k = (\omega \vee (\hat\iota_k)) \circ \hat p. \]
Since $\omega$ is a loop at $x_0$. There is a unique lift $\til\omega$ of $\omega$ to the universal cover such that $\omega(0) = 0 \in \til X \subset \R^3$. Then we have
\[ \omega \star \hat \iota_k = \pi \circ (\til \omega \vee \iota_k) \circ \hat p. \]
Let $l = \til\omega(1) / 2\pi \in \Z$, then clearly $\til \omega \vee \iota_k = \iota_{k+l}$. So 
\[ \omega \star \hat \iota_k = \hat \iota_{k+l}. \]

\end{enumerate}
\end{exercise}

\newpage

\begin{exercise}[2]

We first prove a couple of preliminary lemmas.

\ 

\textbf{Claim 1.} The suspension $\Sigma X$ of a path connected space $X$ is simply connected.

\begin{proof}

The claim follows from an application of Seifert-Van Kampen theorem: just consider the two contractible open subsets of $\Sigma X$ which are the complement of the points to which $X\times\cb{0}$ and $X\times\cb{1}$ are identified (note that their intersection is path connected, since $X$ is path connected, so that the hypothesis of the Seifert-Van Kampen theorem are met).

%Let $[\phi] \in \pi_1(\Sigma X, x_0)$ and $U = \phi^{-1}(X \times (0,1)) \subset I$. Then we have $U = \sqcup_{i \in J} (a_i, b_i)$, where $a_i < b_i$.

%In fact $I$ is finite. For each $i \in I$ choose $c_i \in (a_i, b_i)$ and let $V = I \setminus \{ c_i  \}_i$, then $ V \cup \{ U_i \}_i$ is a cover of $I$ so there must be a finite sub cover. However $c_i$ is only contained in $U_i$ so, $I$ must be finite.

%Fix $x_0 \in X$, let $i \in I$ and $\hat\phi_i : (0,1) \to X\times (0,1)$, be defined by $\hat\phi_i(x) = \phi|_{U_i}\left(\frac{x+a_i}{b_i-a_i}\right)$. Then define $\til\phi_i = p_1 \circ \hat\phi_i$, where $p_1$ is the projection of $X \times (0, 1)$ onto $X$.
%Since $X$ is path connected, there exists a path $\xi: I \to X$, such that $\xi(0) = \til\phi_i(1/2)$ and $\xi(1) = x_0$. Let $K : I \times (0,1) \to (0,1)$ be a contraction of $(0,1)$ onto the point $\{1/2\}$. Then,
%for each $x \in (0,1)$, we define $\phi^x_i = (\xi \star \hat\phi_i)(K(-,x))$, where $\star$ denotes concatenation of paths. In words that is, $\phi^x_i$ is the path following the $X$-component of $\hat\phi_i$ to the middle of the suspension and then following $\xi$ to $x_0$.

%We define $H_i: I\times(0,1) \to X\times (0,1)$ by 
%\[ H_i(t,x) := ( \phi^x_i(t), x ) = ( (\xi \star \hat\phi_i) \circ K)(t,x), x). \]
%The last expression is clearly continues, so $H_i$ is continues. 

%Finally we define $H: I\times(0,1) \to \Sigma X$, by 
%\[ H(t,x) = \pwf{ H_i\left(t, \frac{x+a_i}{b_i-a_i}\right)  &\tif x \in U_i \text{ for some } i \in I \\ 
 %                 \phi(x) &\text{otherwise}  } \]
%This map is continues since $H_i$ fixes the second argument and thus points close to $\phi(x) \in X \times \{0,1\}$, stay close to $\phi(x)$. So $H$ defines a homotopy from $\phi$ to a path contained in $\{x_0\}\times I \subset \Sigma X$, which is clearly contractible and thus $\phi$ contracts to a trivial path.
\end{proof}

\textbf{Claim 2.} For any homology theory there is a natural isomorphism
\[\til H_i(X)\to\til H_{i+1}(\Sigma X)\]

\begin{proof}
This follows from the reduced Mayer–Vietoris sequence
\[ \dots \to \til H_{n+1}(U) \oplus \til H_{n+1}(V) \xto{i_* - j_*} \til H_{n+1}(X) \xto{\de_*} \til H_n(U \cap V) \to \til H_n(U) \oplus \til H_n(V)  \to \dots \]
where $U,V \subset X$ are non-empty and cover $X$, $i: U \into X$, $j: V \into X$.

Let $X = \Sigma X$ and 
\[\Sigma X \supset U = \left(X \times [0, 3/4) \right)/\!\!\sim\
\qq{\text{and}}
\Sigma X \supset V= (X \times (1/4,0] )/\!\!\sim.\]
\vspace{-.5cm}
\begin{figure}[h!] 
\centering
\begin{tikzpicture}[scale=.35]
\node[above] at (1,3) {$U$};
\draw [decorate,
    decoration = {brace}] (-2.5,2.5) --  (5,2.5) ;
\node[below] at (-1.75,-3) {$V$};
\draw [decorate,
    decoration = {brace, mirror}] (-5,-2.5) --  (2.5,-2.5);
\draw (-5,0) -- (0,-2);
\draw (-5,0) -- (0,2);
\draw (0,-2) -- (5,0);
\draw (0,2) --  (5,0);
\draw (0,0) ellipse (.5 and 2);
\draw (-2.5,0) ellipse (.25 and 1);
\draw (2.5,0) ellipse (.25 and 1);
\end{tikzpicture}
\end{figure}
%\end{center}

Then $\til H_{n+1}(U) \oplus \til H_{n+1}(V) = 0$, since both $U$ and $V$ are contractible. And 
$\til H_n(U \cap V) \cong \til H_n(X)$, since $U \cap V = X \times (1/4,3/4)$ is homotopy equivalent to $X$. So the the reduced Mayer–Vietoris sequence looks like 
\[ \dots \to 0 \to \til H_{n+1}(\Sigma X) \to \til H_n(X) \to 0 \to \dots \]
\end{proof}

If $X$ is an acyclic CW complex, then it is path-connected (because $\til H_0(X;\Z)=0$), hence the first claim tells us that its suspension $\Sigma X$ is simply-connected. Moreover, $\Sigma X$ has trivial reduced integral homology groups by the second claim. Since $\Sigma X$ is simply-connected, we can apply Hurewicz theorem, finding that all its homotopy groups are trivial. By Whitehead theorem, since the inclusion of a zero cell induces isomorphisms on all the homotopy groups of $\Sigma X$, $\Sigma X$ deformation retracts to the zero cell.
\end{exercise}

\newpage

\begin{exercise}[3](More an attempt (for (a)) than a solution, but I need some sleep)

(a) We recall that the colimit of the homology groups $H_*(X_k;A)$ is just the quotient of the direct sum $\bigoplus_{k\geq0}H_*(X_k;A)$ by the subgroup generated by the elements $\iota_i(a_i)-\iota_{i+1}(f_i(a_i))$ (where $\iota_i$ is the canonical inclusion $H_*(X_k;A)\hto \bigoplus_k H_*(X_k;A)$).

We want to calculate the homology of $\tel_n X_n$ in a way that will yield the object we just described. We define for $l\geq 1$:
\begin{align*}
U_l&=\cb{(x,t,2l-3)\in\tel_n X_n\mid t>1/3}\cup\cb{(x,t,2l-2)\in\tel_n X_n\mid t<2/3} \\
V_l&=\cb{(x,t,2l-2)\in\tel_n X_n\mid t>1/3}\cup\cb{(x,t,2l-1)\in\tel_n X_n\mid t<2/3}
\end{align*}
where $l$ is the index of the disjoint union in the definition of $\tel_n X_n$ (and $U_1$ is defined differently as $\cb{(x,t,0)\in\tel_n X_n\mid t<2/3}$). Then the $U_l$ are disjoint open sets in $\tel_n X_n$ homotopy equivalent to $X_{2l-2}$, while the $V_l$ are disjoint open sets equivalent to $X_{2l-1}$. Let now $U$ be the union of the $U_l$, $V$ the union of the $V_l$, we have that $U\cap V$ is the disjoint union $\coprod_{l\geq1}U_l\cap V_l$ which is homotopy equivalent to the disjoint union $\coprod_{l\geq0}X_l$, which has homology groups $H_*(\coprod X_l)=\bigoplus_l H_*(X_l)$. Similarly, we have
\begin{align*}
    H_*(U;A)&=\cb{(a_0,0,a_2,0,\dots)\in\bigoplus_l H_*(X_l;A)\mid a_{2i}\in H_*(X_{2i};A)} \\
    H_*(V;A)&=\cb{(0,a_1,0,a_3,\dots)\in\bigoplus_l H_*(X_l;A)\mid a_{2i+1}\in H_*(X_{2i+1};A)}
\end{align*}

We then consider the Mayer-Vietoris sequence associated to the pair $U,V$ of subspaces of $X$:
\begin{center}
    \begin{tikzcd}
    \dots \arrow[r] & H_*(U\cap V;A) \arrow[r,"\Phi"] & H_*(U;A)\oplus H_*(V;A) \arrow[r] & H_*(\tel_n X_n;A) \arrow[r] & \dots
    \end{tikzcd}
\end{center}
where the map $\Phi$ is the map defined as the sum of the map induced by the inclusions of $U\cap V$ respectively in $U$ and $V$, that is:
\[\Phi(a_0,a_1,\dots)=(a_0,a_1+f_{0*}(a_0),a_2+f_{1*}(a_1),\dots)\]

This map clearly has trivial kernel, hence it is injective, so that $H_*(\tel_n X_n;A)$ is isomorphic to its cokernel. Now, the image of $\Phi$ is generated by:
\[\Phi(0,\dots,0,a_i,0,\dots)=(0,\dots,0,a_i,f_{i*}(a_i),0,\dots)\]
so that by composing with the automorphism of  $(x_i)\mapsto((-1)^ix_i)$ we find an isomorphism between $H_*(\tel_n X_n;A)$ and $\colim_k H_*(X_k;A)$, hence that we have proved that the maps $(i_k)_*:H_*(X_k;A)\to H_*(\tel_n X_n;A)$ induce an isomorphism $\colim_k H_*(X_k;A)\to H_*(\tel_n X_n;A)$.

\end{exercise}

\end{document}
