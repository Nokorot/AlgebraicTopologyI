\documentclass[a4paper,11pt,english]{article}
\usepackage{.styles/basic}
\usepackage{.styles/envs}
\frenchspacing

\usepackage{circuitikz}

\usepackage{caption}
\usepackage{subcaption}

%%%%%%% Title %%%%%%%%%%%%%%%%%%%%%%%%%%%
\title{\textbf{Algebraic Topology} - Exercise Sheet 8}
\author{Tor Gjone (2503108) \& Michele Lorenzi (3461634)}

%%%%%%% Definitions %%%%%%%%%%%%%%%%%%%%%

% Michele's stuff
% (I hope I don't break something...)
% (Most of the command are thought for livetexing but some became habits)
% ( :D don't worry. I'm a huge fan of shortcuts myself )

\makeatletter 
% \ni was used in the definitions of other commands. This is a solution
\newcommand{\@ni}{{n-1}}

\newcommand{\pin}{\pi_n(X,x_0)}
\newcommand{\pini}{\pi_\@ni(X,x_0)}
\newcommand{\pinr}{\pi_n(X,A,x_0)}
\newcommand{\pinir}{\pi_\@ni(X,A,x_0)}
\newcommand{\hn}{H_n(X;\Z)}
\newcommand{\hni}{H_\@ni(X;\Z)}
\newcommand{\hnr}{H_n(X,A;\Z)}
\newcommand{\hnir}{H_\@ni(X,A;\Z)}
\newcommand{\dn}{D^n}
\newcommand{\dni}{D^\@ni}
\newcommand{\sn}{S^n}
\newcommand{\sni}{S^{n-1}}
\newcommand{\sph}{(D^n,\de D^n)}
\newcommand{\sphi}{(D^\@ni,\de D^\@ni)}
\newcommand{\sm}{\smallsetminus}
\newcommand{\til}{\tilde}
\newcommand{\fa}{\;\;\forall}
\newcommand{\ul}[1]{\,\underline{#1}\,}
\newcommand{\ol}{\overline}
\newcommand{\nf}{\normalfont}
\newcommand{\xs}{{x_1,\dots,x_n}}
\newcommand{\xso}{{x_0,\dots,x_n}}
\newcommand{\nn}[2]{{{#1}_1,\dots,{#1}_{#2}}}
\newcommand{\nno}[2]{{{#1}_0,\dots,{#1}_{#2}}}
\newcommand{\nns}[3]{{{#1}_1\,{#2}\,\dots\,{#2}\,{#1}_{#3}}}
\newcommand{\nnso}[3]{{{#1}_0\,{#2}\,\dots\,{#2}\,{#1}_{#3}}}
\newcommand{\cb}[1]{\{#1\}}
\newcommand{\de}{\partial}
%\newcommand{\xto}[1]{\xrightarrow{#1}}
\newcommand{\hto}{\hookrightarrow}
\newcommand{\nii}{\@ni}
\newcommand{\cont}{\reflectbox{$\in$}}
\newcommand{\mi}{{m-1}}
\newcommand{\ki}{{k-1}}
\DeclareMathOperator{\colim}{colim}
\DeclareMathOperator{\tel}{tel}
\DeclareMathOperator{\proj}{proj}


% Tor's stuff
\newcommand{\oc}[1]{\overset{\circ}{#1}}
\newcommand{\into}{\hookrightarrow}
\newcommand{\nb}{\nabla}
\renewcommand{\hat}{\widehat}

\usepackage{scalerel,stackengine}
\stackMath
\renewcommand\widehat[1]{%
\savestack{\tmpbox}{\stretchto{%
  \scaleto{%
    \scalerel*[\widthof{\ensuremath{#1}}]{\kern.1pt\mathchar"0362\kern.1pt}%
    {\rule{0ex}{\textheight}}%WIDTH-LIMITED CIRCUMFLEX
  }{\textheight}% 
}{2.4ex}}%
\stackon[-6.9pt]{#1}{\tmpbox}%
}


\newcommand{\diagSquare}[8]{
\begin{tikzpicture}[node distance=2cm, auto]
\node (a)              { $ #5 $ };
\node (b) [right of=a] { $ #6 $ };
\node (c) [below of=a] { $ #7 $ };
\node (d) [right of=c] { $ #8 $ };
\draw[-to] (a) to node { $ #1 $ } (b);
\draw[-to] (a) to node { $ #2 $ } (c);
\draw[-to] (b) to node { $ #3 $ } (d);
\draw[-to] (c) to node { $ #4 $ } (d);
\end{tikzpicture}
}

\newcommand{\congto}{\xrightarrow{\cong}}
\newcommand{\incto}[1][]{ \xhookrightarrow{#1} }
\newcommand{\xto}[1]{\xrightarrow{#1}}

% Text key words
\newcommand{\tif}{\text{if }}
\newcommand{\tand}{\text{and }}
\newcommand{\tsince}{\text{since }}


% Projective spaces
\def\RP{\mathbb{RP}}
\def\CP{\mathbb{CP}}
\def\HP{\mathbb{HP}}

% Common Categories
\DeclareMathOperator{\Top}   {\bf Top}
\DeclareMathOperator{\Ab}    {\bf Ab}
\DeclareMathOperator{\Cat}   {\bf Cat}
\DeclareMathOperator{\CAT}   {\bf CAT}
\DeclareMathOperator{\Mod}   {\bf Mod}
\DeclareMathOperator{\Ring}  {\bf Ring}
\DeclareMathOperator{\Group} {\bf Group}
\DeclareMathOperator{\sSet}  {{\bf sSet}}

% Homological Algebra
\DeclareMathOperator{\Hom}   {Hom}
\DeclareMathOperator{\Tor}   {Tor}
\DeclareMathOperator{\Ext}   {Ext}

% Common Lie Groups 
\DeclareMathOperator{\GL}    {GL}
\DeclareMathOperator{\SU}    {SU}
\DeclareMathOperator{\U}     {U}
\DeclareMathOperator{\Sp}    {Sp}

% Maths Operators
\DeclareMathOperator{\pr}    {pr}
\DeclareMathOperator{\id}    {id}
\DeclareMathOperator{\Ker}   {Ker}
\DeclareMathOperator{\im}    {Im}

% Standard sets
\def \N {\mathbb{N}}
\def \Z {\mathbb{Z}}
\def \Q {\mathbb{Q}}
\def \R {\mathbb{R}}
\def \C {\mathbb{C}}
\def \H {\mathbb{H}}

\def \E {\mathbb{E}}
\def \Z {\mathbb{Z}}
\def \I {\mathbb{I}}
\def \J {\mathbb{J}}

% Vector calculus
\newcommand{\dif}[3][]{
	\ensuremath{\frac{d^{#1} {#2}}{d {#3}^{#1}}}}
\newcommand{\pdif}[3][]{
	\ensuremath{\frac{\partial^{#1} {#2}}{\partial {#3}^{#1}}}}

% Vectors and matrices
\newcommand{\mat}[1]{\begin{matrix} #1 \end{matrix}}
\newcommand{\pmat}[1]{\begin{pmatrix} #1 \end{pmatrix}}
\newcommand{\bmat}[1]{\begin{bmatrix} #1 \end{bmatrix}}

% Add space around the argument
\newcommand{\qq}[1]{\quad#1\quad}
\newcommand{\q}[1]{\:\:#1\:\:}

% Implications
\newcommand{\la} {\ensuremath{\Longleftarrow}}
\newcommand{\ra} {\ensuremath{\Longrightarrow}}
\newcommand{\lra}{\ensuremath{\Longleftrightarrow}}

\newcommand{\pwf}[1]{\begin{cases} #1 \end{cases}}

% Shorthand
\newcommand{\vphi}{\varphi}
\newcommand{\veps}{\varepsilon}

\newcommand{\<}[1]{\langle #1 \rangle}

% Notation
\newcommand{\wddef}[1]{\underline{#1}}
\newcommand{\pref}[1]{(\ref{#1})}

% Maths Operators
\theoremstyle{plain}
\theoremstyle{definition}
\newtheorem{thrm}{Theorem}[section]
\newtheorem{prop}[thrm]{Proposition}
\newtheorem{corol}[thrm]{Corollary}
\newtheorem{lemma}[thrm]{Lemma}

\newtheorem{defn}[thrm]{Definition}
\newtheorem{exmp}[thrm]{Example}
\newtheorem{clame}[thrm]{Clame}

\theoremstyle{remark}
% \newtheorem{remark}[thrm]{\normalfont\large\textit Remark}
\newtheorem{remark}[thrm]{Remark}
\newtheorem{note}[thrm]{Note}


\newSimpleHeaderEnvironment{exercise}{Exercise }

\setlength{\parindent}{0cm}
\setlength{\parskip}{3pt}

\usepackage{enumerate}

\DeclareMathOperator{\holim}{holim}

% I keep adding \ in from of the P's. This makes sure I get an error.
\def\P{\undefined}

%%%%%%%% Content %%%%%%%%%%%%%%%%%%%%%%%%
\begin{document}
\mmaketitle

\begin{exercise}[1]
\begin{enumerate}[(a)]

\item %a
Let $[\xi] \in \pi_m(\holim_n P_n)$. We have
\begin{equation}
\pi_m(q_k)([\xi]) = [q_k \circ \xi] \q\tand 
\Psi \circ \pi_m(q_{k+1}) ([\xi]) = (\omega_k)_\star [p_k \circ q_{k+1} \circ
\xi]
\end{equation}
where $q_k \circ \xi$ is a loop based at $\omega_k(0)$ and $p_k \circ q_{k+1}
\circ \xi$ a loop based at $\omega_k(1)$, both in $P_k$. 

By extending the second loop by $\omega_k$ (ie. applying $(\omega_k)_\star$)
both loops are based at $\omega_k(0)$.

We want to show that these two maps are base homotopic. 

Consider $\xi(-)_k : S^m \to P_k[0,k]$. We denote the adjoint map by 
$\hat \xi_k : S^m \times [0,1] \to P_k$. 
Then $\hat \xi_k$ defines an unbased homotopy from $q_k \circ \xi$ to $p_k \circ q_{k+1} \circ
\xi$. Indeed, for $x \in S^k$,
\begin{align*}
(q_k \xi)(x)_k &=& \xi(x)_k(0) = \hat \xi_k(x,0)& \q{\tand} \\
(p_k \circ q_{k+1} \circ \xi) (x) &=& p_k(\xi(x)_{k+1} (0)) = \xi(x)_k(1) = \hat
\xi_k(x,1)
\end{align*}

Furthermore 
\begin{equation}
\hat \xi_k (1,-) = \xi(1)_k = \omega_k. 
\label{eq:2}
\end{equation}

We define $H : S^m \times [0,1]$ 
\[ H(-,t) := (\omega_k|_{[0,t]})_\star (\hat \xi_k(-,t)). \]
%
%\todo{add a figure}Note that $H$ is continues by \pref{eq:2}. Then $H$ defines a based (at
$\omega_k(0)$) homotopy from $q_k \xi$ to $\omega_\star (p_k \circ q_{k+1}
\circ \xi)$. 

\item %b
Let $\xi = ([\xi_n])_{n\ge 0} \in \lim_n \pi_m(P_n, \omega_n(0))$, i.e. there is $[\xi_n] \in
\pi_m(P_n, \omega_n(0))$ such that $\Psi_n([\xi_{n+1}]) = [\xi_n]$ for all $n\ge
0$. 

Let $H_n : S^m \times [0,1] \to P_n$ be a homotopy realizing $(\omega_n)_\star
(p_n \circ \xi_{n+1}) \sim \xi_n$  (based at $\omega_n(0)$).

Since the homotopy is based, it factors through the quotient \[V = S^m \times [0,1] /
\left(\{x_0\} \times [0,1]\right)\]
($x_0\in S^m$ denotes the base point). Let
$H_n': V \to P_n$ be the map defined by $H_n$. 

By the construction of the conjugation $(\omega_n)_\star$, we may defines a
homeomorphism $h_n : S^m \times [0,1] \to V$, such that $\wtil H_n = H_n' \circ
h_n$ defines a homotopy from $\xi_n$ to $p_n \circ \xi$ and $\wtil H_n(x_0,-) =
\omega_n$. 

% Se figure \pref{fig:2} for a illustration of the construction in
% dimension $m=1$.
% \todo{add fig}

By the exponential adjunction $\til H_n$ defines a map $\bar H_n: S^m \to
P_n^{[0,1]}$. 

By the construction of $\bar H_n$, $\bar H_n(x_0) = \omega_n$ and $\bar H_{n+1}
(x)(1) = p( \xi_{n}(x) ) = p (\bar H_n(x)(0))$, so $\bar H = (\bar H_n)_{n \ge
0}$ represent an element in $\pi_m(\holim_n P_n,\omega)$.

Also $q_n (\bar H(x)) = \bar H_n(x)(0) = \xi_n (x)$ for all $x\in S^m$, so
$\pi_m(q_n)(\bar H) = \xi$.

\item %c

\end{enumerate}
\end{exercise}

\begin{exercise}[2]

(a) This follows from the third exercise on the second exercise sheet (specifically, the second point), since then we have that there is an isomorphism between $\pi_m(T,i_0(x_0))$, where $T$ is the mapping telescope, and $\colim_k\pi_m(X_k,x_k)$. Now, clearly when $n\ne m$ we have that the colimit is zero, otherwise it is the colimit of the sequence of groups given.

(b) $S^1$ is a $K(\Z,1)$ and the map $f$ induces multiplication by $n$, hence we need to find the colimit of the sequence:
\[\Z\xto{n}\Z\xto{n}\cdots\xto{n}\Z\xto{n}\cdots.\]
This is just $\Z[n^{-1}]$, the localization of the integers at $n$ (as an abelian group), and this can be easily seen by showing that there is a morphism from the colimit to the localization, using the universal property of the colimit, and then showing that it is an isomorphism.
\end{exercise}

\begin{exercise}[3]

Let $X = M_n(A)$ be a Moore space with $H_n(X,\Z) \cong A$. 
By the Hurewicz Theorem, $\pi_k(X) = 0$ for $k \le n-1$ and $\pi_n(X) = A$, so 
we may construct an Eilenberg-MacLane, space $\til X$, starting with $X$ and
inductively "kill" the higher homotopy groups. 

Since $\pi_k(X)$ is "correct" (that is, it's what we want of the final space) 
the first homotopy group we might need to kill is in dimension $k=n+1$. 
In order to kill this homotopy we attach $n+2$ cells to $X$, which by cellular
approximation does not effect the homology in dimension $n+1$ and lower.

Similarly, to kill the higher homotopy groups, we attach cells of one dimension
higher. So throughout the construction of $\til X$ the homology group in dimension
$n+1$ is unchanged so we may conclude that $H_{n+1}(\til X, \Z) H_{n+1}(X, \Z) = 0$. 

Finally, since the Eilenberg-MacLane space is unique up to homotopy we may
conclude that $H_{n+1}(K(A,n),\Z) = 0$ independently of the construction of
$K(A,n)$.

\end{exercise}

\end{document}
