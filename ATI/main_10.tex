\documentclass[a4paper,11pt,english]{article}
\usepackage{.styles/basic}
\usepackage{.styles/envs}
\frenchspacing

\usepackage{circuitikz}

\usepackage{caption}
\usepackage{subcaption}

%%%%%%% Title %%%%%%%%%%%%%%%%%%%%%%%%%%%
\title{\textbf{Algebraic Topology} - Exercise Sheet 10}
\author{Tor Gjone (2503108) \& Michele Lorenzi (3461634)}

%%%%%%% Definitions %%%%%%%%%%%%%%%%%%%%%

% Michele's stuff
% (I hope I don't break something...)
% (Most of the command are thought for livetexing but some became habits)
% ( :D don't worry. I'm a huge fan of shortcuts myself )

\makeatletter 
% \ni was used in the definitions of other commands. This is a solution
\newcommand{\@ni}{{n-1}}

\newcommand{\pin}{\pi_n(X,x_0)}
\newcommand{\pini}{\pi_\@ni(X,x_0)}
\newcommand{\pinr}{\pi_n(X,A,x_0)}
\newcommand{\pinir}{\pi_\@ni(X,A,x_0)}
\newcommand{\hn}{H_n(X;\Z)}
\newcommand{\hni}{H_\@ni(X;\Z)}
\newcommand{\hnr}{H_n(X,A;\Z)}
\newcommand{\hnir}{H_\@ni(X,A;\Z)}
\newcommand{\dn}{D^n}
\newcommand{\dni}{D^\@ni}
\newcommand{\sn}{S^n}
\newcommand{\sni}{S^{n-1}}
\newcommand{\sph}{(D^n,\de D^n)}
\newcommand{\sphi}{(D^\@ni,\de D^\@ni)}
\newcommand{\sm}{\smallsetminus}
\newcommand{\til}{\tilde}
\newcommand{\fa}{\;\;\forall}
\newcommand{\ul}[1]{\,\underline{#1}\,}
\newcommand{\ol}{\overline}
\newcommand{\nf}{\normalfont}
\newcommand{\xs}{{x_1,\dots,x_n}}
\newcommand{\xso}{{x_0,\dots,x_n}}
\newcommand{\nn}[2]{{{#1}_1,\dots,{#1}_{#2}}}
\newcommand{\nno}[2]{{{#1}_0,\dots,{#1}_{#2}}}
\newcommand{\nns}[3]{{{#1}_1\,{#2}\,\dots\,{#2}\,{#1}_{#3}}}
\newcommand{\nnso}[3]{{{#1}_0\,{#2}\,\dots\,{#2}\,{#1}_{#3}}}
\newcommand{\cb}[1]{\{#1\}}
\newcommand{\de}{\partial}
%\newcommand{\xto}[1]{\xrightarrow{#1}}
\newcommand{\hto}{\hookrightarrow}
\newcommand{\nii}{\@ni}
\newcommand{\cont}{\reflectbox{$\in$}}
\newcommand{\mi}{{m-1}}
\newcommand{\ki}{{k-1}}
\DeclareMathOperator{\colim}{colim}
\DeclareMathOperator{\tel}{tel}
\DeclareMathOperator{\proj}{proj}


% Tor's stuff
\newcommand{\oc}[1]{\overset{\circ}{#1}}
\newcommand{\into}{\hookrightarrow}
\newcommand{\nb}{\nabla}
\renewcommand{\hat}{\widehat}

\usepackage{scalerel,stackengine}
\stackMath
\renewcommand\widehat[1]{%
\savestack{\tmpbox}{\stretchto{%
  \scaleto{%
    \scalerel*[\widthof{\ensuremath{#1}}]{\kern.1pt\mathchar"0362\kern.1pt}%
    {\rule{0ex}{\textheight}}%WIDTH-LIMITED CIRCUMFLEX
  }{\textheight}% 
}{2.4ex}}%
\stackon[-6.9pt]{#1}{\tmpbox}%
}


\newcommand{\diagSquare}[8]{
\begin{tikzpicture}[node distance=2cm, auto]
\node (a)              { $ #5 $ };
\node (b) [right of=a] { $ #6 $ };
\node (c) [below of=a] { $ #7 $ };
\node (d) [right of=c] { $ #8 $ };
\draw[-to] (a) to node { $ #1 $ } (b);
\draw[-to] (a) to node { $ #2 $ } (c);
\draw[-to] (b) to node { $ #3 $ } (d);
\draw[-to] (c) to node { $ #4 $ } (d);
\end{tikzpicture}
}

\newcommand{\congto}{\xrightarrow{\cong}}
\newcommand{\incto}[1][]{ \xhookrightarrow{#1} }
\newcommand{\xto}[1]{\xrightarrow{#1}}

% Text key words
\newcommand{\tif}{\text{if }}
\newcommand{\tand}{\text{and }}
\newcommand{\tsince}{\text{since }}


% Projective spaces
\def\RP{\mathbb{RP}}
\def\CP{\mathbb{CP}}
\def\HP{\mathbb{HP}}

% Common Categories
\DeclareMathOperator{\Top}   {\bf Top}
\DeclareMathOperator{\Ab}    {\bf Ab}
\DeclareMathOperator{\Cat}   {\bf Cat}
\DeclareMathOperator{\CAT}   {\bf CAT}
\DeclareMathOperator{\Mod}   {\bf Mod}
\DeclareMathOperator{\Ring}  {\bf Ring}
\DeclareMathOperator{\Group} {\bf Group}
\DeclareMathOperator{\sSet}  {{\bf sSet}}

% Homological Algebra
\DeclareMathOperator{\Hom}   {Hom}
\DeclareMathOperator{\Tor}   {Tor}
\DeclareMathOperator{\Ext}   {Ext}

% Common Lie Groups 
\DeclareMathOperator{\GL}    {GL}
\DeclareMathOperator{\SU}    {SU}
\DeclareMathOperator{\U}     {U}
\DeclareMathOperator{\Sp}    {Sp}

% Maths Operators
\DeclareMathOperator{\pr}    {pr}
\DeclareMathOperator{\id}    {id}
\DeclareMathOperator{\Ker}   {Ker}
\DeclareMathOperator{\im}    {Im}

% Standard sets
\def \N {\mathbb{N}}
\def \Z {\mathbb{Z}}
\def \Q {\mathbb{Q}}
\def \R {\mathbb{R}}
\def \C {\mathbb{C}}
\def \H {\mathbb{H}}

\def \E {\mathbb{E}}
\def \Z {\mathbb{Z}}
\def \I {\mathbb{I}}
\def \J {\mathbb{J}}

% Vector calculus
\newcommand{\dif}[3][]{
	\ensuremath{\frac{d^{#1} {#2}}{d {#3}^{#1}}}}
\newcommand{\pdif}[3][]{
	\ensuremath{\frac{\partial^{#1} {#2}}{\partial {#3}^{#1}}}}

% Vectors and matrices
\newcommand{\mat}[1]{\begin{matrix} #1 \end{matrix}}
\newcommand{\pmat}[1]{\begin{pmatrix} #1 \end{pmatrix}}
\newcommand{\bmat}[1]{\begin{bmatrix} #1 \end{bmatrix}}

% Add space around the argument
\newcommand{\qq}[1]{\quad#1\quad}
\newcommand{\q}[1]{\:\:#1\:\:}

% Implications
\newcommand{\la} {\ensuremath{\Longleftarrow}}
\newcommand{\ra} {\ensuremath{\Longrightarrow}}
\newcommand{\lra}{\ensuremath{\Longleftrightarrow}}

\newcommand{\pwf}[1]{\begin{cases} #1 \end{cases}}

% Shorthand
\newcommand{\vphi}{\varphi}
\newcommand{\veps}{\varepsilon}

\newcommand{\<}[1]{\langle #1 \rangle}

% Notation
\newcommand{\wddef}[1]{\underline{#1}}
\newcommand{\pref}[1]{(\ref{#1})}

% Maths Operators
\theoremstyle{plain}
\theoremstyle{definition}
\newtheorem{thrm}{Theorem}[section]
\newtheorem{prop}[thrm]{Proposition}
\newtheorem{corol}[thrm]{Corollary}
\newtheorem{lemma}[thrm]{Lemma}

\newtheorem{defn}[thrm]{Definition}
\newtheorem{exmp}[thrm]{Example}
\newtheorem{clame}[thrm]{Clame}

\theoremstyle{remark}
% \newtheorem{remark}[thrm]{\normalfont\large\textit Remark}
\newtheorem{remark}[thrm]{Remark}
\newtheorem{note}[thrm]{Note}


\newSimpleHeaderEnvironment{exercise}{Exercise }

\setlength{\parindent}{0cm}
\setlength{\parskip}{3pt}

\usepackage{enumerate}

\DeclareMathOperator{\holim}{holim}

% I keep adding \ in from of the P's. This makes sure I get an error.
\def\P{\undefined}

%%%%%%%% Content %%%%%%%%%%%%%%%%%%%%%%%%
\begin{document}
\mmaketitle

\begin{exercise}[1]\ 

The case $n=0$ is trivial, a connected discrete space must be a single point.

For $n\ge1$ the conclusion follows immediately (provided that there is no hidden trap, but I do not see any) from Whitehead's theorem, in the special case when all the homotopy groups are zero and the inclusion of a point is a homotopy equivalence. Indeed since the CW-complex we are considering is $n$-connected, its homotopy groups up to $n$ vanish, and since it is $n$-dimensional, its singular homology groups above $n$ also vanish (this can be seen considering cellular homology, for example), hence by Hurewicz also its homotopy groups above $n$ vanish, i.e. all homotopy groups vanish.

\end{exercise}

\begin{exercise}[2]\ 

(i)$\implies$(ii) (Note: this argument is kind of uselessly involved, one can show this without invoking the homotopy groups $\pi_n(f)$, just by working with relative homotopy groups. I think it works though, so I am keeping it anyway. Hopefully there are no huge mistakes.) Since $f$ is a weak homotopy equivalence, by the long exact sequence associated to it, we obtain that the groups $\pi_n(f)$ are trivial for every $n\ge0$, hence given a commutative square:
\[
    \begin{tikzcd}
    (S^{n-1},s_0) \arrow[d,hook] \arrow[r,"\alpha"] & (X,x_0) \arrow[d,"f"] \\
    (D^n,s_0) \arrow[r,"\beta"] & (Y,y_0)
    \end{tikzcd}
\]
there exists homotopies between $\alpha$, $\beta$ and the constant maps respectively to $x_0$ and $y_0$ that make the following square commute:
\[
    \begin{tikzcd}
    (S^{n-1},s_0)\times[0,1] \arrow[d,hook] \arrow[r,"H"] & (X,x_0) \arrow[d,"f"] \\
    (D^n,s_0)\times[0,1] \arrow[r,"\bar H"] & (Y,y_0)
    \end{tikzcd}
\]

Now, given the homotopy $H:S^{n-1}\times[0,1]\to X$, the HEP for the pair $(D^n,S^{n-1})$ gives a Homotopy $K:D^n\times[0,1]\to X$ such that $K|_{S^{n-1}\times[0,1]}=H$, hence $\lambda:=K(-,0):D^n\to X$ is a map which restricted to $\de D^n$ is equal to $\alpha$. But then the square:
\[
    \begin{tikzcd}
    (S^{n-1},s_0) \arrow[d,hook] \arrow[r,"\alpha"] & (X,x_0) \arrow[d,"f"] \\
    (D^n,s_0) \arrow[r,"{f\circ\lambda}"] & (Y,y_0)
    \end{tikzcd}
\]
is also $0$ in $\pi_n(f)$. Hence again we get a homotopy from $f\circ\lambda$ to the constant map to $y_0$. Using this latter homotopy and $\bar H$ we obtain a based homotopy $M$ between $\beta$ and $f\circ\lambda$. This homotopy is not relative to $\de D^n$, but we can obtain a homotopy that satisfies this property by composing $M:D^n\times[0,1]\to Y$ with a homotopy relative $S^{n-1}\times\cb{0}$ between $D^n\times\cb{0}$ and $D^n\times\cb{1}\cup S^{n-1}\times[0,1]$ inside $D^n\times[0,1]$.

(ii)$\implies$(iii) We can extend a map $L\to X$ cell by cell using the previous implication, to obtain the desired map $K\to X$ (it will be continuous by virtue of the final topology on the CW-complex K).

(iii)$\implies$(iv) Surjectivity. Given a map $\beta:K\to Y$, consider a $0$-cell $c$ of $K$ and the map $\alpha$ from the subcomplex $L$ of $K$ consisting of just $c$ that sends $c$ to a point of $f^{-1}(\beta(p))$. By the previous point we get a lift $\lambda:K\to X$ such that $f\circ\lambda$ is homotopic to $\beta$.

Injectivity. Consider two maps $\alpha,\alpha':K\to X$ such that $f\circ\alpha$ and $f\circ\alpha'$ are homotopic and the resulting diagram:
\[
    \begin{tikzcd}
    K\times\de[0,1] \arrow[d,hook] \arrow[r] & X \arrow[d,"f"] \\
    K\times[0,1] \arrow[r,"H"] & Y
    \end{tikzcd}
\]
where $H$ is an homotopy between $f\circ\alpha$ and $f\circ\alpha'$. By the previous point we get a lift $\bar H:K\times[0,1]\to X$ that is a homotopy between $\alpha$ and $\alpha'$.

(iv)$\implies$(i) I do not know how to prove this one.

\end{exercise}

\begin{exercise}[3]\ 

(i) This is clear from the theorem about classification of cohomology operations, because then we have that natural transformations are in bijection with a cohomology group which is zero since $m<n$ and $K(A,n)$ (being $(n-1)$-connected) can be chose to have a single $0$-cell as the $(n-1)$-skeleton.

Time out.

\end{exercise}

\end{document}
