\documentclass[a4paper,11pt,english]{article}
\usepackage{.styles/basic}
\usepackage{.styles/envs}

%%%%%%% Title %%%%%%%%%%%%%%%%%%%%%%%%%%%
\title{\textbf{Algebraic Topology} - Exercise Sheet 3}
\author{Tor Gjone (2503108) \& Michele Lorenzi (3461634)}

%%%%%%% Definitions %%%%%%%%%%%%%%%%%%%%%

% Michele's stuff
% (I hope I don't break something...)
% (Most of the command are thought for livetexing but some became habits)
% ( :D don't worry. I'm a huge fan of shortcuts myself )

\makeatletter 
% \ni was used in the definitions of other commands. This is a solution
\newcommand{\@ni}{{n-1}}

\newcommand{\pin}{\pi_n(X,x_0)}
\newcommand{\pini}{\pi_\@ni(X,x_0)}
\newcommand{\pinr}{\pi_n(X,A,x_0)}
\newcommand{\pinir}{\pi_\@ni(X,A,x_0)}
\newcommand{\hn}{H_n(X;\Z)}
\newcommand{\hni}{H_\@ni(X;\Z)}
\newcommand{\hnr}{H_n(X,A;\Z)}
\newcommand{\hnir}{H_\@ni(X,A;\Z)}
\newcommand{\dn}{D^n}
\newcommand{\dni}{D^\@ni}
\newcommand{\sn}{S^n}
\newcommand{\sni}{S^{n-1}}
\newcommand{\sph}{(D^n,\de D^n)}
\newcommand{\sphi}{(D^\@ni,\de D^\@ni)}
\newcommand{\sm}{\smallsetminus}
\newcommand{\til}{\tilde}
\newcommand{\fa}{\;\;\forall}
\newcommand{\ul}[1]{\,\underline{#1}\,}
\newcommand{\ol}{\overline}
\newcommand{\nf}{\normalfont}
\newcommand{\xs}{{x_1,\dots,x_n}}
\newcommand{\xso}{{x_0,\dots,x_n}}
\newcommand{\nn}[2]{{{#1}_1,\dots,{#1}_{#2}}}
\newcommand{\nno}[2]{{{#1}_0,\dots,{#1}_{#2}}}
\newcommand{\nns}[3]{{{#1}_1\,{#2}\,\dots\,{#2}\,{#1}_{#3}}}
\newcommand{\nnso}[3]{{{#1}_0\,{#2}\,\dots\,{#2}\,{#1}_{#3}}}
\newcommand{\cb}[1]{\{#1\}}
\newcommand{\de}{\partial}
%\newcommand{\xto}[1]{\xrightarrow{#1}}
\newcommand{\hto}{\hookrightarrow}
\newcommand{\nii}{\@ni}
\newcommand{\cont}{\reflectbox{$\in$}}
\newcommand{\mi}{{m-1}}
\newcommand{\ki}{{k-1}}
\DeclareMathOperator{\colim}{colim}
\DeclareMathOperator{\tel}{tel}
\DeclareMathOperator{\proj}{proj}


% Tor's stuff
\newcommand{\oc}[1]{\overset{\circ}{#1}}
\newcommand{\into}{\hookrightarrow}
\newcommand{\nb}{\nabla}
\renewcommand{\hat}{\widehat}

\usepackage{scalerel,stackengine}
\stackMath
\renewcommand\widehat[1]{%
\savestack{\tmpbox}{\stretchto{%
  \scaleto{%
    \scalerel*[\widthof{\ensuremath{#1}}]{\kern.1pt\mathchar"0362\kern.1pt}%
    {\rule{0ex}{\textheight}}%WIDTH-LIMITED CIRCUMFLEX
  }{\textheight}% 
}{2.4ex}}%
\stackon[-6.9pt]{#1}{\tmpbox}%
}


\newcommand{\diagSquare}[8]{
\begin{tikzpicture}[node distance=2cm, auto]
\node (a)              { $ #5 $ };
\node (b) [right of=a] { $ #6 $ };
\node (c) [below of=a] { $ #7 $ };
\node (d) [right of=c] { $ #8 $ };
\draw[-to] (a) to node { $ #1 $ } (b);
\draw[-to] (a) to node { $ #2 $ } (c);
\draw[-to] (b) to node { $ #3 $ } (d);
\draw[-to] (c) to node { $ #4 $ } (d);
\end{tikzpicture}
}

\newcommand{\congto}{\xrightarrow{\cong}}
\newcommand{\incto}[1][]{ \xhookrightarrow{#1} }
\newcommand{\xto}[1]{\xrightarrow{#1}}

% Text key words
\newcommand{\tif}{\text{if }}
\newcommand{\tand}{\text{and }}
\newcommand{\tsince}{\text{since }}


% Projective spaces
\def\RP{\mathbb{RP}}
\def\CP{\mathbb{CP}}
\def\HP{\mathbb{HP}}

% Common Categories
\DeclareMathOperator{\Top}   {\bf Top}
\DeclareMathOperator{\Ab}    {\bf Ab}
\DeclareMathOperator{\Cat}   {\bf Cat}
\DeclareMathOperator{\CAT}   {\bf CAT}
\DeclareMathOperator{\Mod}   {\bf Mod}
\DeclareMathOperator{\Ring}  {\bf Ring}
\DeclareMathOperator{\Group} {\bf Group}
\DeclareMathOperator{\sSet}  {{\bf sSet}}

% Homological Algebra
\DeclareMathOperator{\Hom}   {Hom}
\DeclareMathOperator{\Tor}   {Tor}
\DeclareMathOperator{\Ext}   {Ext}

% Common Lie Groups 
\DeclareMathOperator{\GL}    {GL}
\DeclareMathOperator{\SU}    {SU}
\DeclareMathOperator{\U}     {U}
\DeclareMathOperator{\Sp}    {Sp}

% Maths Operators
\DeclareMathOperator{\pr}    {pr}
\DeclareMathOperator{\id}    {id}
\DeclareMathOperator{\Ker}   {Ker}
\DeclareMathOperator{\im}    {Im}

% Standard sets
\def \N {\mathbb{N}}
\def \Z {\mathbb{Z}}
\def \Q {\mathbb{Q}}
\def \R {\mathbb{R}}
\def \C {\mathbb{C}}
\def \H {\mathbb{H}}

\def \E {\mathbb{E}}
\def \Z {\mathbb{Z}}
\def \I {\mathbb{I}}
\def \J {\mathbb{J}}

% Vector calculus
\newcommand{\dif}[3][]{
	\ensuremath{\frac{d^{#1} {#2}}{d {#3}^{#1}}}}
\newcommand{\pdif}[3][]{
	\ensuremath{\frac{\partial^{#1} {#2}}{\partial {#3}^{#1}}}}

% Vectors and matrices
\newcommand{\mat}[1]{\begin{matrix} #1 \end{matrix}}
\newcommand{\pmat}[1]{\begin{pmatrix} #1 \end{pmatrix}}
\newcommand{\bmat}[1]{\begin{bmatrix} #1 \end{bmatrix}}

% Add space around the argument
\newcommand{\qq}[1]{\quad#1\quad}
\newcommand{\q}[1]{\:\:#1\:\:}

% Implications
\newcommand{\la} {\ensuremath{\Longleftarrow}}
\newcommand{\ra} {\ensuremath{\Longrightarrow}}
\newcommand{\lra}{\ensuremath{\Longleftrightarrow}}

\newcommand{\pwf}[1]{\begin{cases} #1 \end{cases}}

% Shorthand
\newcommand{\vphi}{\varphi}
\newcommand{\veps}{\varepsilon}

\newcommand{\<}[1]{\langle #1 \rangle}

% Notation
\newcommand{\wddef}[1]{\underline{#1}}
\newcommand{\pref}[1]{(\ref{#1})}

% Maths Operators
\theoremstyle{plain}
\theoremstyle{definition}
\newtheorem{thrm}{Theorem}[section]
\newtheorem{prop}[thrm]{Proposition}
\newtheorem{corol}[thrm]{Corollary}
\newtheorem{lemma}[thrm]{Lemma}

\newtheorem{defn}[thrm]{Definition}
\newtheorem{exmp}[thrm]{Example}
\newtheorem{clame}[thrm]{Clame}

\theoremstyle{remark}
% \newtheorem{remark}[thrm]{\normalfont\large\textit Remark}
\newtheorem{remark}[thrm]{Remark}
\newtheorem{note}[thrm]{Note}


\newSimpleHeaderEnvironment{exercise}{Exercise }

\setlength{\parindent}{0cm}
\setlength{\parskip}{3pt}

\renewcommand{\S}[1]{\mathcal{S}(#1)}
%%%%%%%% Content %%%%%%%%%%%%%%%%%%%%%%%%
\begin{document}

\mmaketitle

\begin{exercise}[1]

\paragraph{a)}
We have 
\[ |X| := (\bigsqcup_{n\ge 0} X_n \times \nb^n ) / \sim,  \]
where $\sim$ is generated by, for all $\alpha: [n]\to[m]$, $x\in X_m$, $t \in
\nb^n$
\[ (x,\alpha_*(t)) \sim (\alpha^*(x), t). \]

So what we need to show is that the map defined on the union factors through the
quotient. We have

\begin{align*}
f_n(\alpha^*(x))(t) &= \alpha^*(f_m(x)) (t)
&& \tsince f \text{ is a morphism
in } \sSet, \\
&= f_m(x)(\alpha_*(t)) 
&&\text{by definition of } \alpha^*.
\end{align*}
So $\hat{f}: |X| \to T$ defined by $(x,t) \mapsto f_n(x)(t)$, is well-defined.

\paragraph{b)}
Let 
\[ \Phi: \Hom_{\sSet}(X, \S T) \to \Hom_{\Top} (|X|, T); \quad f \mapsto \hat f. \]
To show that $\Phi$ is a bijection, we construct an inverse. 

Let $g: |X| \to T$ and $\til g = g \circ q: \bigsqcup_{n\ge 0} X_n\times \nb^n \to
T$, where $q$ is the quotient map defined by the equivalence relation
$\sim$ above. Define $\bar g: X \to \S T$, by 
\[ \bar g_n (x)(t) := \til g(x,t),  \]
for all $x \in X_n$ and $t\in T$. We need to show that this construction is actually a morphism in $\sSet$. 

Let $\alpha: [n]\to[m]$, then for $x\in X_m$ and $t \in \nb^n$
\begin{align*}
(\bar g_n \circ \alpha^*)(x)(t) &= \til g(\alpha^*(x), t) \\
&= \til g (x, \alpha_*(t)) && \tsince \til g \text{ factors through } g, \\
&= \bar g_m (x)(\alpha_*(t)) \\
&= (\alpha^* \circ \bar g_m)(x)(t)
\end{align*}
Clearly the maps $\Phi$ and $(g \mapsto \bar g)$ are mutual inverses and thus
$\Phi$ must be a bijection.

\paragraph{c)}
We start by showing naturality in the first variable. 
Let $X,Y \in \sSet$ and $\phi : X \to Y$ be a morphism in $\sSet$. Then we want
to show that the following diagram (in which, we're suppressing the subscripts on
$\Hom$) commutes:

\begin{center}
\begin{tikzcd}[row sep=15pt]
f\circ \phi &
\Hom(X, \S T) \arrow[r, "\Phi"] & \Hom(|X|, T) \\
f \arrow[u, mapsto] & 
\Hom(Y,\S T) \arrow[r,"\Phi"] \arrow[u, "{\Hom(\phi,\S T)}"] 
& \Hom(|Y|,T) \arrow[u, swap, "{\Hom(|\phi|, T)}"] \\ 
% & f \arrow[r, mapsto] & \hat f
\end{tikzcd} 
\end{center}

Or equivalently: for all $f \in \Hom_{\sSet}(Y \to \S T)$,
\[ \hat{f \circ \phi} = \hat f \circ |\phi| : |X| \to T \]
where $|\phi|: |X|\to|Y|$ is
defined by $|\phi|(x,t) := (\phi_n(x), t)$, for $x \in X_n$, $t \in \nb^n$.

Let $f: Y \to \S T$, $x \in X_n$ and $t \in \nb^n$, then
\begin{align*}
\hat{f \circ \phi} (x,t) &= (f \circ \phi)_n(x)(t)
&& \text{ by definition of }\hat{f \circ \phi}, \\
&= (f_n \circ \phi_n)(x)(t)\\
&= f_n(\phi_n(x))(t) \\
&= \hat f(\phi_n(x), t), 
&& \text{ by definition of } \hat{f}, \\
&= (\hat f \circ |\phi|)(x, t) 
&& \text{ by definition of } |\phi|. 
\end{align*}

Naturality in the second argument is similar. Let $\psi: T \to S$ be a continuous
map. Then we want to show that the following diagram commutes:

\begin{center}
\begin{tikzcd}[row sep=15pt]
f \arrow[d, mapsto]  &
\Hom(X, \S T) \arrow[r, "\Phi"] \arrow[d, swap, "{\Hom(X, \S\psi)}"] 
& \Hom(|X|, T) \arrow[d, "{\Hom(|X|,\psi)}"] \\
\S\psi \circ f & 
\Hom(X,\S S) \arrow[r,"\Phi"]  
& \Hom(|X|,S)  \\
% & f \arrow[r, mapsto] & \hat f
\end{tikzcd} 
\end{center}

Or equivalently: for all $f\in \Hom_{\sSet}(Y, \S T)$, 
\[ \hat{{\S\psi \circ f }} = \psi \circ \hat f : |X| \to S, \]
where $\S\psi: \S X \to \S Y$ is 
defined by $\S\psi_n(\xi) := \psi \circ \xi$, for $\xi \in \S X_n$.

Let $f: X \to \S T$, $x \in X_n$ and $t \in \nb^n$, then
\begin{align*}
\hat{\S\psi \circ f}(x,t) &= (\S\psi \circ f)_n(x)(t) \\
&= (\S\psi_n \circ f_n)(x)(t) \\
&= \S\psi_n(f_n(x))(t) \\
&= (\psi \circ f_n(x))(t), 
&& \text{by definition of } \S\psi \\
&= (\psi \circ \hat f)(x)(t), 
&& \text{by definition of } \hat f.
\end{align*}

\end{exercise}

\begin{exercise}[2] (A bit tight on details...)

First we prove that $|\Delta[n]|$ is homeomorphic to $\nb^n$. The homeomorphism $h:|\Delta[n]|\to\nb^n$ is given by $(\alpha:[m]\to[n],t)\mapsto\alpha_*(t)$, which is well defined because of the adjunction of exercise 1 (since sending $\alpha$ to $\alpha_*$ determines a morphism of the simplicial sets $\Delta[n]$ and $\S{\nb^n}$), with inverse $s\mapsto[\id_{[n]},s]$. The composition $hh^{-1}$ is manifestly the identity, while the other composition is the identity because \[(\alpha,t)\mapsto \alpha_*(t)\mapsto (\id_{[n]},\alpha_*(t))=(\alpha^*\id_{[n]},t)=(\id_{[n]}\alpha,t)=(\alpha,t) \text{ in } |\Delta[n]|.\]
The topological $n$-simplex $\nb^n$ is clearly compact and Hausdorff, hence if we can show that $|\Delta[n]|$ is compact we have proved that $h$ is indeed an homeomorphism. But indeed, the space $|\Delta[n]|$ is clearly compact because it has only finitely many non-degenerate simplices (all $k$-simplices with $k>n$ get identified to some $m$-simplex with $m\leq n$ in the realization), hence it is a quotient of a compact space (in particular, of a finite disjoint union of products of compact spaces, since $\Delta[n]_m$ has finitely many elements and $\nb^m$ is compact).

Now, we know that the map $|\Delta[n]\times\Delta[1]|\to |\Delta[n]|\times|\Delta[1]|\cong \nb^n\times\nb^1$ is a map between a compact space (the same argument as above works if one wants to show compactness of the domain $|\Delta[n]\times\Delta[1]|$) and a compact Hausdorff space. Hence we are left with showing that the map is bijective. Surjectivity is clear, since it is possible to choose affine maps $\alpha_*$ and $\beta_*$ such that $\nb^n\times\nb^1\ni p=(\alpha_*(t),\beta_*(t))$ for some $t\in\nb^{n+1}$ (we can do this by subdividing $\nb^n\times\nb^1$ in $(n+1)$-simplices and then choose appropriate projections $\alpha_*:\nb^{n+1}\to\nb^n$ and $\beta_*:\nb^{n+1}\to\nb^1$), the point $((\alpha,\beta),t)$ is then sent to $p$. Injectivity is also clear, because if $((\alpha,\beta),t)$ and $(\alpha',\beta'),t')$ are mapped to the same point $p\in\nb^n\times\nb^1$, we have that $\alpha_*(t)=\alpha'_*(t')$ and $\beta_*(t)=\beta'_*(t')$, which means that
\[(\alpha_*,\beta_*)(t)=(\alpha'_*,\beta'_*)(t')\]
which means that $((\alpha,\beta),t)\sim((\alpha',\beta'),t')$ in $|\Delta[n]\times\Delta[1]|$. \bigskip

\textbf{Our initial attempt at exercise 2.} (The argument below is not complete but with some work it should be possible to salvage it, so we are keeping it.)

We'll write 
\[ \Psi = (|p_1|, |p_2|) : |\Delta[n] \times \Delta[1]| \to |\Delta[n]|\times
|\Delta[1]|. \]

To show that $\Psi$ is a homeomorphism, we will construct an inverse $\Phi$.

Every element on the left side can be represented by
\begin{equation}
\label{eq:1}
\left((\alpha, \beta), t\right)
\end{equation}
where $\alpha : [k] \to [n]$, $\beta: [k] \to [1]$ and $t \in \nb^k$.
On the other hand, an arbitrary element on the right side can be represented by
\begin{equation}
\label{eq:2}
\left((\alpha,t), (\beta, s)\right),
\end{equation}
where $\alpha: [m] \to [n]$, $\beta: [l] \to [1]$, $t \in \nb^m$ and
$s\in\nb^l$.

Then, on representatives, the map $\Psi$ is defined by
\[ ((\alpha,\beta), t) \mapsto ((\alpha,t), (\beta,t)). \]

To construct $\Phi$, we will consider some representative as in (2). 

We claim that we may wlog. assume that $m \le n$ and $l \le 1$. Or equivalently 
for all representatives as in (2), there exists $\alpha':
[m'] \to [n]$, $\beta': [l'] \to [1]$, $t'\in \nb^{m'}$ and $s'\in
\nb^{l'}$ such that $m' \le n$, $l' \le 1$ and 
\[ (\alpha,t) \sim (\alpha',t') \q{\tand} (\beta,s) \sim (\beta',s'). \]

\begin{proof}[proof of claim]
\todo{proof missing}
\end{proof}

Note that $(t,s) \in \nb^m \times \nb^l$. By the claim $l = 0,1$. If $l = 0$, then $k=m$
and $\nb^m \times \nb^0$ is already a $k$-simplex. So we will mostly consider
the case where $l=1$. (In the following arguments $l=1$, but we will still write
$l$ to keep track of it)
The geometrical idea of the constructions of $\Phi$,
consists of partitioning the polyhedra $\nb^m \times \nb^l$ into $k$-simplices 
where $k = m + l$.

Observe that the vertices of $\nb^m \times \nb^l$ can be parametrise by
$[m] \times [l]$. Define an ordering of on $[m] \times [l]$ by 
$(i,j) \ge (i',j')$ iff $i+j \ge i'+j$. 
Let $S\subset [m]\times [l]$, such that $|S| = m+l+1 = k+1$, then the convex hull
of $S$ defines a $k$-simplex $\nb_S \subseteq \nb^m \times \nb^l$. 
In particular, if $\xi: [k] \to [m]\times[l]$ is a strictly increasing map 
(wrt. the order defined above) then the convex hull of $\xi([k])$ defines a 
$k$-simplex $\nb_\xi \subseteq \nb^m \times \nb^l$.

We claim that $ \Xi_{m,l} = \{ \nb_\xi \q| \xi : [k] \to [m]\times[l] \}$ defines a
partitioning of $\nb^m \times \nb^l$ into $k$-simplices. That is 
\[ \bigsqcup_{\nb_\xi \in \Xi_{m,l}} \nb_\xi  = \nb^m \times \nb^l, \] 
and if $\xi, \xi' \in \Xi_{m,l}$ are distinct then 
\begin{equation}
\label{eq:3}
\oc{\nb}_\xi \cap \oc{\nb}_{\xi'} = \emptyset. 
\end{equation}

\begin{proof}[proof of claim]
\todo{proof missing}
\end{proof}

\begin{comment}
Now that we have a partition, consider a point $(t,s) \in
\nb^m\times\nb^l$. We claim that we may with out loss of generality 
assume that $(t,s)$ is contained in the interior of $\nb^m\times\nb^l$. That is 
if there $(t,s) \in \pd\left(\nb^m\times\nb^l\right)$ then $(\alpha, t) \sim
(\alpha', t')$ and $(\beta,s) \sim (\beta',s')$ such that 
$(t,s) \in \left(\nb^m\times\nb^l\right)^{\circ}$.

\begin{proof}[proof of claim]
\todo{proof}
\end{proof}
\end{comment}

Since $\Xi_{m,l}$ partitions $\nb^m\times\nb^l$, there exists $\xi \in
\nb^m\times\nb^l$ such that $(t,s) \in \nb_\xi$ and if $(t,s) \in \oc{\nb}_\xi$,
then $\xi$ is unique.

Define $\hat{\alpha}_\xi = \alpha \circ \pr_1 \circ \xi : [k] \to [n]$,
$\hat{\beta}_\xi = \beta \circ \pr_2 \circ \xi : [k] \to [1]$ and $\hat{t} =
(t,s) \in \nb_\xi$.
Then $((\hat{\alpha}_\xi,\hat{\beta}_\xi), \hat{t})$ defines a representative
in $|\nb[n]\times\nb[1]|$. 
We define $\Phi$ by 
\[ ((\alpha,s), (\beta,t)) \mapsto ((\hat{\alpha}_\xi,\hat{\beta}_\xi),
\hat{t}). \]

We still need to check that this definition is independent of the $\xi$, 
well-defined and continues. And we need to check that $\Phi$ actually defines an
inverse of $\Psi$.

Suppose $\hat{t} = (t,s) \in \nb_\xi \cap \nb_{\xi'}$. Then, by \pref{eq:3},
$\hat t$ most be contained in a face of both $\nb_\xi$ and $\nb_{\xi'}$.

\end{exercise}

\begin{exercise}[3]

\paragraph{a)} We show that the formulas given for the face and degeneracy maps satisfy the usual simplicial identities that face and degeneracy maps of a simplicial set must satisfy: if this is true, then there is a unique simplicial set extending the given data.

The simplicial identities are:
\[d_id_j=d_{j-1}d_i\quad i<j\]
\[s_is_j=s_{j+1}s_i\quad i\leq j\] 
\[d_is_j=\begin{cases}
1 & i=j,j+1 \\
s_{j-1}d_i & i<j \\
s_jd_{i-1} & i>j+1
\end{cases}\]

It becomes clear that the face and degeneracy maps given satisfy these identities once we write elements of $NI$ as strings:
\[c_1\xto{\,\ f_1\ }c_2\xto{\,\ f_2\ }\dots\xto{f_{n-1}}c_{n}\xto{\,\ f_{n}\ }c_{n+1}\]
since then the face map correspond to "removing" an object $c_i$ and composing the arrows to and from it and the degeneracy map consists in adding a copy of a $c_i$ and the corresponding identity map. Then the first relation is satisfied because for $i<j$ first removing the object $c_j$ and then removing the object $c_i$ is the same as removing the object $c_i$ and then the object that now is $c_{j-1}$, and similar arguments work for the remaining cases.

\paragraph{b)} We need to show commutativity of the diagram:
\begin{center}
    \begin{tikzcd}[column sep=large]
    (NI)_n \arrow[r, "(NF)_n"] & (NJ)_n \\
    (NI)_m \arrow[u, "\alpha^*"] \arrow[r, "(NF)_m"] & (NJ)_m \arrow[u, "\alpha^*" right]
    \end{tikzcd}
\end{center}
and it is enough to do so for the face and degeneracy maps, since the case for a general map $\alpha^*$ will then follow from its canonical factorization. In both cases this is clear, because $F(f_{i+1}\circ f_i)=F(f_{i+1})\circ F(f_i)$ and $F(\id_{c_i})=\id_{F(c_i)}$.

\paragraph{c)} This amounts to showing that the morphisms $(N\!\proj_I,N\!\proj_J)_n$ are isomorphisms, but this becomes clear from the definition of the nerve functor once one writes down what the morphisms are:
\begin{align*}
(N\!\proj_I,N\!\proj_J)_n:N(I\times J)_n&\to NI_n\times NJ_n=(NI\times NJ)_n \\
(f_n\times g_n,\dots,f_1\times g_1)&\mapsto (f_n,\dots,f_1)\times(g_n,\dots,g_1)
\end{align*}
indeed, any map $f_n\times g_n$ is uniquely determined by $f_n$ and $g_n$ (by the universal property of products), thus making the morphisms bijections.

\paragraph{d)} We define the functor $H:I\times[1]\to J$ as follows:
\[H(i,0)=Fi,\ H(i,1)=Gi\]
on objects and:
\[H(h,\id_0)=Fh,\ H(h,\id_1)=Gh,\ H(h,f)=\tau_{i'}\circ Fh=Gh\circ \tau_i\]
on arrows. In particular, if we call $i$ the arrow $\id_i$ we have that $H(i,f)=\tau_i$.

Now, using point c) and the functor $NH:N(I\times[1])\to NJ$, we obtain a morphism of simplicial sets 
\[NH'=(NH)\,(N\!\proj_I,N\!\proj_J)^{-1}:NI\times N[1]\to NJ\]

This will determine a simplicial homotopy $NH''$ between $NF$ and $NG$, since the two simplicial sets $N[1]$ and $\Delta[1]$ are isomorphic with the isomorphism $K$ between them being the morphism whose components $N[1]_n\to\Delta[1]_n$ send a string of type \[0 \to \dots \to 0 \xto{f} 1 \to \dots \to 1\]
to the obvious corresponding non-decreasing function $[n]\to[1]$. Explicitly, $NH''$ will then be $(NH')(\id_{NI}\times K^{-1})$.

In particular, we have that $NH''$ is an homotopy because with these definitions $NH''\circ i_0$ is clearly equal to $NF$ and $NH''\circ i_1$ to $NG$.

\paragraph{e)} Two small categories $I,J$ are equivalent if there exists two functors $F:I\to J$ and $G:J\to I$ such that $FG\cong\id_J$ and $GF\cong\id_I$. In particular, by point d) the natural transformations $FG\Rightarrow\id_J$ and $GF\Rightarrow\id_I$ determine simplicial homotopies between $N(FG)=(NF)(NG)$ and $N\id_J=\id_{NJ}$ and between $(NG)(NF)$ and $\id_{NJ}$, i.e. the morphisms $NF$ and $NG$ are simplicial homotopy equivalences between $NI$ and $NJ$.

\end{exercise}

\end{document}
