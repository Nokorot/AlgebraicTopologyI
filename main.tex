\documentclass[a4paper,11pt,english]{article}
\usepackage{.styles/basic}
\usepackage{.styles/envs}

\usepackage{circuitikz}

\usepackage{caption}
\usepackage{subcaption}

%%%%%%% Title %%%%%%%%%%%%%%%%%%%%%%%%%%%
\title{\textbf{Algebraic Topology} - Exercise Sheet 6}
\author{Tor Gjone (2503108) \& Michele Lorenzi (3461634)}

%%%%%%% Definitions %%%%%%%%%%%%%%%%%%%%%

% It's here because it's messing up my syntax highlighting 
%% NOTE: The align* environment doesn't work for some reason.
% this is a temporary fix
\renewenvironment{align*}{
\[ \arraycolsep=2pt \def\arraystretch{1.5}
\begin{array}{rl}
}{ \end{array} \] }


% Michele's stuff
% (I hope I don't break something...)
% (Most of the command are thought for livetexing but some became habits)
% ( :D don't worry. I'm a huge fan of shortcuts myself )

\makeatletter 
% \ni was used in the definitions of other commands. This is a solution
\newcommand{\@ni}{{n-1}}

\newcommand{\nmi}{{n-1}}
\newcommand{\npi}{{n+1}}
\newcommand{\nii}{\@ni} % \ni is an inverted inclusion

\newcommand{\mi}{{m-1}}
\newcommand{\ki}{{k-1}}

\newcommand{\smsh}{\wedge}
\newcommand{\pin}{\pi_n(X,x_0)}
\newcommand{\pini}{\pi_\@ni(X,x_0)}
\newcommand{\pinr}{\pi_n(X,A,x_0)}
\newcommand{\pinir}{\pi_\@ni(X,A,x_0)}
\newcommand{\hn}{H_n(X;\Z)}
\newcommand{\hni}{H_\@ni(X;\Z)}
\newcommand{\hnr}{H_n(X,A;\Z)}
\newcommand{\hnir}{H_\@ni(X,A;\Z)}
\newcommand{\dn}{D^n}
\newcommand{\dni}{D^\@ni}
\newcommand{\sn}{S^n}
\newcommand{\sni}{S^{n-1}}
\newcommand{\sph}{(D^n,\de D^n)}
\newcommand{\sphi}{(D^\@ni,\de D^\@ni)}
\newcommand{\sm}{\smallsetminus}
\newcommand{\til}{\tilde}
\newcommand{\fa}{\;\;\forall}
\newcommand{\ul}[1]{\,\underline{#1}\,}
\newcommand{\ol}{\overline}
\newcommand{\nf}{\normalfont}
\newcommand{\xs}{{x_1,\dots,x_n}}
\newcommand{\xso}{{x_0,\dots,x_n}}
\newcommand{\nn}[2]{{{#1}_1,\dots,{#1}_{#2}}}
\newcommand{\nno}[2]{{{#1}_0,\dots,{#1}_{#2}}}
\newcommand{\nns}[3]{{{#1}_1\,{#2}\,\dots\,{#2}\,{#1}_{#3}}}
\newcommand{\nnso}[3]{{{#1}_0\,{#2}\,\dots\,{#2}\,{#1}_{#3}}}
\newcommand{\cb}[1]{\{#1\}}
\newcommand{\de}{\partial}
%\newcommand{\xto}[1]{\xrightarrow{#1}}
\newcommand{\hto}{\hookrightarrow}
\newcommand{\cont}{\reflectbox{$\in$}}
\DeclareMathOperator{\Colim}{colim}
\DeclareMathOperator{\tel}{tel}
\DeclareMathOperator{\proj}{proj}
\DeclareMathOperator{\ev}{ev}


% Tor's stuff
\newcommand{\orth}{\bot}
\newcommand{\se}{\subseteq}
\renewcommand{\ss}{\subset}
\renewcommand{\i}{^{-1}}
\newcommand{\oc}[1]{\overset{\circ}{#1}}
\newcommand{\into}{\hookrightarrow}
\newcommand{\onto}{\twoheadrightarrow}
\newcommand{\nb}{\nabla}
\renewcommand{\hat}{\widehat}
\newcommand{\wtil}{\widetilde}

\usepackage{scalerel,stackengine}
\stackMath
\renewcommand\widehat[1]{%
\savestack{\tmpbox}{\stretchto{%
  \scaleto{%
    \scalerel*[\widthof{\ensuremath{#1}}]{\kern.1pt\mathchar"0362\kern.1pt}%
    {\rule{0ex}{\textheight}}%WIDTH-LIMITED CIRCUMFLEX
  }{\textheight}% 
}{2.4ex}}%
\stackon[-6.9pt]{#1}{\tmpbox}%
}

\makeatletter
\newsavebox{\@Wedge}
\sbox\@Wedge{%
% $\mathlarger{ \mathlarger{\mathlarger{ \mathlarger{ \wedge }}}}$ }
\scalebox{1.5}{\raisebox{-1pt}{$\wedge$}} }
\newcommand{\Wedge}{\usebox{\@Wedge}}
\makeatother


% Direct limit and colimit
\makeatletter
\newcommand{\lim@}[2]{%
  \vtop{\m@th\ialign{##\cr
    \hfil$#1\operator@font lim$\hfil\cr
    \noalign{\nointerlineskip\kern1.5\ex@}#2\cr
    \noalign{\nointerlineskip\kern-\ex@}\cr}}%
}
\newcommand{\colim}{%
  \mathop{\mathpalette\varlim@{\rightarrowfill@\scriptscriptstyle}}\nmlimits@
}
\newcommand{\invlim}{%
  \mathop{\mathpalette\varlim@{\leftarrowfill@\scriptscriptstyle}}\nmlimits@
}
\makeatother


\newcommand{\diagSquare}[8]{
\begin{tikzpicture}[node distance=2cm, auto]
\node (a)              { $ #5 $ };
\node (b) [right of=a] { $ #6 $ };
\node (c) [below of=a] { $ #7 $ };
\node (d) [right of=c] { $ #8 $ };
\draw[-to] (a) to node { $ #1 $ } (b);
\draw[-to] (a) to node { $ #2 $ } (c);
\draw[-to] (b) to node { $ #3 $ } (d);
\draw[-to] (c) to node { $ #4 $ } (d);
\end{tikzpicture}
}

\newcommand{\simto}{\xrightarrow{\sim}}
\newcommand{\incto}[1][]{ \xhookrightarrow{#1} }
\newcommand{\xto}[1]{\xrightarrow{#1}}

% Text key words
\newcommand{\tif}{\text{if }}
\newcommand{\tand}{\text{and }}
\newcommand{\tsince}{\text{since }}


% Projective spaces
\def\RP{\mathbb{RP}}
\def\CP{\mathbb{CP}}
\def\HP{\mathbb{HP}}

% Common Categories
\DeclareMathOperator{\Top}   {\bf Top}
\DeclareMathOperator{\Ab}    {\bf Ab}
\DeclareMathOperator{\Cat}   {\bf Cat}
\DeclareMathOperator{\CAT}   {\bf CAT}
\DeclareMathOperator{\Mod}   {\bf Mod}
\DeclareMathOperator{\Ring}  {\bf Ring}
\DeclareMathOperator{\Group} {\bf Group}
\DeclareMathOperator{\sSet}  {{\bf sSet}}

% Homological Algebra
\DeclareMathOperator{\Hom}   {Hom}
\DeclareMathOperator{\Tor}   {Tor}
\DeclareMathOperator{\Ext}   {Ext}

% Common Lie Groups 
\let \O \undefined

\DeclareMathOperator{\GL}    {GL}
\DeclareMathOperator{\SU}    {SU}
\DeclareMathOperator{\U}     {U}
\DeclareMathOperator{\O}     {O}
\DeclareMathOperator{\SO}    {SO}
\DeclareMathOperator{\Sp}    {Sp}
\DeclareMathOperator{\Gr}    {Gr}

% Maths Operators
\DeclareMathOperator{\pr}    {pr}
\DeclareMathOperator{\id}    {id}
\DeclareMathOperator{\Ker}   {Ker}
\DeclareMathOperator{\im}    {Im}

% \let \span \undefined
\DeclareMathOperator{\tspan}  {span}

% Standard sets
\def \N {\mathbb{N}}
\def \Z {\mathbb{Z}}
\def \Q {\mathbb{Q}}
\def \R {\mathbb{R}}
\def \C {\mathbb{C}}
\def \H {\mathbb{H}}

\def \E {\mathbb{E}}
\def \Z {\mathbb{Z}}
\def \I {\mathbb{I}}
\def \J {\mathbb{J}}

% Vector calculus
\newcommand{\dif}[3][]{
	\ensuremath{\frac{d^{#1} {#2}}{d {#3}^{#1}}}}
\newcommand{\pdif}[3][]{
	\ensuremath{\frac{\partial^{#1} {#2}}{\partial {#3}^{#1}}}}

% Vectors and matrices
\newcommand{\mat}[1]{\begin{matrix} #1 \end{matrix}}
\newcommand{\pmat}[1]{\begin{pmatrix} #1 \end{pmatrix}}
\newcommand{\bmat}[1]{\begin{bmatrix} #1 \end{bmatrix}}

% Add space around the argument
\newcommand{\qq}[1]{\quad#1\quad}
\newcommand{\q}[1]{\:\:#1\:\:}

% Implications
\newcommand{\la} {\ensuremath{\Longleftarrow}}
\newcommand{\ra} {\ensuremath{\Longrightarrow}}
\newcommand{\lra}{\ensuremath{\Longleftrightarrow}}

\newcommand{\pwf}[1]{\begin{cases} #1 \end{cases}}

% Shorthand
\newcommand{\vphi}{\varphi}
\newcommand{\veps}{\varepsilon}
\newcommand{\eps}{\epsilon}

\newcommand{\<}[1]{\langle #1 \rangle}

% Notation
\newcommand{\wddef}[1]{\underline{#1}}
\newcommand{\pref}[1]{(\ref{#1})}

% Maths Operators
\theoremstyle{plain}
\theoremstyle{definition}
\newtheorem{thrm}{Theorem}[section]
\newtheorem{prop}[thrm]{Proposition}
\newtheorem{corol}[thrm]{Corollary}
\newtheorem{lemma}[thrm]{Lemma}

\newtheorem{defn}[thrm]{Definition}
\newtheorem{exmp}[thrm]{Example}
\newtheorem{clame}[thrm]{Clame}

\theoremstyle{remark}
% \newtheorem{remark}[thrm]{\normalfont\large\textit Remark}
\newtheorem{remark}[thrm]{Remark}
\newtheorem{note}[thrm]{Note}


\newSimpleHeaderEnvironment{exercise}{Exercise }

\setlength{\parindent}{0cm}
\setlength{\parskip}{3pt}

\newcommand{\orth}{\bot}

\renewcommand{\S}[1]{\mathcal{S}(#1)}
%%%%%%%% Content %%%%%%%%%%%%%%%%%%%%%%%%
\begin{document}

\mmaketitle

\begin{exercise}[1]
In the case $k = 0$, $p = \id^2: S^n \to S^n$ defines a fibre bundle, with fibre
$S^0$, for all $n$. So $0 = k \ne n-1$ and $m \ne 2n -1$ in general. 
So we suppose that also $k \ge 1$.

% Suppose $k = 0$, then $m=n$ so $p$ is a self map on $S^n$. With out loss of
% generality we may suppose $p$ fixes some base point $x_0$, so $p$ defines some
% element in $\pi_n(S^n,x_0)$. 

It is clear that $m = n+k$. Since $p: S^m \to S^n$ is a fibre bundel we have the
following long exact sequence
\[ \dots \to \pi_i(S^k,b) \to \pi_i(S^m,b) \to \pi_i(S^n,e) \to \pi_{i-1}(S^k, b) \to
\dots \]
where $e \in S^n$ and $b \in p\i(e)$. 
Considering the sequence with $i \sim k$, we have
\[ \dots \to \pi_{k+1}(S^n,e) \q\to \pi_{k}(S^k, b) = \Z \q\to \pi_k(S^m,b) = 0 \to \dots \]
since $m = n+k \ge k+1$. So we may conclude that $\pi_{k+1}(S^n,e)\ne 0$, and
thus $n \le k+1$.

Considering the sequence with $i \sim n$, we have
\[ \dots \to \pi_n(S^m,b) = 0 \q\to \pi_n(S^n,e) = \Z \q\to \pi_{n-1}(S^k,b)
\q\to \pi_{n-1}(S^m,b) = 0 \to \dots \]
since $m = n+k \ge n+1$. So $\pi_{n-1}(S^k,b) = \Z \ne 0$ and we may conclude
that $n-1 \ge k$. Hence $k = n-1$ and thus $m = k+n = 2n-1$.




\end{exercise}


\begin{exercise}[3]
Let $J = I_{0,1} \sqcup_1 I_{1,2} \subset \nb^2$, be the subset consisting of
the faces $I_{0,1}$ and $I_{1,2}$ (see figure \pref{fig:1a}).

Let $H: \nb^2 \times [0,1] \to \nb^2$ be the homotopy illustrated in figure
\pref{fig:1b}, such that $H(-,0) = \id_{\nb^2}$, $\im(H(-,1)) = J$ and
$H|_J(-,t) = \id_J$, for all $t$. Let $p = H(-,1)$ and $\psi, \phi \in \Omega
X$. Considering $\phi$ as a map from $I_{0,1}$, (going from $1$ to $0$) and
$\psi$ from $I_{1,2}$ (going from $2$ to $1$), $\phi \sqcup_1 \psi$ defines a
map from $J$ to $X$, mapping all the three vertices to $x_0$. Define 
\[ \Psi : \Omega X \times \Omega X \to E \; \quad (\phi, \psi) \mapsto
(\phi\sqcup_1 \psi) \circ p. \]

\begin{figure}[h]
\centering
\caption{}
\begin{subfigure}[b]{0.45\textwidth}
\centering
\begin{tikzpicture}
\node[above] (c) at (1,1.5) {$2$};
\node[left]  (a) at (0,0) {$0$};
\node[right] (b) at (2,0) {$1$};
\draw[line width=1.5px] (0,0) 
    -- (2,0)    node[sloped, scale=.9, pos=0.5, xscale=-1] {$>$} 
                node[scale=1, pos=0.5, below] {$I_{0,1}$} 
    -- (1,1.5)  node[sloped, scale=.9, pos=0.5] {$>$} 
                node[scale=1, pos=0.5, above right] {$I_{1,2}$};

\draw (0,0) -- (1,1.5);

\end{tikzpicture}

\caption{The think line is $J$.}
\label{fig:1a}
\end{subfigure}
\begin{subfigure}[b]{0.45\textwidth}
\centering
\begin{tikzpicture}
\node[above] (c) at (1,1.5) {$2$};
\node[left]  (a) at (0,0) {$0$};
\node[right] (b) at (2,0) {$1$};

%  for t \in [0,1] : (t,1.5)  --  (4*t ,0) if t < .5 else (3-2*t, 3*(t-.5))  
\draw[cyan, densely dotted] (0.143, 0.214) -- (0.571, 0.000);
\draw[cyan, densely dotted] (0.286, 0.429) -- (1.143, 0.000);
\draw[cyan, densely dotted] (0.429, 0.643) -- (1.714, 0.000);
\draw[cyan, densely dotted] (0.571, 0.857) -- (1.857, 0.214);
\draw[cyan, densely dotted] (0.714, 1.071) -- (1.571, 0.643);
\draw[cyan, densely dotted] (0.857, 1.286) -- (1.286, 1.071);

% a=.488, t=.2,.4,.6,.8: (2-(cos(a)*2)*t*cos(a), (cos(a)*2)*t*sin(a))
\draw [cyan] plot [smooth] coordinates { (0,0) (1.687, 0.165) (1,1.5)};
\draw [cyan] plot [smooth] coordinates { (0,0) (1.375, 0.331) (1,1.5)};
\draw [cyan] plot [smooth] coordinates { (0,0) (1.063, 0.496) (1,1.5)};
\draw [cyan] plot [smooth] coordinates { (0,0) (0.751, 0.662) (1,1.5)};

% t=.3 to t=.78
\draw[cyan, -to] (0.783, 0.647) -- (1.532, 0.248);

\draw[cyan, dotted] (0.783, 0.647) -- (1.532, 0.248);

\draw[line width=1px] (0,0) 
    -- (2,0)    node[sloped, scale=.9, pos=0.5, xscale=-1] {$>$} 
%                node[scale=1, pos=0.5, below] {$I_{0,1}$} 
    -- (1,1.5)  node[sloped, scale=.9, pos=0.5] {$>$};
%                node[scale=1, pos=0.5, above right] {$I_{1,2}$};

\draw (0,0) -- (1,1.5);

\end{tikzpicture}

\caption{Deformation retract along the dotted lines.}
\label{fig:1b}
\end{subfigure}
\end{figure}

It is quite clear that $\Phi \circ \Psi = \id_{\Omega X \times \Omega X}$. We
have $p \circ i = i$ and $p \circ j = j$, where we consider $i,j$ to be maps
into $J$, so 
\[  (\Phi \circ \Psi)(\phi, \psi) = ( (\phi \sqcup_1 \psi) \circ i, (\phi
\sqcup_1 \psi) \circ j ) = (\phi, \psi). \]

Conversely, $\til H : \id_E \cong \Phi \circ \Psi$, where $\til H$ is defined
by pre composition with $H$.
\end{exercise}

\end{document}
