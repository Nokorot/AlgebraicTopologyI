\documentclass[a4paper,11pt,english]{article}
\usepackage{.styles/basic}
\usepackage{.styles/envs}

%%%%%%% Title %%%%%%%%%%%%%%%%%%%%%%%%%%%
\title{\textbf{Algebraic Topology} - Exercise Sheet 3}
\author{Tor Gjone (2503108) \& Michele Lorenzi (3461634)}

%%%%%%% Definitions %%%%%%%%%%%%%%%%%%%%%

% Michele's stuff
% (I hope I don't break something...)
% (Most of the command are thought for livetexing but some became habits)
% ( :D don't worry. I'm a huge fan of shortcuts myself )

\makeatletter 
% \ni was used in the definitions of other commands. This is a solution
\newcommand{\@ni}{{n-1}}

\newcommand{\pin}{\pi_n(X,x_0)}
\newcommand{\pini}{\pi_\@ni(X,x_0)}
\newcommand{\pinr}{\pi_n(X,A,x_0)}
\newcommand{\pinir}{\pi_\@ni(X,A,x_0)}
\newcommand{\hn}{H_n(X;\Z)}
\newcommand{\hni}{H_\@ni(X;\Z)}
\newcommand{\hnr}{H_n(X,A;\Z)}
\newcommand{\hnir}{H_\@ni(X,A;\Z)}
\newcommand{\dn}{D^n}
\newcommand{\dni}{D^\@ni}
\newcommand{\sn}{S^n}
\newcommand{\sni}{S^{n-1}}
\newcommand{\sph}{(D^n,\de D^n)}
\newcommand{\sphi}{(D^\@ni,\de D^\@ni)}
\newcommand{\sm}{\smallsetminus}
\newcommand{\til}{\tilde}
\newcommand{\fa}{\;\;\forall}
\newcommand{\ul}[1]{\,\underline{#1}\,}
\newcommand{\ol}{\overline}
\newcommand{\nf}{\normalfont}
\newcommand{\xs}{{x_1,\dots,x_n}}
\newcommand{\xso}{{x_0,\dots,x_n}}
\newcommand{\nn}[2]{{{#1}_1,\dots,{#1}_{#2}}}
\newcommand{\nno}[2]{{{#1}_0,\dots,{#1}_{#2}}}
\newcommand{\nns}[3]{{{#1}_1\,{#2}\,\dots\,{#2}\,{#1}_{#3}}}
\newcommand{\nnso}[3]{{{#1}_0\,{#2}\,\dots\,{#2}\,{#1}_{#3}}}
\newcommand{\cb}[1]{\{#1\}}
\newcommand{\de}{\partial}
%\newcommand{\xto}[1]{\xrightarrow{#1}}
\newcommand{\hto}{\hookrightarrow}
\newcommand{\nii}{\@ni}
\newcommand{\cont}{\reflectbox{$\in$}}
\newcommand{\mi}{{m-1}}
\newcommand{\ki}{{k-1}}
\DeclareMathOperator{\colim}{colim}
\DeclareMathOperator{\tel}{tel}
\DeclareMathOperator{\proj}{proj}


% Tor's stuff
\newcommand{\oc}[1]{\overset{\circ}{#1}}
\newcommand{\into}{\hookrightarrow}
\newcommand{\nb}{\nabla}
\renewcommand{\hat}{\widehat}

\usepackage{scalerel,stackengine}
\stackMath
\renewcommand\widehat[1]{%
\savestack{\tmpbox}{\stretchto{%
  \scaleto{%
    \scalerel*[\widthof{\ensuremath{#1}}]{\kern.1pt\mathchar"0362\kern.1pt}%
    {\rule{0ex}{\textheight}}%WIDTH-LIMITED CIRCUMFLEX
  }{\textheight}% 
}{2.4ex}}%
\stackon[-6.9pt]{#1}{\tmpbox}%
}


\newcommand{\diagSquare}[8]{
\begin{tikzpicture}[node distance=2cm, auto]
\node (a)              { $ #5 $ };
\node (b) [right of=a] { $ #6 $ };
\node (c) [below of=a] { $ #7 $ };
\node (d) [right of=c] { $ #8 $ };
\draw[-to] (a) to node { $ #1 $ } (b);
\draw[-to] (a) to node { $ #2 $ } (c);
\draw[-to] (b) to node { $ #3 $ } (d);
\draw[-to] (c) to node { $ #4 $ } (d);
\end{tikzpicture}
}

\newcommand{\congto}{\xrightarrow{\cong}}
\newcommand{\incto}[1][]{ \xhookrightarrow{#1} }
\newcommand{\xto}[1]{\xrightarrow{#1}}

% Text key words
\newcommand{\tif}{\text{if }}
\newcommand{\tand}{\text{and }}
\newcommand{\tsince}{\text{since }}


% Projective spaces
\def\RP{\mathbb{RP}}
\def\CP{\mathbb{CP}}
\def\HP{\mathbb{HP}}

% Common Categories
\DeclareMathOperator{\Top}   {\bf Top}
\DeclareMathOperator{\Ab}    {\bf Ab}
\DeclareMathOperator{\Cat}   {\bf Cat}
\DeclareMathOperator{\CAT}   {\bf CAT}
\DeclareMathOperator{\Mod}   {\bf Mod}
\DeclareMathOperator{\Ring}  {\bf Ring}
\DeclareMathOperator{\Group} {\bf Group}
\DeclareMathOperator{\sSet}  {{\bf sSet}}

% Homological Algebra
\DeclareMathOperator{\Hom}   {Hom}
\DeclareMathOperator{\Tor}   {Tor}
\DeclareMathOperator{\Ext}   {Ext}

% Common Lie Groups 
\DeclareMathOperator{\GL}    {GL}
\DeclareMathOperator{\SU}    {SU}
\DeclareMathOperator{\U}     {U}
\DeclareMathOperator{\Sp}    {Sp}

% Maths Operators
\DeclareMathOperator{\pr}    {pr}
\DeclareMathOperator{\id}    {id}
\DeclareMathOperator{\Ker}   {Ker}
\DeclareMathOperator{\im}    {Im}

% Standard sets
\def \N {\mathbb{N}}
\def \Z {\mathbb{Z}}
\def \Q {\mathbb{Q}}
\def \R {\mathbb{R}}
\def \C {\mathbb{C}}
\def \H {\mathbb{H}}

\def \E {\mathbb{E}}
\def \Z {\mathbb{Z}}
\def \I {\mathbb{I}}
\def \J {\mathbb{J}}

% Vector calculus
\newcommand{\dif}[3][]{
	\ensuremath{\frac{d^{#1} {#2}}{d {#3}^{#1}}}}
\newcommand{\pdif}[3][]{
	\ensuremath{\frac{\partial^{#1} {#2}}{\partial {#3}^{#1}}}}

% Vectors and matrices
\newcommand{\mat}[1]{\begin{matrix} #1 \end{matrix}}
\newcommand{\pmat}[1]{\begin{pmatrix} #1 \end{pmatrix}}
\newcommand{\bmat}[1]{\begin{bmatrix} #1 \end{bmatrix}}

% Add space around the argument
\newcommand{\qq}[1]{\quad#1\quad}
\newcommand{\q}[1]{\:\:#1\:\:}

% Implications
\newcommand{\la} {\ensuremath{\Longleftarrow}}
\newcommand{\ra} {\ensuremath{\Longrightarrow}}
\newcommand{\lra}{\ensuremath{\Longleftrightarrow}}

\newcommand{\pwf}[1]{\begin{cases} #1 \end{cases}}

% Shorthand
\newcommand{\vphi}{\varphi}
\newcommand{\veps}{\varepsilon}

\newcommand{\<}[1]{\langle #1 \rangle}

% Notation
\newcommand{\wddef}[1]{\underline{#1}}
\newcommand{\pref}[1]{(\ref{#1})}

% Maths Operators
\theoremstyle{plain}
\theoremstyle{definition}
\newtheorem{thrm}{Theorem}[section]
\newtheorem{prop}[thrm]{Proposition}
\newtheorem{corol}[thrm]{Corollary}
\newtheorem{lemma}[thrm]{Lemma}

\newtheorem{defn}[thrm]{Definition}
\newtheorem{exmp}[thrm]{Example}
\newtheorem{clame}[thrm]{Clame}

\theoremstyle{remark}
% \newtheorem{remark}[thrm]{\normalfont\large\textit Remark}
\newtheorem{remark}[thrm]{Remark}
\newtheorem{note}[thrm]{Note}


\newSimpleHeaderEnvironment{exercise}{Exercise }

\setlength{\parindent}{0cm}
\setlength{\parskip}{3pt}

\renewcommand{\S}[1]{\mathcal{S}(#1)}
%%%%%%%% Content %%%%%%%%%%%%%%%%%%%%%%%%
\begin{document}

\mmaketitle


\begin{exercise}[1]

\begin{enumerate}

\item[(a)]
We have 
\[ |X| := (\bigsqcup_{n\ge 0} X_n \times \nb^n ) / \sim,  \]
where $\sim$ is generated by, for all $\alpha: [n]\to[m]$, $x\in X_m$, $t \in
\nb^n$
\[ (x,\alpha_*(t)) \sim (\alpha^*(x), t). \]

So what we need to show is that the map defined on the union factors through the
quotient. We have

\begin{align*}
f_n(\alpha^*(x))(t) &= \alpha^*(f_m(x)) (t)
&& \tsince f \text{ is a morphism
in } \sSet, \\
&= f_m(x)(\alpha_*(t)) 
&&\text{by definition of } \alpha^*.
\end{align*}
So $\hat{f}: |X| \to T$ defined by $(x,t) \mapsto f_n(x)(t)$, is well-defined.

\item[(b)]
Let 
\[ \Phi: \Hom_{\sSet}(X, \S T) \to \Hom_{\Top} (|X|, T); \quad f \mapsto \hat f. \]
To show that $\Phi$ is a bijection, we construct an inverse. 

Let $g: |X| \to T$ and $\til g = g \circ q: \bigsqcup_{n\ge 0} X_n\times \nb^n \to
T$, where $q$ is the quotient map defined by the equivalence relation
$\sim$ above. Define $\bar g: X \to \S T$, by 
\[ \bar g_n (x)(t) := \til g(x,t),  \]
for all $x \in X_n$ and $t\in T$. We need to show that this construction is actually a morphism in $\sSet$. 

Let $\alpha: [n]\to[m]$, then for $x\in X_m$ and $t \in \nb^n$
\begin{align*}
(\bar g_n \circ \alpha^*)(x)(t) &= \til g(\alpha^*(x), t) \\
&= \til g (x, \alpha_*(t)) && \tsince \til g \text{ factors through } g, \\
&= \bar g_m (x)(\alpha_*(t)) \\
&= (\alpha^* \circ \bar g_m)(x)(t)
\end{align*}
Clearly the maps $\Phi$ and $(g \mapsto \bar g)$ are mutual inverses and thus
$\Phi$ must be a bijection.

\item[(c)]
We start by showing naturality in the first variable. 
Let $X,Y \in \sSet$ and $\phi : X \to Y$ be a morphism in $\sSet$. Then we want
to show that the following diagram (in which, we're suppressing the subscripts on
$\Hom$) commutes:

\begin{center}
\begin{tikzcd}[row sep=15pt]
f\circ \phi &
\Hom(X, \S T) \arrow[r, "\Phi"] & \Hom(|X|, T) \\
f \arrow[u, mapsto] & 
\Hom(Y,\S T) \arrow[r,"\Phi"] \arrow[u, "{\Hom(\phi,\S T)}"] 
& \Hom(|Y|,T) \arrow[u, swap, "{\Hom(|\phi|, T)}"] \\ 
% & f \arrow[r, mapsto] & \hat f
\end{tikzcd} 
\end{center}

Or equivalently: for all $f \in \Hom_{\sSet}(Y \to \S T)$,
\[ \hat{f \circ \phi} = \hat f \circ |\phi| : |X| \to T \]
where $|\phi|: |X|\to|Y|$ is
defined by $|\phi|(x,t) := (\phi_n(x), t)$, for $x \in X_n$, $t \in \nb^n$.

Let $f: Y \to \S T$, $x \in X_n$ and $t \in \nb^n$, then
\begin{align*}
\hat{f \circ \phi} (x,t) &= (f \circ \phi)_n(x)(t)
&& \text{ by definition of }\hat{f \circ \phi}, \\
&= (f_n \circ \phi_n)(x)(t)\\
&= f_n(\phi_n(x))(t) \\
&= \hat f(\phi_n(x), t), 
&& \text{ by definition of } \hat{f}, \\
&= (\hat f \circ |\phi|)(x, t) 
&& \text{ by definition of } |\phi|. 
\end{align*}

Naturality in the second argument is similar. Let $\psi: T \to S$ be a continuous
map. Then we want to show that the following diagram commutes:

\begin{center}
\begin{tikzcd}[row sep=15pt]
f \arrow[d, mapsto]  &
\Hom(X, \S T) \arrow[r, "\Phi"] \arrow[d, swap, "{\Hom(X, \S\psi)}"] 
& \Hom(|X|, T) \arrow[d, "{\Hom(|X|,\psi)}"] \\
\S\psi \circ f & 
\Hom(X,\S S) \arrow[r,"\Phi"]  
& \Hom(|X|,S)  \\
% & f \arrow[r, mapsto] & \hat f
\end{tikzcd} 
\end{center}

Or equivalently: for all $f\in \Hom_{\sSet}(Y, \S T)$, 
\[ \hat{{\S\psi \circ f }} = \psi \circ \hat f : |X| \to S, \]
where $\S\psi: \S X \to \S Y$ is 
defined by $\S\psi_n(\xi) := \psi \circ \xi$, for $\xi \in \S X_n$.

Let $f: X \to \S T$, $x \in X_n$ and $t \in \nb^n$, then
\begin{align*}
\hat{\S\psi \circ f}(x,t) &= (\S\psi \circ f)_n(x)(t) \\
&= (\S\psi_n \circ f_n)(x)(t) \\
&= \S\psi_n(f_n(x))(t) \\
&= (\psi \circ f_n(x))(t), 
&& \text{by definition of } \S\psi \\
&= (\psi \circ \hat f)(x)(t), 
&& \text{by definition of } \hat f.
\end{align*}

\end{enumerate}

\end{exercise}


\begin{exercise}[2]

\end{exercise}


\begin{exercise}[3]



\end{exercise}

\end{document}
