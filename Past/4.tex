\documentclass[a4paper,11pt,english]{article}
\usepackage{.styles/basic}
\usepackage{.styles/envs}

\usepackage{circuitikz}

%%%%%%% Title %%%%%%%%%%%%%%%%%%%%%%%%%%%
\title{\textbf{Algebraic Topology} - Exercise Sheet 4}
\author{Tor Gjone (2503108) \& Michele Lorenzi (3461634)}

%%%%%%% Definitions %%%%%%%%%%%%%%%%%%%%%

% Michele's stuff
% (I hope I don't break something...)
% (Most of the command are thought for livetexing but some became habits)
% ( :D don't worry. I'm a huge fan of shortcuts myself )

\makeatletter 
% \ni was used in the definitions of other commands. This is a solution
\newcommand{\@ni}{{n-1}}

\newcommand{\pin}{\pi_n(X,x_0)}
\newcommand{\pini}{\pi_\@ni(X,x_0)}
\newcommand{\pinr}{\pi_n(X,A,x_0)}
\newcommand{\pinir}{\pi_\@ni(X,A,x_0)}
\newcommand{\hn}{H_n(X;\Z)}
\newcommand{\hni}{H_\@ni(X;\Z)}
\newcommand{\hnr}{H_n(X,A;\Z)}
\newcommand{\hnir}{H_\@ni(X,A;\Z)}
\newcommand{\dn}{D^n}
\newcommand{\dni}{D^\@ni}
\newcommand{\sn}{S^n}
\newcommand{\sni}{S^{n-1}}
\newcommand{\sph}{(D^n,\de D^n)}
\newcommand{\sphi}{(D^\@ni,\de D^\@ni)}
\newcommand{\sm}{\smallsetminus}
\newcommand{\til}{\tilde}
\newcommand{\fa}{\;\;\forall}
\newcommand{\ul}[1]{\,\underline{#1}\,}
\newcommand{\ol}{\overline}
\newcommand{\nf}{\normalfont}
\newcommand{\xs}{{x_1,\dots,x_n}}
\newcommand{\xso}{{x_0,\dots,x_n}}
\newcommand{\nn}[2]{{{#1}_1,\dots,{#1}_{#2}}}
\newcommand{\nno}[2]{{{#1}_0,\dots,{#1}_{#2}}}
\newcommand{\nns}[3]{{{#1}_1\,{#2}\,\dots\,{#2}\,{#1}_{#3}}}
\newcommand{\nnso}[3]{{{#1}_0\,{#2}\,\dots\,{#2}\,{#1}_{#3}}}
\newcommand{\cb}[1]{\{#1\}}
\newcommand{\de}{\partial}
%\newcommand{\xto}[1]{\xrightarrow{#1}}
\newcommand{\hto}{\hookrightarrow}
\newcommand{\nii}{\@ni}
\newcommand{\cont}{\reflectbox{$\in$}}
\newcommand{\mi}{{m-1}}
\newcommand{\ki}{{k-1}}
\DeclareMathOperator{\colim}{colim}
\DeclareMathOperator{\tel}{tel}
\DeclareMathOperator{\proj}{proj}


% Tor's stuff
\newcommand{\oc}[1]{\overset{\circ}{#1}}
\newcommand{\into}{\hookrightarrow}
\newcommand{\nb}{\nabla}
\renewcommand{\hat}{\widehat}

\usepackage{scalerel,stackengine}
\stackMath
\renewcommand\widehat[1]{%
\savestack{\tmpbox}{\stretchto{%
  \scaleto{%
    \scalerel*[\widthof{\ensuremath{#1}}]{\kern.1pt\mathchar"0362\kern.1pt}%
    {\rule{0ex}{\textheight}}%WIDTH-LIMITED CIRCUMFLEX
  }{\textheight}% 
}{2.4ex}}%
\stackon[-6.9pt]{#1}{\tmpbox}%
}


\newcommand{\diagSquare}[8]{
\begin{tikzpicture}[node distance=2cm, auto]
\node (a)              { $ #5 $ };
\node (b) [right of=a] { $ #6 $ };
\node (c) [below of=a] { $ #7 $ };
\node (d) [right of=c] { $ #8 $ };
\draw[-to] (a) to node { $ #1 $ } (b);
\draw[-to] (a) to node { $ #2 $ } (c);
\draw[-to] (b) to node { $ #3 $ } (d);
\draw[-to] (c) to node { $ #4 $ } (d);
\end{tikzpicture}
}

\newcommand{\congto}{\xrightarrow{\cong}}
\newcommand{\incto}[1][]{ \xhookrightarrow{#1} }
\newcommand{\xto}[1]{\xrightarrow{#1}}

% Text key words
\newcommand{\tif}{\text{if }}
\newcommand{\tand}{\text{and }}
\newcommand{\tsince}{\text{since }}


% Projective spaces
\def\RP{\mathbb{RP}}
\def\CP{\mathbb{CP}}
\def\HP{\mathbb{HP}}

% Common Categories
\DeclareMathOperator{\Top}   {\bf Top}
\DeclareMathOperator{\Ab}    {\bf Ab}
\DeclareMathOperator{\Cat}   {\bf Cat}
\DeclareMathOperator{\CAT}   {\bf CAT}
\DeclareMathOperator{\Mod}   {\bf Mod}
\DeclareMathOperator{\Ring}  {\bf Ring}
\DeclareMathOperator{\Group} {\bf Group}
\DeclareMathOperator{\sSet}  {{\bf sSet}}

% Homological Algebra
\DeclareMathOperator{\Hom}   {Hom}
\DeclareMathOperator{\Tor}   {Tor}
\DeclareMathOperator{\Ext}   {Ext}

% Common Lie Groups 
\DeclareMathOperator{\GL}    {GL}
\DeclareMathOperator{\SU}    {SU}
\DeclareMathOperator{\U}     {U}
\DeclareMathOperator{\Sp}    {Sp}

% Maths Operators
\DeclareMathOperator{\pr}    {pr}
\DeclareMathOperator{\id}    {id}
\DeclareMathOperator{\Ker}   {Ker}
\DeclareMathOperator{\im}    {Im}

% Standard sets
\def \N {\mathbb{N}}
\def \Z {\mathbb{Z}}
\def \Q {\mathbb{Q}}
\def \R {\mathbb{R}}
\def \C {\mathbb{C}}
\def \H {\mathbb{H}}

\def \E {\mathbb{E}}
\def \Z {\mathbb{Z}}
\def \I {\mathbb{I}}
\def \J {\mathbb{J}}

% Vector calculus
\newcommand{\dif}[3][]{
	\ensuremath{\frac{d^{#1} {#2}}{d {#3}^{#1}}}}
\newcommand{\pdif}[3][]{
	\ensuremath{\frac{\partial^{#1} {#2}}{\partial {#3}^{#1}}}}

% Vectors and matrices
\newcommand{\mat}[1]{\begin{matrix} #1 \end{matrix}}
\newcommand{\pmat}[1]{\begin{pmatrix} #1 \end{pmatrix}}
\newcommand{\bmat}[1]{\begin{bmatrix} #1 \end{bmatrix}}

% Add space around the argument
\newcommand{\qq}[1]{\quad#1\quad}
\newcommand{\q}[1]{\:\:#1\:\:}

% Implications
\newcommand{\la} {\ensuremath{\Longleftarrow}}
\newcommand{\ra} {\ensuremath{\Longrightarrow}}
\newcommand{\lra}{\ensuremath{\Longleftrightarrow}}

\newcommand{\pwf}[1]{\begin{cases} #1 \end{cases}}

% Shorthand
\newcommand{\vphi}{\varphi}
\newcommand{\veps}{\varepsilon}

\newcommand{\<}[1]{\langle #1 \rangle}

% Notation
\newcommand{\wddef}[1]{\underline{#1}}
\newcommand{\pref}[1]{(\ref{#1})}

% Maths Operators
\theoremstyle{plain}
\theoremstyle{definition}
\newtheorem{thrm}{Theorem}[section]
\newtheorem{prop}[thrm]{Proposition}
\newtheorem{corol}[thrm]{Corollary}
\newtheorem{lemma}[thrm]{Lemma}

\newtheorem{defn}[thrm]{Definition}
\newtheorem{exmp}[thrm]{Example}
\newtheorem{clame}[thrm]{Clame}

\theoremstyle{remark}
% \newtheorem{remark}[thrm]{\normalfont\large\textit Remark}
\newtheorem{remark}[thrm]{Remark}
\newtheorem{note}[thrm]{Note}


\newSimpleHeaderEnvironment{exercise}{Exercise }

\setlength{\parindent}{0cm}
\setlength{\parskip}{3pt}

\renewcommand{\S}[1]{\mathcal{S}(#1)}
%%%%%%%% Content %%%%%%%%%%%%%%%%%%%%%%%%
\begin{document}

\mmaketitle
\begin{exercise}[1]
\begin{enumerate}
\item[(a)]

\item[(b)]
The fact that the two boundary points are mapped to the base point follows from
\[ d_0(g) = \{1\} = d_1(g). \]

Let $\delta_i: [n-1] \to [n]$, be the face map such that $\delta_i^* = d_i$.
Then
\[ (BG)_1 \times \nb^1 \ni (g,0) = (g,{\delta_0}_*(1)) \sim (d_0(g),1) = (1,1)
\in (BG)_0\times \nb^1, \]
so $(g,1) \in \{g\}\times \nb^1$ is mapped to $(g,0) = (1,1)$ in $|BG|$.
Similarly
\[ (g,1) = (g,(\delta_1)_*(1)) \sim (d_1(g),1) = (1,1), \]
so also $(g,1) \in \{g\}\times \nb^1$ is mapped to $(g,1) = (1,1)$ in $|BG|$.

\item[(c)]
Let $g,h \in G$ and $\eta(g,h) : \nb^2 \to |BG|$ defined by the composition
\[ \{ (g,h) \}\times \nb^2 \into \bigcup_{n\ge 0} (BG)_n \times \nb^n \onto
|BG|. \]

Like in (b), let $d_i = \delta_i^*$. Then
\[
\delta_0; \pwf{0\mapsto 1 \\ 1 \mapsto 2} \quad
\delta_1; \pwf{0\mapsto 0 \\ 1 \mapsto 2} \quad
\delta_2; \pwf{0\mapsto 0 \\ 1 \mapsto 1}
\]

So for $t\in \nb^2$, we have
\[ \omega(g)(t) = (g,t) = (d_0(g,h), t) \sim
((g,h),(\delta_0)_*(t)) = (\eta(g,h) (\delta_0)_*)(t), \]
and similarly
\[ \omega(h)(t) = (\eta(g,h) (\delta_2)_*)(t) \qq\tand
\omega(g\cdot h)(t) = (\eta(g,h) (\delta_1)_*)(t), \]
So $\omega$ on $g,h$ and $gh$ correspond to the loops defined $\eta$ restricted
to the three faces as illustrated in figure \pref{fig:1}

\begin{figure}
\caption{}
\label{fig:1}
\centering
\begin{tikzpicture}
\node[above] (c) at (1,1.5) {$2$};
\node[left]  (a) at (0,0) {$0$};
\node[right] (b) at (2,0) {$1$};
\draw (0,0) -- (2,0)   node[sloped, scale=.5, pos=0.5] {$>$}
                       node[sloped, scale=1, pos=0.5, below] {$h$};
\draw (0,0) -- (1,1.5) node[sloped, scale=.5, pos=0.5] {$>$} 
                       node[scale=1, pos=0.5, above left] {$gh$};
\draw (2,0) -- (1,1.5) node[sloped, scale=.6, pos=0.5, xscale=-1] {$>$} 
                       node[scale=1, pos=0.5, above right] {$g$};

% a=.488, t=.2,.4,.6,.8: (2-(cos(a)*2)*t*cos(a), (cos(a)*2)*t*sin(a))
\draw [cyan] plot [smooth] coordinates { (0,0) (1.6879319857569757, 0.1656525284552739)  (1,1.5)};
\draw [cyan] plot [smooth] coordinates { (0,0) (1.3758639715139513, 0.3313050569105478)  (1,1.5)};
\draw [cyan] plot [smooth] coordinates { (0,0) (1.0637959572709272, 0.49695758536582163) (1,1.5)};
\draw [cyan] plot [smooth] coordinates { (0,0) (0.7517279430279029, 0.6626101138210956)  (1,1.5)};

% t=.3 to t=.78
\draw[cyan, -to] (1.5318979786354636, 0.24847879268291082) --
(0.7829347444522055, 0.6460448609755681);
\end{tikzpicture}

\end{figure}

Also illustrated in the figure, by the cyan curves, $\eta(g,h)$ defines a
homotopy from $\omega(g)\cdot \omega(h)$ to $\omega(g\cdot h)$.
\end{enumerate}
\end{exercise}

\begin{exercise}[2]
\end{exercise}


\begin{exercise}[3]
\begin{enumerate}
\item[(a)]
Consider $S^1 = [0,1] / (0\sim 1)$. And let $U_1 = (0,1) \ss S^1$ and $U_2 =
[0,1] \setminus \{1/2\} \ss S^1$. Then $\{ U_i \}$ is a cover of $S^1$ and we claim that
$p^{-1}(U_i) \cong F \times U_i$.

In the case of $i=1$ this is clear and for $i=2$ we have
\begin{align*}
p\i (U_2) &\cong \left( F\times [0,1/2) \sqcup F \times (1/2,1] \right) /
\left( (x,0) \sim (f(x),1) \right) \\
&\cong_\phi \left( F\times [0,1/2) \sqcup F \times (1/2,1] \right) /
\left( (x,0) \sim (x,1) \right) \\
&\cong F \times U_2
\end{align*}
where $\phi = \id_F \times \id \sqcup f^{-1} \times \id$. $\phi$ is well-defined
since $\id(x) = x = f^{-1}(f(x))$ and it is clearly a homeomorphism.

\item[(b)]
\begin{lemma}
Let $\pi: E \to [0,1]$ be a fibre bundle, the $\pi$ is trivial. That is
$E \cong F \times [0,1]$. (Here $F$ is the unique fibre. Unique since $[0,1]$ is
connected.)
\end{lemma}


%\newcommand{\Bes}{[s-\eps,s+\eps]}
%\newcommand{\Bes}{\bar B(s,\eps)}
\newcommand{\Bes}{I_{s,\eps}}

\begin{proof}
Let $S = \{s\in[0,1] \q| \pi\i([0,s]) \cong F \times [0,s] \}$. We want to show
that $S$ is both closed and open and thus equal to $[0,1]$.

We start with open. Let $s \in S$, then since $\pi$ is a fiblre bundle there
exists an open neigboorhood $V\ni s$, such that $\pi\i(V) \cong F \times V$.
Let $\eps > 0$, such that $\Bes \ss V$ (closed ball centred at $s$ with
radius $\eps$, ie. $\Bes = [s-\eps, s+\eps]$.)

Since $s \in S$, there exists $f_s: \pi\i([0,s]) \congto F\times [0,s]$. And by
restricting the homeomorphism assosiated to $V$, we have $g: \pi\i(\Bes) \congto F
\times \Bes$.

We want to make $g$ and $f$ agree on the intersection, so we define a map to
make up for the difference. Define $\til g: F\times \Bes \to F\times\Bes$ by
\[ \til g(f,t) := \pwf{
(f_t \circ g_t\i)(f) &\tif t\in[s-\eps, s], \\
(f_s \circ g_s\i)(f) &\tif t\in[s, s+\eps] } \]
where $f_t : F \to F$ is the map $f$ resticted to the fiber over $t$.
$\til g$ is clearly well-defined and continues, since $g$ is a homeomorphim
m and
both $g$ and $f$ maps the fiber over $t$ to itself, by the definition of a fiber
bundle.

We observe that $\til g\circ g|_{\pi\i([s-t,s])} = f|_{\pi\i([s-t,s])}$. So
the following map is well defined. Define $\til f: \pi\i([0,s+\eps]) \to F\times
[0,s+\eps]$ by
\[ x \mapsto \pwf{
f(x) &\tif \pi(x) \in [0,s], \\
(\til g \circ g)(x) &\tif \pi(x) \in [s-\eps,s+\eps] } \]

It is quite clear that this defines a homeomorphism, so $s + \eps \in S$ and
thus $S$ is open.
% This clearly defines a homeomorphism on the fibers and by our observation above
% it is well-defined. It is also clear that

For the closed case, suppose $(s_i)_{i\in \N}$ is a sequence in $S$ converging
to $s \in [0,1]$. Let $\eps > 0$ such that $\pi\i(\Bes)$ is trivielizable (by
the same argument as above), and $g : \pi\i(\Bes) \congto F \times \Bes$. Since
$(s_i)_{i\in \N}$ converges to $s$ there exits $n \in \N$, such that $s_n \in
\pi\i(\Bes)$. Like above, we want to extend the trivialization over $[0,s_n]$ to
$[0,s+\eps]$.
% , by modifying one of the trivializations to make them agree on the inteersection and gluing them together. 
The same construction works also in this case, so we may conclude that
$s+\eps\in S$, and thus also $s\in S$. Hence $S$ is also closed.
\end{proof}

Let $p : E \to S^1$ be a fibre bundle with fibre $F$ ($S^1$ is connected), and
$U_1 = [0, 1/2] \ss S^1$ and $U_2 = ([1/2,1]) \ss S^1$. Then, for $i=1,2$, 
$p_i = p|_{p\i(U_i)} : p\i(U_i) \to U_i$ defines fibre bundle over the closed
interval $U_i$. So by the lemma above, $p_i$ is trivial. Let $f_i : p\i(U_i)
\congto F \times U_i$.

The two intervals overlap on exactly on the points $0 \sim 1$ and $1/2$. We define
the difference maps $\xi_1, \xi_2: F \to F$ by
\[ \xi_1 = f_2 \circ f_1\i(-,0) \qq\tand
\xi_2 = f_1 \circ f_2\i(-,1/2). \]

Clearly both $\xi_1$ and $\xi_2$ defines homeomorphisms. We want to modify $f_2$
such that it agrees with $f_1$ over the point $1/2$.
Let $\til f = (\xi_2 \times \id) \circ f_2$. Then by constuction $\til f$ and
$f_1$ agrees over the point $1/2$. Furthermore $f_2 = \xi_1 \circ (f_1)_0$
(where $\cdot_0$ means over the point $0\sim 1$), so
\[ (\til f)_0 = \xi_2 \circ \xi_1 \circ (f_1)_0. \]
Let $\xi = \xi_2 \circ \xi_1$, then $E \cong T_\xi$, by the map $f : E \congto
T_\xi$ defined by the compositon
\[ E \xto{ f_1 \sqcup_{1/2} \til f_2 } F \times [0,1]
\xto{quationt} T_\xi. \]
\end{enumerate}
\end{exercise}

\end{document}
